\documentclass[10pt,class=article,crop=false]{standalone}
\usepackage{../exo7formules}


\begin{document}
	
	
%%%%%%%%%%%%%%%%%%%%%%%%%%%%%%%%%%%%%%%%%%
\section{Arithmétique}

\begin{multicols}{2}
	
	
%-----------------------------------------
\subsection{Division euclidienne et pgcd}

Soient $a,b \in \Zz$. On dit que $b$ \defi{divise}\index{divisibilite@divisibilité} $a$ et on note $b | a$ s'il existe $q \in \Zz$ tel que $a = bq$.

\begin{theoreme}[Division euclidienne]
	\index{division euclidienne}
	Soit $a\in \Zz$ et $b\in \Nn\setminus \{0\}$.
	Il \evidence{existe} des entiers $q,r \in \Zz$ tels que
	\mybox{$a=bq+r \qquad \text{ et } \qquad  0 \le r < b$}
	De plus $q$ et $r$ sont \evidence{uniques}.
\end{theoreme}

Terminologie : $q$ est le \defi{quotient}\index{quotient} et $r$ est le \defi{reste}\index{reste}.


Nous avons donc l'équivalence : $r=0$ si et seulement si $b$ divise $a$.



%---------------------------------------------------------------
\textbf{Pgcd de deux entiers}

	Soient $a,b\in\Zz$ deux entiers, non tous les deux nuls.
	Le plus grand entier qui divise à la fois $a$ et $b$ s'appelle
	le \defi{plus grand diviseur commun}\index{pgcd} de $a$, $b$
	et se note $\pgcd(a,b)$.
	


%---------------------------------------------------------------
\textbf{Algorithme d'Euclide}

\begin{lemme}
	\label{lem:algoeuclide}
	Soient $a,b \in \Nn^*$. \'Ecrivons la division euclidienne $a=bq+r$.
	Alors
	\mybox{$\pgcd(a,b)=\pgcd(b,r)$}
\end{lemme}



\textbf{Algorithme d'Euclide.}
\index{algorithme d'Euclide}

On souhaite calculer le pgcd de $a,b \in \Nn^*$. On peut supposer $a \ge b$.
On calcule des divisions euclidiennes successives. Le pgcd sera le dernier reste
non nul :
\begin{itemize}
	\item division de $a$ par $b$, $a=bq_1+r_1$. Par le lemme, $\pgcd(a,b)=\pgcd(b,r_1)$ et
	si $r_1=0$ alors $\pgcd(a,b)=b$ sinon on continue :
	\item $b=r_1 q_2 + r_2$,  \quad $\pgcd(a,b)= \pgcd(b,r_1)=\pgcd(r_1,r_2)$,
	\item $r_1 = r_2 q_3 + r_3$, \quad $\pgcd(a,b)=\pgcd(r_2,r_3)$,
	\item \ldots
\end{itemize}


%---------------------------------------------------------------
\textbf{Nombres premiers entre eux}

	Deux entiers $a,b$ sont
	\defi{premiers entre eux}\index{nombre!premiers entre eux}
	si $\pgcd(a,b)=1$.



Si deux entiers $a,b \in \Zz$ ne sont pas premiers entre eux, on peut s'y ramener en divisant par $d = \pgcd(a,b)$.
$$\left\{\begin{array}{ll} a &= \ a'd \\ b &=\  b'd \\ \end{array} \quad \text{ avec }\quad  a',b' \in \Zz \text{ et } \pgcd(a',b')=1 \right.$$


%-----------------------------------------
\subsection{Théorème de Bézout}

%---------------------------------------------------------------
%\subsection{Théorème de Bézout}

\begin{theoreme}[Théorème de Bézout]
	\index{theoreme@théorème!de Bézout}
	Soient $a,b$ des entiers. Il existe
	des entiers $u,v \in \Zz$ tels que
	\mybox{$au+bv=\pgcd(a,b)$}
\end{theoreme}

Les entiers $u,v$ sont des \defi{coefficients de Bézout}\index{coefficients de Bezout@coefficients de Bézout}.
Ils s'obtiennent en \og remontant \fg{} l'algorithme d'Euclide.


%---------------------------------------------------------------
%\subsection{Corollaires du théorème de Bézout}

\begin{corollaire}
	Si $d|a$ et $d|b$ alors $d | \pgcd(a,b)$.
\end{corollaire}



\begin{corollaire}
	Soient $a, b$ deux entiers. $a$ et $b$ sont premiers entre eux
	\textbf{si et seulement si} il existe $u,v \in \Zz$ tels que
	\mybox{$au+bv=1$}
\end{corollaire}

Remarque. Si on trouve deux entiers $u',v'$ tels que $au'+bv'=d$, cela n'implique
	\textbf{pas} que $d=\pgcd(a,b)$. On sait seulement alors que $\pgcd(a,b)|d$.
	
\begin{corollaire}[Lemme de Gauss]
	\index{lemme!de Gauss}
	Soient $a,b,c \in \Zz$.
	\mybox{Si \ \ $a | bc$ \ \  et \ \  $\pgcd(a,b)=1$ \ \  alors \ \  $a|c$}
\end{corollaire}


%---------------------------------------------------------------
\textbf{\'Equations $ax+by=c$}
%\label{ssec:dioph}

\begin{proposition}
	\label{prop:dioph}
	Considérons l'équation
	\begin{equation}
		\label{eq:bezout}
		\tag{E}
		ax+by=c
	\end{equation}
	où $a,b,c \in \Zz$.
	\begin{enumerate}
		\item L'équation (\ref{eq:bezout}) possède des solutions $(x,y)\in \Zz^2$ si et seulement si
		$\pgcd(a,b) | c$.
		\item Si $\pgcd(a,b) | c$ alors il existe même une infinité de solutions entières et elles sont exactement les
		$(x,y) = (x_0+ \alpha k, y_0 + \beta k)$ avec $x_0,y_0,\alpha,\beta \in \Zz$ fixés et $k$ parcourant $\Zz$.
	\end{enumerate}
\end{proposition}


%---------------------------------------------------------------
\textbf{ppcm}


	Le $\text{ppcm}(a,b)$ (\defi{plus petit multiple commun}\index{ppcm})
	est le plus petit entier $\ge 0$ divisible par $a$ et par $b$.


\begin{proposition}
	Si $a,b$ sont des entiers (non tous les deux nuls) alors
	\mybox{$\pgcd(a,b) \times \text{ppcm}(a,b) = |ab|$}
\end{proposition}

\begin{proposition}
	Si $a|c$ et $b|c$ alors $\text{ppcm}(a,b)|c$.
\end{proposition}



%-----------------------------------------
\subsection{Nombres premiers}

	Un \defi{nombre premier}\index{nombre!premier} $p$ est un entier $\ge 2$ dont les seuls diviseurs
	positifs sont $1$ et $p$.


\begin{proposition}
	Il existe une infinité de nombres premiers.
\end{proposition}


Remarque. Si un nombre $n$ n'est pas premier alors un de ses facteurs est $\le \sqrt{n}$.

\begin{proposition}[Lemme d'Euclide]
	\index{lemme!d'Euclide}
	Soit $p$ un nombre premier.
	Si $p | ab$ alors $p|a$ ou $p | b$.
\end{proposition}

\begin{theoreme}[Décomposition en facteurs premiers]
	Soit $n \ge 2$ un entier. Il existe des nombres premiers $p_1 < p_2 < \cdots < p_r$
	et des exposants entiers $\alpha_1, \alpha_2, \dots, \alpha_r \ge 1$ tels que :
	$$n = p_1^{\alpha_1} \times p_2^{\alpha_2} \times \cdots \times p_r^{\alpha_r}.$$
	De plus les $p_i$ et les $\alpha_i$ ($i=1,\ldots,r$) sont uniques.
\end{theoreme}


%-----------------------------------------
\subsection{Congruences}


	Soit $n \ge 2$ un entier. On dit que $a$ est \defi{congru}\index{congruence} à $b$ \defi{modulo}\index{modulo} $n$,
	si $n$ divise $b-a$.
	On note alors
	$$a \equiv b \pmod n.$$


On note aussi parfois $a=b \pmod n$ ou $a \equiv b [n]$.
Une autre formulation est
\mybox{$a \equiv b \pmod n \quad \iff \quad \exists k \in \Zz \quad a = b +kn.$}

Remarquez que $n$ divise $a$ si et seulement si $a\equiv 0 \pmod n$.

\begin{proposition}
	\sauteligne
	\begin{enumerate}
		\item La relation \og congru modulo $n$ \fg{} est une relation d'équivalence :
		\begin{itemize}
			\item (Réflexivité) $a \equiv a \pmod n$,
			\item (Symétrie) si $a \equiv b \pmod n$ alors $b \equiv a \pmod n$,
			\item (Transitivité) si $a \equiv b \pmod n$ et $b \equiv c \pmod n$ alors $a \equiv c \pmod n$.
		\end{itemize}
		\item Si $a \equiv b \pmod n$ et $c \equiv d \pmod n$ alors
		$a+c \equiv b+d \pmod n$.
		\item Si $a \equiv b \pmod n$ et $c \equiv d \pmod n$ alors
		$a\times c \equiv b \times d \pmod n$.
		\item Si $a \equiv b \pmod n$ alors pour tout $k \ge 0$, $a^k \equiv b^k \pmod n$.
	\end{enumerate}
\end{proposition}



%---------------------------------------------------------------
\textbf{\'Equation de congruence $ax \equiv b \pmod n$}

\begin{proposition}
	Soit $a \in \Zz^*$, $b \in \Zz$ fixés et $n \ge 2$.
	Considérons l'équation
	\myboxinline{$ax \equiv b \pmod n$} d'inconnue $x \in \Zz$ :
	\begin{enumerate}
		\item Il existe des solutions si et seulement si $\pgcd(a,n) | b$.
		\item Les solutions sont de la forme $x = x_0 + \ell \frac{n}{\pgcd(a,n)}$, $\ell \in \Zz$
		où $x_0$ est une solution particulière. Il existe donc $\pgcd(a,n)$ classes de solutions.
	\end{enumerate}
\end{proposition}


%---------------------------------------------------------------
%\subsection{Petit théorème de Fermat}

\begin{theoreme}[Petit théorème de Fermat]
	\index{theoreme@théorème! de Fermat (petit)}
	Si $p$ est un nombre premier et $a \in \Zz$ alors
	\mybox{$a^p \equiv a \pmod p$}
\end{theoreme}

\begin{corollaire}
	Si $p$ ne divise pas $a$ alors
	\mybox{$a^{p-1} \equiv 1 \pmod p$}
\end{corollaire}



\end{multicols}

\end{document}



