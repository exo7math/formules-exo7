\documentclass[10pt,class=article,crop=false]{standalone}
\usepackage{../exo7formules}


\begin{document}
	
%%%%%%%%%%%%%%%%%%%%%%%%%%%%%%%%%%%%%%%%%%
\section{Déterminants}

\begin{multicols}{2}
	
$\Kk$ est un corps commutatif, par exemple $\Kk=\Rr$ ou $\Kk=\Cc$.

%-----------------------------------------
\subsection{Déterminant en dimension $2$ et $3$}

%-------------------------------------------------------
\textbf{Matrice $2 \times 2$.}

\begin{center}
\begin{minipage}{0.2\textwidth}
$$\det \begin{pmatrix}a&b\\c&d\end{pmatrix} = ad-bc$$
\end{minipage}
\begin{minipage}{0.2\textwidth}
\myfigure{2}{
	\tikzinput{fig_determinants01}
}	
\end{minipage}
\end{center}	

C'est donc le produit des éléments sur la diagonale principale
(\couleurnb{en bleu}{en gris foncé}) moins le produit des éléments sur l'autre diagonale (\couleurnb{en orange}{en gris clair}).

\begin{proposition}\label{prop:aire}
	L'aire du parallélogramme délimité par $v_1= \left(\begin{smallmatrix}a\\c\end{smallmatrix}\right)$ et
	$v_2= \left(\begin{smallmatrix}b\\d\end{smallmatrix}\right)$ est donnée par la valeur absolue du déterminant:
\begin{center}
\begin{minipage}{0.27\textwidth}	
	$$\mathcal{A}
	= \Big|\det(v_1,v_2)\Big|
	= \Big|\det
	\begin{pmatrix}
		a & b \\
		c & d
	\end{pmatrix}\Big|
	$$
\end{minipage}
\begin{minipage}{0.15\textwidth}
\myfigure{0.6}{
	\tikzinput{fig_determinants04}
}
	
\end{minipage}
\end{center}		
\end{proposition}


%-------------------------------------------------------
\textbf{Matrice $3 \times 3$.}
Soit $A \in M_3(\Kk)$ une matrice $3 \times 3$:
$$A = \begin{pmatrix}
	a_{11} & a_{12} & a_{13} \\
	a_{21} & a_{22} & a_{23} \\
	a_{31} & a_{32} & a_{33} \\
\end{pmatrix}.$$

Voici la formule pour le déterminant:
$$\det A =
a_{11} a_{22} a_{33}
+ a_{12} a_{23} a_{31}
+ a_{13} a_{21} a_{32}
- a_{31} a_{22} a_{13}
- a_{32} a_{23} a_{11}
- a_{33} a_{21} a_{12}$$

\defi{Règle de Sarrus}\index{regle@règle!de Sarrus} :
Addition de trois produits de trois termes le long de la diagonale descendante (\couleurnb{en bleu}{en gris foncé}, à gauche)
puis soustraction de trois produits de trois termes le long de la diagonale montante (\couleurnb{en orange}{en gris clair}, à droite)

\myfigure{1.5}{
	\tikzinput{fig_determinants02ter}
}


\begin{proposition}\label{prop:volume}
Le volume du parallélépipède 
délimité par trois vecteurs 
$v_1=\left(\begin{smallmatrix}a_{11}\\a_{21}\\a_{31}\end{smallmatrix}\right)$,
$v_2=\left(\begin{smallmatrix}a_{12}\\a_{22}\\a_{32}\end{smallmatrix}\right)$,
$v_3=\left(\begin{smallmatrix}a_{13}\\a_{23}\\a_{33}\end{smallmatrix}\right)$
est donné par la valeur absolue du déterminant de la matrice correspondante : $\mathcal{V} = \Big|\det(A)\Big|.$
\end{proposition}

\myfigure{0.6}{
	\tikzinput{fig_determinants05}
}


%-----------------------------------------
\subsection{Définition du déterminant}

\begin{theoreme}[Existence et d'unicité du déterminant]
	\label{th:def:determinant}
	Il existe une unique application de $M_n(\Kk)$ dans $\Kk$,
	appelée \defi{déterminant}\index{determinant@déterminant}, telle que
	\begin{itemize}
		\item[(i)] le déterminant est linéaire par rapport à chaque
		vecteur colonne, les autres étant fixés ;
		\item[(ii)] si une matrice $A$ a deux colonnes identiques,
		alors son déterminant est nul ;
		\item[(iii)] le déterminant de la matrice identité $I_n$ vaut $1$.
	\end{itemize}
\end{theoreme}

Le déterminant est donc la seule \defi{forme multilinéaire} (propriété (i)),   \defi{alternée} (propriété (ii)) qui prend comme valeur $1$ sur la matrice
$I_n$.

Si on note $C_{i}$ la $i$-ème colonne de $A$, alors
$$\det A=\left|\begin{matrix}
	C_1&C_2&\cdots&C_n
\end{matrix}\right|
= \det (C_1,C_2,\ldots,C_n) \,.
$$
Avec cette notation, la propriété (i) de linéarité par rapport à la colonne $j$ s'écrit:
pour tout $\lambda,\mu \in \Kk$,
$\det (C_1,\ldots,\lambda C_j + \mu C'_j,\ldots, C_n)
= \lambda  \det (C_1,\ldots,C_j,\ldots, C_n)+ \mu \det (C_1,\ldots,C_j',\ldots, C_n)$, soit



\begin{proposition}[Opérations élémentaires sur les colonnes]
\sauteligne
\begin{enumerate}
	\item $C_i \leftarrow \lambda C_i$ avec $\lambda \neq 0$:
	$A'$ est obtenue en multipliant une colonne de $A$ par un scalaire non nul.
	Alors $\det A' = \lambda \det A$.
	
	\item $C_i \leftarrow C_i+\lambda C_j$ avec $\lambda \in \Kk$ (et $j\neq i$):
	$A'$ est obtenue en ajoutant à une colonne de $A$ un multiple d'une autre colonne de $A$.
	Alors $\det A' = \det A$.
	
	\item $C_i \leftrightarrow C_j$: $A'$ est obtenue en échangeant
	deux colonnes distinctes de $A$. Alors \myboxinline{$\det A' = - \det A$}. \'Echanger deux colonnes change le signe du déterminant.
\end{enumerate}
\end{proposition}


\begin{corollaire} \label{coro:colonnes-liees}
Si une colonne $C_i$ de la matrice $A$ est combinaison linéaire des autres colonnes, alors $\det A=0$.
\end{corollaire}


\begin{proposition}
	\label{prop:dettriang}
	Le déterminant d'une matrice triangulaire supérieure (ou inférieure, ou diagonale)
	est égal au produit des termes diagonaux.
\end{proposition}


%-----------------------------------------
\subsection{Propriétés du déterminant}




%-------------------------------------------------------
\textbf{Déterminant d'un produit}


\begin{theoreme}~
	\mybox{$\det (AB)=\det A \cdot \det B$}
\end{theoreme}


%-------------------------------------------------------
\textbf{Déterminant des matrices inversibles}

\begin{corollaire}
	Une matrice carrée $A$ est inversible si et seulement si son
	déterminant est non nul. De plus si $A$ est inversible,
	alors:
	\mybox{$\displaystyle \det \big(A^{-1}\big)=\frac1{\det A}$}
\end{corollaire}



%-------------------------------------------------------
\textbf{Déterminant de la transposée}

\begin{corollaire}
	~
	\mybox{$\det \big(A^T\big)=\det A$}
\end{corollaire}

Conséquence. Par transposition, tout ce que l'on a
dit des déterminants à propos des colonnes est
vrai pour les lignes : le déterminant est multilinéaire
par rapport aux lignes ; si une matrice a deux lignes égales,
son déterminant est nul ; on ne modifie pas un déterminant en
ajoutant à une ligne une combinaison linéaire des autres lignes, etc.
	


%-----------------------------------------
\subsection{Calculs de déterminants}


\textbf{Cofacteur}

	Soit $A = \big( a_{ij}\big) \in M_n(\Kk)$ une matrice carrée.
	
	\begin{itemize}
		\item On note $A_{ij}$ la matrice extraite, obtenue en effaçant la ligne~$i$ et la colonne $j$ de $A$.
		\item Le nombre $\det A_{ij}$ est un \defi{mineur d'ordre $n-1$}\index{mineur} de la matrice $A$.
		\item Le nombre $C_{ij} = (-1)^{i+j}\det A_{ij}$ est le \defi{cofacteur}\index{cofacteur} de $A$
		relatif au coefficient $a_{ij}$.
	\end{itemize}


\myfigure{2}{
	\tikzinput{fig_determinants07}
}
\[
A_{ij} = \begin{pmatrix}
	a_{1,1} & \dots & a_{1,j-1} & a_{1,j+1} &\dots & a_{1,n}\\
	\vdots &&\vdots &\vdots &&\vdots \\
	a_{i-1,1} & \dots & a_{i-1,j-1} & a_{i-1,j+1} &\dots & a_{i-1,n}\\
	a_{i+1,1} & \dots & a_{i+1,j-1} & a_{i+1,j+1} &\dots & a_{i+1,n}\\
	\vdots &&\vdots &&&\vdots \\
	a_{n,1} & \dots & a_{n,j-1} & a_{n,j+1}& \dots & a_{n,n}
\end{pmatrix}
\]



Pour déterminer si $C_{ij} = +\det A_{ij} $ ou $C_{ij} = -\det A_{ij}$, on peut se souvenir
que l'on associe des signes en suivant le schéma d'un échiquier:
$$  A =
\begin{pmatrix}
	+ & - & + & - &\dots\\
	- & + & - & + &\dots \\
	+ & - & + & - &\dots\\
	\vdots & \vdots & \vdots & \vdots &
\end{pmatrix}
$$

\begin{theoreme}[Développement suivant une ligne ou une colonne]
	Formule de développement par rapport à la ligne $i$:
	$$\det A
	= \sum_{j=1}^n (-1)^{i+j} a_{ij} \det A_{ij}
	= \sum_{j=1}^n a_{ij} C_{ij}$$
	
	Formule de développement par rapport à la colonne $j$:
	$$\det A
	= \sum_{i=1}^n (-1)^{i+j} a_{ij} \det A_{ij}
	= \sum_{i=1}^n a_{ij}C_{ij}$$
\end{theoreme}

On commence souvent par faire apparaître
un maximum de zéros par des opérations élémentaires sur les lignes et/ou les colonnes
qui ne modifient pas le déterminant,
avant de développer le déterminant suivant la ligne ou la colonne qui a le plus de zéros.

%-------------------------------------------------------
\textbf{Inverse d'une matrice}

Soit $A \in M_n(\Kk)$ une matrice carrée.
La matrice $C$ des cofacteurs, appelée \defi{comatrice}\index{comatrice}, et notée
$\mathrm{Com}(A)$:
$$C = (C_{ij}) = \left(
\begin{array}{cccc}
	C_{11} & C_{12} & \cdots & C_{1n}\\
	C_{21} & C_{22} & \cdots & C_{2n}\\
	\vdots & \vdots & & \vdots\\
	C_{n1} & C_{n2} & \cdots & C_{nn}
\end{array}\right)
$$

\begin{theoreme}
	Soient $A$ une matrice inversible, et $C$ sa comatrice.
	On a alors
	\mybox{$\displaystyle A^{-1} = \frac{1}{\det A} \, C^T$}
\end{theoreme}





%-----------------------------------------
\subsection{Applications des déterminants}



%-------------------------------------------------------
\textbf{Méthode de Cramer}


Considérons le système d'équations linéaires à $n$ équations et $n$ inconnues suivant:
\[
\left\{
\begin{array}{ccc}
	a_{11} x_1 + a_{12} x_2 + \dots + a_{1n} x_{n} & = & b_1\\
	a_{21} x_1 + a_{22} x_2 + \dots + a_{2n} x_n & = & b_2\\
	\qquad\qquad \dots \qquad\qquad &  &\\
	a_{n1} x_1 + a_{n2} x_2 + \dots + a_{nn} x_n & = & b_n
\end{array}
\right.
\]
Ce système s'écrit sous forme matricielle $AX=B$ où
$$
A = \left(
\begin{array}{cccc}
	a_{11} & a_{12} & \cdots & a_{1n}\\
	a_{21} & a_{22} & \cdots & a_{2n}\\
	\vdots & \vdots && \vdots\\
	a_{n1} & a_{n2}& \cdots &a_{nn}
\end{array}\right) \in M_{n}(\Kk), 
\quad X= \left( \begin{array}{c}x_1\\x_2 \\\vdots \\x_n \end{array} \right) 
\quad B=\left( \begin{array}{c}b_1\\b_2 \\\vdots \\b_n \end{array} \right)
$$


Définissons la matrice $A_j \in M_{n}(\Kk)$ obtenue en remplaçant la $j$-ème
colonne de $A$ par le second membre $B$ :
$$
A_j = \left(
\begin{array}{ccccccc}
	a_{11} &  \dots & a_{1,j-1} & {\color{myred}b_1} & a_{1,j+1} & \dots & a_{1n}\\
	a_{21} & \dots & a_{2,j-1}& {\color{myred}b_2} & a_{2,j+1}& \dots & a_{2n}\\
	\vdots &  & \vdots & {\color{myred}\vdots} & \vdots& &\vdots\\
	a_{n1} &\dots & a_{n,j-1}& {\color{myred}b_n}& a_{n,j+1}& \dots & a_{nn}
\end{array}\right)
$$


\begin{theoreme}[Règle de Cramer] 
Soit $AX = B$ un système de $n$ équations  à $n$ inconnues. Supposons que $\det A \neq 0$.
Alors l'unique solution $(x_1,x_2,\ldots,x_n)$ du système est donnée par:
$$
x_1 = \frac{\det A_1}{\det A} \qquad x_2 = \frac{\det A_2}{\det A} \qquad \ldots \qquad x_n = \frac{\det A_n}{\det A}.
$$
\end{theoreme}





%-------------------------------------------------------
\textbf{Déterminant et base}


\begin{theoreme}
	Une famille de $n$ vecteurs de $\Rr^n$
	$$\begin{pmatrix}a_{11}\\a_{21}\\\vdots\\a_{n1}\end{pmatrix}
	\quad
	\begin{pmatrix}a_{12}\\a_{22}\\\vdots\\a_{n2}\end{pmatrix}
	\quad \cdots
	\quad
	\begin{pmatrix}a_{1n}\\a_{2n}\\\vdots\\a_{nn}\end{pmatrix}$$
	forme une base si et seulement si
	$\det \ (a_{ij}) \neq 0$.
\end{theoreme}


%-------------------------------------------------------
\textbf{Calcul du rang d'une matrice}

Soit $A=(a_{ij}) \in M_{n,p}(\Kk)$ une matrice à $n$ lignes
et $p$ colonnes à coefficients dans $\Kk$.
Le \defi{rang}\index{rang!d une matrice@d'une matrice} de $A$ est la dimension de l'espace vectoriel engendré par les vecteurs colonnes. C'est donc le nombre maximum de vecteurs colonnes linéairement indépendants.

Soit $k$ un entier inférieur à $n$ et à $p$.
On appelle \defi{mineur d'ordre $k$}\index{mineur}
le déterminant d'une matrice carrée de taille $k$ obtenue à partir
de $A$ en supprimant $n-k$ lignes et $p-k$ colonnes.


\begin{theoreme}
	\label{th:mineur}
	Le rang d'une matrice $A \in M_{n,p}(\Kk)$ est le plus grand entier $r$
	tel qu'il existe un mineur d'ordre~$r$ extrait de $A$ non nul.
\end{theoreme}

\begin{proposition}
	Le rang de $A$ est égal au rang de sa transposée $A^T$.
\end{proposition}



\end{multicols}
\end{document}



\end{document}


