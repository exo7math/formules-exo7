\documentclass[10pt,class=article,crop=false]{standalone}
\usepackage{../exo7formules}


\begin{document}

\newcommand{\assertion}[1]{{\og\emph{#1}\fg}} % pour chapitre logique

%%%%%%%%%%%%%%%%%%%%%%%%%%%%%%%%%%%%%%%%%%
\section{Ensembles et applications}

\begin{multicols}{2}
	

%-----------------------------------------
\subsection{Ensembles}


\begin{itemize}
	\item Un \defi{ensemble}\index{ensemble} est une collection d'éléments.

	\item L'\defi{ensemble vide}\index{ensemble vide}, $\varnothing$\index{$\varnothing$}
	est l'ensemble ne contenant aucun élément.
	\item On note
	\mybox{$x \in E$}
	\index{$\in$}
	si $x$ est un élément de $E$, et $x \notin E$\index{$\notin$} dans le cas contraire.
\end{itemize}




\begin{itemize}
	
	\item L'\defi{inclusion}\index{inclusion}. $E \subset F$\index{$\subset$} si tout élément de $E$ est aussi un élément de $F$.
	Autrement dit: $\forall x \in E \; (x \in F)$. On dit alors que $E$ est un \defi{sous-ensemble}\index{sous-ensemble} de $F$
	ou une \defi{partie}\index{partie} de $F$.
	
	\item L'\defi{égalité}. $E = F$ si et seulement si $E \subset F$ et $F \subset E$.
	
	\item \defi{Ensemble des parties} de $E$. On note $\mathcal{P}(E)$\index{$\mathcal{P}(E)$} l'ensemble des parties de $E$.
	Par exemple si $E= \{1,2,3\}$ :
	$$\mathcal{P}(\{1,2,3\}) =
	\big\{ \varnothing, \{1\}, \{2\}, \{3\}, \{1,2\}, \{1,3\}, \{2,3\}, \{1,2,3\} \big\}.$$
	
	\item \defi{Complémentaire}\index{complementaire@complémentaire}. Si $A \subset E$,
	\mybox{$\complement_E A = \big\{ x \in E \mid x \notin A \big\}$}
	On le note aussi $E \setminus A$\index{$E \setminus A$} et juste $\complement A$\index{$\complement$}
	s'il n'y a pas d'ambigu\"ité (et parfois aussi $A^c$ ou $\overline{A}$).
	
	
	\item \defi{Union}\index{union}\index{$\cup$}. Pour $A, B \subset E$,
	\mybox{$A \cup B = \big\{ x \in E \mid x \in A \text{ ou } x \in B \big\}$}
	Le \emph{\og ou \fg} n'est pas exclusif : $x$ peut appartenir à $A$ et à $B$ en même temps.

	
	\item \defi{Intersection}\index{intersection}\index{$\cap$}.
	\mybox{$A \cap B = \big\{ x \in E \mid x \in A \text{ et } x \in B \big\}$}
	
\end{itemize}


\myfigure{0.7}{
	\tikzinput{fig_ensembles01} \quad
	\tikzinput{fig_ensembles02} \quad
	\tikzinput{fig_ensembles03}
}
%---------------------------------------------------------------
\textbf{Règles de calculs}

Soient $A, B, C$ des parties d'un ensemble $E$.

\begin{itemize}
	\item $A\cap B = B \cap A$
	\item $A \cap (B \cap C) = (A \cap B) \cap C$
	\qquad (on peut écrire $A\cap B \cap C$ sans ambigüité)
	\item $A \cap \varnothing = \varnothing$, \quad $A \cap A= A$,  \quad $A \subset B \Longleftrightarrow A\cap B = A$
\end{itemize}

\medskip

\begin{itemize}
	\item $A\cup B = B \cup A$
	\item $A \cup (B \cup C) = (A \cup B) \cup C$
	\qquad (on peut écrire $A\cup B \cup C$ sans ambiguïté)
	\item $A \cup \varnothing = A$, \quad  $A \cup A= A$, \quad  $A \subset B \Longleftrightarrow A\cup B = B$
\end{itemize}

\medskip

\begin{itemize}
	\item \myboxinline{$A \cap (B \cup C) = (A \cap B) \cup (A \cap C)$}
	\item \myboxinline{$A \cup (B \cap C) = (A \cup B) \cap (A \cup C)$}
\end{itemize}

\medskip

\begin{itemize}
	\item $\complement \left( \complement A \right) = A$ \quad et donc
	\quad $A \subset B \Longleftrightarrow \complement B \subset \complement A$
	\item $\complement \left( A \cap B \right) = \complement A \cup \complement B$
	\item $\complement \left( A \cup B \right) = \complement A \cap \complement B$
\end{itemize}



%---------------------------------------------------------------
\textbf{Produit cartésien}

Soient $E$ et $F$ deux ensembles.
Le \defi{produit cartésien}\index{produit cartesien@produit cartésien}, noté $E \times F$, est l'ensemble des couples $(x,y)$ où $x \in E$ et $y \in F$.



%-----------------------------------------
\subsection{Applications}

\begin{itemize}
	\item Une \defi{application}\index{application} (ou une \defi{fonction}\index{fonction}) $f : E \to F$,
	c'est la donnée pour chaque élément $x \in E$ d'un unique élément de $F$ noté $f(x)$.
	
	

	
	\myfigure{0.6}{
		\tikzinput{fig_ensembles07a}\qquad 
		\tikzinput{fig_ensembles07b}
	}
	
	\item \defi{Égalité}. Deux applications $f,g : E \to F$ sont égales si et seulement
	si pour tout $x \in E$, $f(x)=g(x)$. On note alors $f=g$.
	
	\item Le \defi{graphe}\index{graphe} de $f : E \to F$ est
	
	\begin{minipage}{0.22\textwidth}
	\mybox{$\Gamma_f = \Big\{ \big(x,f(x)\big) \in E \times F  \mid x \in E \Big\}$}
	\end{minipage}
	\begin{minipage}{0.25\textwidth}	
	\myfigure{0.6}{
		\tikzinput{fig_ensembles08}
	}
	\end{minipage}
	\item \defi{Composition}\index{composition}. Soient $f : E \to F$ et $g : F \to G$ alors $g \circ f : E \to G$ est l'application définie par $g \circ f (x) = g\big( f(x) \big)$.
	$$
	\begin{tikzpicture}[node distance=2cm, every edge/.style={draw,thick,->},scale=0.7, transform shape]
		\node(E){$E$};
		\node[right of=E](F){$F$};
		\node[right of=F](G){$G$};
		\draw (E) edge (F);\draw (E) edge[myred, bend left=45] node[above]{$f$} (F);
		\draw (F) edge (G);\draw (F) edge[myred, bend left=45] node[above]{$g$} (G);
		\draw (E) edge[myred, bend right] node[below]{$g\circ f$} (G);
	\end{tikzpicture}
	$$
	
	\item \defi{Restriction}\index{restriction}. Soient $f : E \to F$ et $A\subset E$ alors la restriction de $f$ à $A$ est l'application 
	$\begin {array}{rccc}
	f_{|_A}\ : \ & A &\longrightarrow& F \\
	&x & \longmapsto & f(x)\\
\end{array}$.


\end{itemize}



%---------------------------------------------------------------

\textbf{Image directe, image réciproque}

Soient $E, F$ deux ensembles.

\begin{definition}
Soit $A \subset E$ et $f : E \to F$, l'\defi{image directe}\index{image directe} de $A$ par $f$ est l'ensemble
\mybox{$f(A) = \big\{ f(x) \mid x \in A \big\}$}
\end{definition}
\myfigure{0.6}{
\tikzinput{fig_ensembles09a}
\qquad \qquad
\tikzinput{fig_ensembles09b}
}

\begin{definition}
Soit $B \subset F$ et $f : E \to F$, l'\defi{image réciproque}\index{image reciproque@image réciproque} de $B$ par $f$ est l'ensemble
\mybox{$f^{-1}(B) = \big\{ x \in E \mid f(x) \in B \big\}$}
\end{definition}

\myfigure{0.6}{
\tikzinput{fig_ensembles10a}
\qquad \qquad
\tikzinput{fig_ensembles10b}
}


Fixons $y \in F$. Tout élément $x\in E$ tel que $f(x)=y$ est un \defi{antécédent}\index{antecedent@antécédent}
de $y$.
En termes d'image réciproque l'ensemble des antécédents de $y$ est $f^{-1}(\{y\})$.




%-----------------------------------------
\subsection{Injection, surjection, bijection}


%---------------------------------------------------------------
\textbf{Injection, surjection}

Soit $E, F$ deux ensembles et $f : E \to F$ une application.

\begin{definition}
	$f$ est \defi{injective}\index{injection} si pour tout $x,x' \in E$ avec $f(x)=f(x')$ alors $x=x'$.
	Autrement dit :
	\mybox{$\forall x, x' \in E \quad \big( f(x)=f(x') \implies x=x'\big)$}
\end{definition}



Les applications $f$ représentées sont injectives :
\myfigure{0.6}{
	\tikzinput{fig_ensembles11a}
	\qquad \qquad
	\tikzinput{fig_ensembles11b}
}


\begin{definition}
	$f$ est \defi{surjective}\index{surjection} si pour tout $y \in F$, il existe $x \in E$ tel que $y=f(x)$.
	Autrement dit :
	\mybox{$\forall y \in F \quad \exists x \in E \quad \big( y = f(x) \big)$}
\end{definition}

Une autre formulation : $f$ est surjective si et seulement si $f(E)=F$.

Les applications $f$ représentées sont surjectives :
\myfigure{0.6}{
	\tikzinput{fig_ensembles11c}
	\qquad
	\tikzinput{fig_ensembles11d}
}


\begin{itemize}
	\item $f$ est injective si et seulement si tout élément $y$ de $F$ a \emph{au plus} un antécédent (et éventuellement aucun).
	\item $f$ est surjective si et seulement si tout élément $y$ de $F$ a \emph{au moins} un antécédent.
\end{itemize}



%---------------------------------------------------------------
\textbf{Bijection}

\begin{definition}
	$f$ est \defi{bijective}\index{bijection} si elle est injective et surjective. Cela équivaut à :
	pour tout $y \in F$ il existe un unique $x \in E$ tel que $y=f(x)$.
	Autrement dit :
	\mybox{$\forall y \in F \quad \exists! x \in E \quad \big( y = f(x) \big)$}
\end{definition}

Autrement dit, tout élément de $F$ a un unique antécédent par $f$.

\myfigure{0.6}{
	\tikzinput{fig_ensembles13a}
	\qquad \qquad
	\tikzinput{fig_ensembles13b}
}
\begin{proposition}
	\label{prop:bij1}
	Soit $E, F$ des ensembles et $f : E \to F$ une application.
	\begin{enumerate}
		\item L'application $f$ est bijective si et seulement si il existe une application $g : F \to E$
		telle que $f \circ g = \id_F$ et $g \circ f = \id_E$.
		\item Si $f$ est bijective alors l'application $g$ est unique et elle aussi est bijective.
		L'application $g$ s'appelle la \defi{bijection réciproque}\index{bijection reciproque@bijection réciproque} de $f$ et est notée $f^{-1}$.
		De plus $\left( f^{-1} \right)^{-1} = f$.
	\end{enumerate}
\end{proposition}





\begin{proposition}
	\label{prop:bij2}
	Soient $f : E \to F$ et $g : F \to G$ des applications bijectives.
	L'application $g \circ f$ est bijective et sa bijection réciproque est
	\mybox{$(g\circ f)^{-1} = f^{-1} \circ g^{-1}$}
\end{proposition}



%-----------------------------------------
\subsection{Ensembles finis}


\begin{definition}
	Un ensemble $E$ est \defi{fini} s'il existe un entier $n\in \Nn$ et
	une bijection de $E$ vers $\{1,2,\ldots,n\}$.
	Cet entier $n$ est unique et s'appelle le \defi{cardinal}\index{cardinal} de $E$ (ou le \defi{nombre d'éléments})
	et est noté $\Card E$. Le cardinal de l'ensemble vide est $0$.
\end{definition}



Quelques propriétés :
\begin{itemize}
	\item Si $A$ est un ensemble fini et $B \subset A$ alors $B$ est aussi un ensemble fini et $\Card B \le \Card A$.
	\item Si $A, B$ sont des ensembles finis disjoints (c'est-à-dire $A \cap B = \varnothing$) alors
	$\Card (A \cup B) = \Card A + \Card B$.
	\item Si $A$ est un ensemble fini et $B \subset A$ alors $\Card (A \setminus B) = \Card A - \Card B$.
	En particulier si $B \subset A$ et  $\Card A = \Card B$ alors $A=B$.
	\item Enfin pour $A,B$ deux ensembles finis quelconques :
	\mybox{$\Card (A \cup B) = \Card A + \Card B - \Card (A\cap B)$}
\end{itemize}



%---------------------------------------------------------------
\textbf{Injection, surjection, bijection}

\begin{proposition}
	Soit $E,F$ deux ensembles finis et $f : E \to F$ une application.
	\begin{enumerate}
		\item \label{it:bij1} Si $f$ est injective alors $\Card E \le \Card F$.
		\item \label{it:bij2} Si $f$ est surjective alors $\Card E \ge \Card F$.
		\item \label{it:bij3} Si $f$ est bijective alors $\Card E = \Card F$.
	\end{enumerate}
\end{proposition}



\begin{proposition}
	Soit $E,F$ deux ensembles finis et $f : E \to F$ une application.
	Si
	$\Card E = \Card F$
	alors les assertions suivantes sont équivalentes :
	\begin{enumerate}
		\item[i.] $f$ est injective,
		\item[ii.] $f$ est surjective,
		\item[iii.] $f$ est bijective.
	\end{enumerate}
\end{proposition}


\begin{proposition}[Principe des tiroirs]
	Si l'on range dans $k$ tiroirs, $n > k$ paires de chaussettes
	alors il existe (au moins) un tiroir contenant (au moins) deux paires
	de chaussettes.
\end{proposition}



%---------------------------------------------------------------
\textbf{Nombres d'applications}

Soient $E,F$ des ensembles finis, non vides. On note $\Card E=n$ et $\Card F=p$.

\begin{proposition}
	Le nombre d'applications différentes de $E$ dans $F$ est :
	\myboxinline{$p^n$}
\end{proposition}

Autrement dit c'est \myboxinline{$(\Card F)^{\Card E}$}.


\begin{proposition}
	\label{prop:nbinj}
	Le nombre d'injections de $E$ dans $F$ est :
	$$p\times(p-1)\times\cdots\times(p-(n-1)).$$
\end{proposition}


Notation \defi{factorielle}\index{factorielle}\index{$n"!"$} : $n! = 1\times 2 \times 3 \times \cdots \times n$.
Avec $1!=1$ et par convention $0!=1$.

\begin{proposition}
	Le nombre de bijections d'un ensemble $E$ de cardinal $n$ dans lui-même est :
	\myboxinline{$n!$}
\end{proposition}








%---------------------------------------------------------------
\textbf{Coefficients du binôme de Newton}

Soit $E$ un ensemble fini de cardinal $n$.

\begin{proposition}
	Il y a $2^{\Card E}$ sous-ensembles de $E$ :
	\myboxinline{$\Card \mathcal{P}(E) = 2^n$}
\end{proposition}


\begin{definition}
	Le nombre de parties à $k$ éléments d'un ensemble à $n$ éléments est
	noté $\binom{n}{k}$ ou $C_n^k$.\index{$\binom{n}{k}$}\index{$C_n^k$}
\end{definition}


\begin{proposition}
	\sauteligne
	\begin{itemize}
		\item $\binom{n}{0}=1$, $\binom{n}{1}=n$, $\binom{n}{n}=1$. \\
		\item \myboxinline{$\binom{n}{n-k} = \binom{n}{k}$} \\
		\item \myboxinline{$\binom{n}{0}+\binom{n}{1}+\cdots+\binom{n}{k}+\cdots+\binom{n}{n} = 2^n$}
	\end{itemize}
\end{proposition}

\begin{proposition}\ \\[-2em]
	\label{prop:bin}
	\mybox{$\displaystyle \binom n k = \binom{n-1}{k} + \binom{n-1}{k-1} \qquad (0<k<n)$}
\end{proposition}


Le triangle de Pascal\index{triangle de Pascal} est un algorithme pour calculer ces coefficients $\binom{n}{k}$.
Chaque élément est obtenu en ajoutant les deux nombres qui lui sont
juste au-dessus et au-dessus à gauche.

$$\small\begin{array}{cccccc}
1  &&&&&\\
1 & 1 &&&&\\	
1 & 2 & 1 &&&\\
1 & 3 & 3 & 1 &&\\
1 & 4 & 6 & 4 & 1 &\\
1 & 5 & 10 & 10 & 5 & 1 \\
\end{array}$$

\begin{proposition}\
	\mybox{$\displaystyle \binom n k = \frac{n!}{k!(n-k)!}$}
\end{proposition}



%---------------------------------------------------------------
\textbf{Formule du binôme de Newton}
\index{formule!du binôme de Newton}

\begin{theoreme}
	Soient $a,b \in\Rr$ (ou $\Cc$) et $n$ un entier positif alors:
	\mybox{$\displaystyle (a+b)^n = \sum_{k=0}^n \binom{n}{k} \ a^{n-k} \cdot b^{k}$}
\end{theoreme}

Autrement dit :
{\small
	$$(a+b)^n = \binom{n}{0}\ a^n\cdot b^0 + \binom{n}{1}\ a^{n-1}\cdot b^{1}
	+ \cdots + \binom{n}{k} \ a^{n-k} \cdot b^{k}+\cdots + \binom{n}{n}\ a^0\cdot b^n$$
}




%-----------------------------------------
\subsection{Relation d'équivalence}


Une \defi{relation} sur un ensemble $E$, c'est la donnée pour tout couple
$(x,y)\in E \times E$ de \og Vrai \fg{} (s'ils sont en relation), ou de \og Faux \fg{} sinon.


\begin{definition}
	Soit $E$ un ensemble et $\mathcal{R}$ une relation, c'est une
	\defi{relation d'équivalence}\index{relation d equivalence@relation d'équivalence} si :
	\begin{itemize}
		\item $\forall x \in E$, $x \mathcal{R}x$, \quad  (\defi{réflexivité}\index{reflexivite@réflexivité})

		\item $\forall x,y \in E$, $x \mathcal{R}y \implies y\mathcal{R}x$, \quad  (\defi{symétrie}\index{symetrie@symétrie})

		\item $\forall x,y,z \in E$, $x \mathcal{R}y \text{ et }  y\mathcal{R}z \implies x\mathcal{R}z$, \quad
		(\defi{transitivité}\index{transitivite@transitivité})

	\end{itemize}
\end{definition}



\begin{definition}
	Soit $\mathcal{R}$ une relation d'équivalence sur un ensemble $E$.
	Soit $x\in E$, la \defi{classe d'équivalence}\index{classe d equivalence@classe d'équivalence} de $x$ est
	\mybox{$\text{cl}(x) = \big\{ y  \in E \mid y\mathcal{R}x \big\}$}
\end{definition}


$\text{cl}(x)$ est donc un sous-ensemble de $E$, on le note aussi $\overline{x}$.
Si $y \in \text{cl}(x)$, on dit que $y$ un \defi{représentant}\index{representant@représentant} de $\text{cl}(x)$.


\begin{proposition}
\sauteligne
	\begin{enumerate}
		\item $\text{cl}(x)=\text{cl}(y) \Longleftrightarrow x\mathcal{R}y$.
		\item Pour tout $x,y \in E$, $\text{cl}(x)=\text{cl}(y)$ ou $\text{cl}(x) \cap \text{cl}(y) = \varnothing$.
		\item Soit $C$ un ensemble de représentants de toutes les classes alors $\big\{ \text{cl}(x) \mid x \in C \big\}$
		constitue une partition de $E$.
	\end{enumerate}
\end{proposition}

Une \defi{partition}\index{partition} de $E$ est un ensemble $\{E_i\}$ de parties de $E$ tel que
$E = \bigcup_i E_i$ et $E_i\cap E_j = \varnothing$ (si $i\neq j$).
\myfigure{0.6}{
	\tikzinput{fig_ensembles21}
}



%---------------------------------------------------------------
\textbf{L'ensemble $\Zz/n\Zz$}

Soit $n\ge 2$ un entier fixé.
La relation suivante sur l'ensemble $E = \Zz$ est une relation d'équivalence:
\mybox{$a \equiv b \pmod n \quad \Longleftrightarrow \quad a-b \text{ est un multiple de } n$}


La classe d'équivalence de $a\in\Zz$ est notée $\overline a$ :
$$\overline a = \text{cl}(a) = \big\{ b \in \Zz \mid b \equiv a \pmod n \big\}.$$
Comme un tel $b$ s'écrit $b=a+kn$ pour un certain $k\in \Zz$ alors :
$$\overline a = a + n\Zz = \big\{ a+kn \mid k\in \Zz \big\}.$$

L'ensemble des classes d'équivalence est l'ensemble
\mybox{$\Zz/n\Zz = \big\{ \overline 0, \overline 1, \overline 2, \ldots, \overline{n-1} \big\}$}
qui contient exactement $n$ éléments.




\end{multicols}

\end{document}	



%%%%%%%%%%%%%%%%%%%%%%%%%%%%%%%%%%%%%%%%%%%%%%%%%%%%%%%%%%%%%%%%
\section{Applications}
