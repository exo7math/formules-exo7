\documentclass[10pt,class=article,crop=false]{standalone}
\usepackage{../exo7formules}


\begin{document}

\newcommand{\GL}{\mathcal{G}\!\ell}

%%%%%%%%%%%%%%%%%%%%%%%%%%%%%%%%%%%%%%%%%%
\section{Groupes}

\begin{multicols}{2}
	

%-----------------------------------------
\subsection{Définition}


Un \defi{groupe}\index{groupe} $(G,\star)$ est un ensemble $G$ auquel est associé une opération $\star$
(la \defi{loi de composition}\index{loi de composition}) vérifiant les quatre propriétés suivantes :
\begin{enumerate}
	\item \label{it:groupei} pour tout $x,y \in G$, \quad $x \star y \in G$
	\quad  ($\star$ est une \defi{loi de composition interne})
	
	\item \label{it:groupeii} pour tout $x,y,z \in G$, \quad $(x \star y) \star z = x \star (y \star z)$
	\quad (la loi est \defi{associative}\index{associativite@associativité})
	
	\item \label{it:groupeiii} il existe $e \in G$ tel que \quad $\forall x \in G$, $x \star e = x$ \  et \  $e \star x = x$
	\quad ($e$ est l'\defi{élément neutre}\index{element neutre@élément neutre})
	
	\item \label{it:groupeiv} pour tout $x \in G$ il existe $x' \in G$ tel que \quad  $x \star x' = x' \star x = e$
	\quad ($x'$ est l'\defi{inverse} de $x$ et est noté $x^{-1}$)
\end{enumerate}

Si de plus l'opération vérifie
$$ \text{pour tout } x,y \in G,\qquad x \star y = y \star x,$$
on dit que $G$ est un groupe \defi{commutatif}\index{commutatif} (ou \defi{abélien}).


\textbf{Exemples.}

\begin{itemize}
	\item $(\Rr^*,\times)$, $(\Qq^*,\times)$, $(\Cc^*, \times)$ sont des groupes commutatifs.
		
	\item $(\Zz,+)$, $(\Qq,+)$, $(\Rr,+)$, $(\Cc,+)$ sont des groupes commutatifs.
	
	\item L'ensemble des matrices $n\times n$ inversibles,
	muni de la multiplication des matrices $\times$, forme un groupe $(\GL_2,\times)$, il est non-commutatif car en général $M\times M' \neq M'\times M$.
	
	
	
\end{itemize}



\textbf{Puissance.} Soit un groupe $(G,\star)$ et $x \in G$.
\begin{itemize}
	\item $x^{n} = \underbrace{x\star x \star \cdots \star x}_{n \text{ fois}}$,
	\item $x^0 = e$,
	\item $x^{-n} = \underbrace{x^{-1}\star \cdots \star x^{-1}}_{n \text{ fois}}$.
\end{itemize}

Pour $x,y \in G$ et $m,n \in\Zz$ nous avons :
\begin{itemize}
	\item $x^m \star x^n = x^{m+n}$,
	\item $(x^m)^n = x^{mn}$,
	\item $(x \star y)^{-1} = y^{-1} \star x^{-1}$, \quad attention à l'ordre !
	\item \textbf{Si} $(G,\star)$ est \textbf{commutatif} alors $(x\star y)^n = x^n \star y^n$.
\end{itemize}




%-----------------------------------------
\subsection{Sous-groupes}

Soit $(G,\star)$ un groupe.	Une partie $H \subset G$ est un  \defi{sous-groupe}\index{sous-groupe} de $G$ si :
	\begin{enumerate}
		\item $e \in H$,
		\item pour tout $x, y \in H$, on a $x \star y \in H$,
		\item pour tout $x \in H$, on a $x^{-1} \in H$.
	\end{enumerate}

Notez qu'un sous-groupe $H$ est aussi un groupe $(H,\star)$.
La façon la plus rapide de montrer que $(H,\star)$ est un groupe est donc de montrer que c'est un sous-groupe d'un groupe $(G,\star)$.

Critère pratique pour prouver que $H$ est un sous-groupe de $G$ est:
	\begin{itemize}
		\item $H$ contient au moins un élément,
		\item et pour tout $x,y \in H$, $x \star y^{-1} \in H$.
	\end{itemize}

Exemples :
\begin{itemize}
	\item $(\Rr^*_+,\times)$ est un sous-groupe de $(\Rr^*,\times)$.
	\item $(\mathbb{U},\times)$ est un sous-groupe de $(\Cc^*,\times)$, où
	$\mathbb{U} = \{ z \in \Cc \mid |z|=1 \}$.
	\item $(\Zz,+)$ est un sous-groupe de $(\Rr,+)$.
	\item $\{e\}$ et $G$ sont les \defi{sous-groupes triviaux}\index{sous-groupe!trivial} du groupe $G$.
\end{itemize}

\begin{proposition}
Les sous-groupes de $(\Zz,+)$ sont les $n\Zz$, pour $n\in \Zz$.
\end{proposition}

L'ensemble $n\Zz$ désigne l'ensemble des multiples de $n$ : $n\Zz = \bigg\{ k\cdot n \mid  k \in \Zz \bigg\}$.


Soit $(G,\star)$ un groupe et $E \subset G$ un sous-ensemble de $G$.
Le \defi{sous-groupe engendré}\index{sous-groupe!engendré} par $E$ est le plus petit sous-groupe de
$G$ contenant $E$.

Exemple : dans $(\Zz,+)$ et $E = \{a,b\}$, le sous-groupe engendré est $H = n\Zz$ où $n=\pgcd(a,b)$.

%-----------------------------------------
\subsection{Morphismes de groupes}


Soient $(G,\star)$ et $(G',\diamond)$ deux groupes. Une application
$f : G \longrightarrow G'$ est un \defi{morphisme de groupes}\index{morphisme} si :
\mybox{$\text{pour tout } x,x' \in G  \qquad f(x \star x') = f(x) \diamond f(x')$}


Exemple : $\exp : (\Rr,+) \to (\Rr_+^*,\times)$, $\exp(x+x')= \exp(x) \times \exp(x')$.

Pour un morphisme
\begin{itemize}
	\item $f(e_G) = e_{G'}$,
	\item pour tout $x \in G$, $f(x^{-1}) = \big(f(x)\big)^{-1}$.
\end{itemize}


Un morphisme bijectif est un \defi{isomorphisme}\index{isomorphisme}. Deux groupes $G, G'$ sont \defi{isomorphes} s'il existe un morphisme bijectif $f : G \longrightarrow G'$.

Exemple : $\exp : (\Rr,+) \to (\Rr_+^*,\times)$ est un isomorphisme bijectif, sa bijection réciproque étant le morphisme : $\ln : (\Rr_+^*,\times) \to (\Rr,+)$ avec $\ln(x \times x') = \ln(x) + \ln(x')$.


\textbf{Noyau et image}

Soit $f: G \longrightarrow G'$ un morphisme de groupes.

\begin{itemize}
	\item Le \defi{noyau}\index{noyau} de $f$ est
	\mybox{$\Ker f = \big\{ x \in G \mid f(x) = e_{G'} \big\}$}
	
Le noyau est donc l'ensemble des éléments de $G$ qui s'envoient
par $f$ sur l'élément neutre de $G'$.

    \item L'\defi{image}\index{image} de $f$ est
	\mybox{$\Im f = \big\{ f(x)  \mid x \in G \big\}$}
	
	Ce sont les éléments de $G'$ qui ont (au moins) un antécédent par $f$.
\end{itemize}	
	
\begin{proposition}
	Soit $f: G \longrightarrow G'$ un morphisme de groupes.
	\begin{enumerate}
		\item $\Ker f$ est un sous-groupe de $G$.
		\item $\Im f$ est un sous-groupe de $G'$.
		\item \myboxinline{$f$ est injectif si et seulement si $\Ker f = \{ e_G \}$}
		\item $f$ est surjectif si et seulement si $\Im f = G'$.
	\end{enumerate}
\end{proposition}




%-----------------------------------------
\subsection{Le groupe $\Zz/n\Zz$}


Fixons $n \ge 1$. $\Zz/n\Zz$
est l'ensemble
$$\Zz/n\Zz =\left\{ \overline{0}, \overline{1}, \overline{2},\ldots, \overline{n-1} \right\}$$
où $\overline p$ désigne la classe d'équivalence de $p$ modulo $n$.

\mybox{$\overline p = \overline q \Longleftrightarrow p \equiv q \pmod n$}
ou encore
$\overline p = \overline q \Longleftrightarrow \exists k \in \Zz \quad p = q + kn$.


L'\defi{addition} sur $\Zz/n\Zz$ est définie par : $\overline p + \overline q = \overline{p+q}$.



\begin{proposition}
$(\Zz/n\Zz,+)$ est un groupe commutatif de cardinal $n$
\end{proposition}

L'élément neutre est $\overline{0}$.
L'opposé de $\overline k$ est $-\overline{k}=\overline{-k}=\overline{n-k}$.


\textbf{Groupes cycliques de cardinal fini}


Un groupe $(G,\star)$ est un groupe \defi{cyclique}\index{groupe!cyclique} s'il existe un élément $a \in G$ tel que :
$$\text{pour tout } x \in G, \text{ il existe } k \in \Zz \text{ tel que } x = a^k$$


Autrement dit le groupe $G$ est engendré par un seul élément $a$.

Le groupe $(\Zz/n\Zz,+)$ est un groupe cyclique. En effet il est engendré par $a=\overline 1$,
car tout élément $\overline k$ s'écrit $\overline k =
\underbrace{\overline 1 + \overline 1 + \cdots \overline 1}_{k \text{ fois}} = k\cdot \overline 1$.

\begin{theoreme}
	\label{prop:cyclique}
	Si $(G,\star)$ un groupe cyclique de cardinal $n$, alors
	$(G,\star)$ est isomorphe à $(\Zz/n\Zz,+)$.
\end{theoreme}



%-----------------------------------------
\subsection{Le groupe des permutations}

Fixons un entier $n\ge 2$.
\begin{proposition}
	L'ensemble des bijections de $\{1,2,\ldots,n\}$ dans lui-même, muni de la composition
	des fonctions est un groupe, noté $(\mathcal{S}_n,\circ)$.
	Le cardinal de $\mathcal{S}_n$ est $n!$~.
\end{proposition}

Une bijection de $\{1,2,\ldots,n\}$ (dans lui-même) s'appelle une \defi{permutation}\index{permutation}.
Le groupe $(\mathcal{S}_n,\circ)$ s'appelle le \defi{groupe des permutations}
(ou le \defi{groupe symétrique}).

L'élément neutre du groupe est l'identité $\id$, le produit est ici la composition et l'inverse correspond à la bijection réciproque.



\end{multicols}

\end{document}	