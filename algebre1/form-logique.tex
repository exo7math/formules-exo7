\documentclass[10pt,class=article,crop=false]{standalone}
\usepackage{../exo7formules}


\begin{document}

\newcommand{\assertion}[1]{{\og\emph{#1}\fg}} % pour chapitre logique

%%%%%%%%%%%%%%%%%%%%%%%%%%%%%%%%%%%%%%%%%%
\section{Logique et raisonnements}

\begin{multicols}{2}
	

%-----------------------------------------
\subsection{Logique}




Une \defi{assertion}\index{assertion} est une phrase soit vraie, soit fausse, pas les deux en même temps.

%--------------
\textbf{Et logique.}
L'assertion \assertion{$P$ \defi{et} $Q$}\index{et logique@\og et \fg{} logique} est vraie si $P$ est vraie et $Q$ est vraie. L'assertion \assertion{$P$ et $Q$} est fausse sinon.

Sa \defi{table de vérité}\index{table de verite@table de vérité} :
\begin{figure}[H]
	\centering
	\begin{tabular}{c|c|c}
		\textcolor{olive}{$P$} $\backslash$ \textcolor{blue}{$Q$} & \textcolor{blue}{V} & \textcolor{blue}{F} \\ \hline
		\textcolor{olive}{V} & \textcolor{red}{V} & \textcolor{red}{F} \\ \hline
		\textcolor{olive}{F} & \textcolor{red}{F} & \textcolor{red}{F} \\
	\end{tabular}

	{Table de vérité de \assertion{$P$ et $Q$}}
\end{figure}

%--------------
\textbf{Ou logique.}
L'assertion \assertion{$P$ \defi{ou} $Q$}\index{ou logique@\og ou \fg{} logique} est vraie si l'une (au moins) des deux assertions $P$ ou $Q$ est vraie.
L'assertion \assertion{$P$ ou $Q$} est fausse si les deux assertions $P$ et $Q$ sont fausses.

\begin{figure}[H]
	\centering
	\begin{tabular}{c|c|c}
		\textcolor{olive}{$P$} $\backslash$ \textcolor{blue}{$Q$}  & \textcolor{blue}{V} & \textcolor{blue}{F} \\ \hline
		\textcolor{olive}{V} & \textcolor{red}{V} & \textcolor{red}{V} \\ \hline
		\textcolor{olive}{F} & \textcolor{red}{V} & \textcolor{red}{F} \\
	\end{tabular}
	
	{Table de vérité de \assertion{$P$ ou $Q$}}
\end{figure}


L'assertion \assertion{\defi{non} $P$}\index{négation} est vraie si $P$ est fausse, et fausse si $P$ est vraie.

\begin{figure}[H]
	\centering
	\begin{tabular}{c|c|c}
		\textcolor{blue}{$P$}  &  \textcolor{blue}{V} &  \textcolor{blue}{F} \\ \hline
		\textcolor{red}{non $P$}    & \textcolor{red}{F} & \textcolor{red}{V} \\
	\end{tabular}
	
	{Table de vérité de \assertion{non $P$}}
\end{figure}


%--------------
\textbf{L'implication $\implies$}


La définition mathématique est la suivante :
\mybox{
	L'assertion \assertion{(non $P$) ou $Q$} est notée \assertion{$P \implies Q$}\index{implication}\index{$\implies$}.
}

\begin{figure}[H]
	\centering
	\begin{tabular}{c|c|c}
		\textcolor{olive}{$P$} $\backslash$ \textcolor{blue}{$Q$}  & \textcolor{blue}{V} & \textcolor{blue}{F} \\ \hline
		\textcolor{olive}{V} & \textcolor{red}{V} & \textcolor{red}{F} \\ \hline
		\textcolor{olive}{F} & \textcolor{red}{V} & \textcolor{red}{V} \\
	\end{tabular}
	
	{Table de vérité de \assertion{$P \implies Q$}}
\end{figure}


%--------------
\textbf{L'équivalence $\iff$}


L'\defi{équivalence} est définie par :
\mybox{
	\assertion{$P \iff Q$}\index{equivalence@équivalence}\index{$\iff$}  est l'assertion \assertion{($P \implies Q$) \  et \  ($Q \implies P$)}.
}

\begin{figure}[H]
	\centering
	\begin{tabular}{c|c|c}
		\textcolor{olive}{$P$} $\backslash$ \textcolor{blue}{$Q$}  & \textcolor{blue}{V} & \textcolor{blue}{F} \\ \hline
		\textcolor{olive}{V} & \textcolor{red}{V} & \textcolor{red}{F} \\ \hline
		\textcolor{olive}{F} & \textcolor{red}{F} & \textcolor{red}{V} \\
	\end{tabular}
	
	{Table de vérité de \assertion{$P \iff Q$}}
\end{figure}


\begin{proposition}
	\label{prop:log}
	Soient $P, Q, R$ trois assertions.
	Nous avons les équivalences (vraies) suivantes :
	\begin{enumerate}
		\item \emph{$P \iff \text{ non}(\text{non}(P))$}
		\item \emph{$(P \text{ et } Q) \iff (Q \text{ et } P)$}
		\item \emph{$(P \text{ ou } Q) \iff (Q \text{ ou } P)$}
		\item \emph{$\text{non}(P \text{ et } Q)  \iff  (\text{non } P)  \text{ ou } (\text{non }Q)$}
		\item \emph{$\text{non}(P \text{ ou } Q)  \iff  (\text{non } P)  \text{ et } (\text{non }Q)$}
		\item \emph{$\big(P \text{ et } (Q \text{ ou } R)  \big)   \iff
			(P \text{ et } Q) \text{ ou } (P \text{ et }  R)$}
		\item \emph{$\big(P \text{ ou } (Q \text{ et } R)  \big)   \iff
			(P \text{ ou } Q) \text{ et } (P \text{ ou }  R)$}
		\item  \assertion{$P \implies Q$}  $\iff$ \assertion{$\text{non}(Q) \implies \text{non}(P)$}
	\end{enumerate}
\end{proposition}


%---------------------------------------------------------------
%\subsection{Quantificateurs}
\index{quantificateur}

%--------------
\textbf{Le quantificateur $\forall$ : \assertion{pour tout}}
\index{$\forall$}

L'assertion
$$\forall x \in E \quad P(x)$$
est une assertion vraie lorsque les assertions $P(x)$ sont vraies pour tous les éléments $x$
de l'ensemble $E$.


%--------------
\textbf{Le quantificateur $\exists$ : \assertion{il existe}}
\index{$\exists$}

L'assertion
$$\exists x \in E \quad P(x)$$
est une assertion vraie lorsque l'on peut trouver au moins un $x$ de $E$ pour lequel $P(x)$ est vraie.


%--------------
\textbf{La négation des quantificateurs}

\mybox{
	La négation de \assertion{$\forall x \in E \quad P(x)$} \ \ est \ \
	\assertion{$\exists x \in E \quad \text{non } P(x)$} .
}



\mybox{
	La négation de \assertion{$\exists x \in E \quad P(x)$} \ \ est \ \ \assertion{$\forall x \in E \quad \text{non } P(x)$}.
}



L'ordre des quantificateurs est très important.



%-----------------------------------------
\subsection{Raisonnements}


%---------------------------------------------------------------
\textbf{Raisonnement direct}

On veut montrer que l'assertion \assertion{$P \implies Q$} est vraie.
On suppose que $P$ est vraie et on montre qu'alors $Q$ est vraie.
C'est la méthode à laquelle vous êtes le plus habitué.

\smallskip

%---------------------------------------------------------------
\textbf{Cas par cas}

Si l'on souhaite vérifier une assertion $P(x)$ pour tous les $x$ dans un ensemble $E$, on
montre l'assertion pour les $x$ dans une partie $A$ de $E$, puis pour les $x$
n'appartenant pas à $A$. C'est la méthode de \defi{disjonction}\index{disjonction} ou du \defi{cas par cas}.

\smallskip

%---------------------------------------------------------------
\textbf{Contraposée}

Le raisonnement par \defi{contraposition}\index{contraposition} est basé sur l'équivalence suivante :
\mybox{
	L'assertion \assertion{$P \implies Q$}  \  est équivalente à \  \assertion{$\text{non}(Q) \implies \text{non}(P)$}.
}

Donc si l'on souhaite montrer l'assertion \assertion{$P \implies Q$}, on montre en fait
que si $\text{non}(Q)$ est vraie alors $\text{non}(P)$ est vraie.



\smallskip

%---------------------------------------------------------------
\textbf{Absurde}

Le \defi{raisonnement par l'absurde}\index{absurde} pour montrer \assertion{$P \implies Q$}  repose sur le principe suivant :
on suppose à la fois que $P$ est vraie et que $Q$ est fausse et on cherche une contradiction.
Ainsi si $P$ est vraie alors $Q$ doit être vraie et donc \assertion{$P \implies Q$} est vraie.


\smallskip

%---------------------------------------------------------------
\textbf{Contre-exemple}

Si l'on veut montrer qu'une assertion du type \assertion{$\forall x \in E \quad P(x)$} est fausse alors il suffit de trouver $x \in E$ tel que $P(x)$
soit fausse. 
Trouver un tel $x$ c'est trouver un \defi{contre-exemple}.


\smallskip

%---------------------------------------------------------------
\textbf{Récurrence}

Le \defi{principe de récurrence}\index{recurrence@récurrence} permet de montrer qu'une assertion $P(n)$, dépendant de $n$,
est vraie pour tout $n \in \Nn$.
La démonstration se déroule en trois étapes :
\begin{itemize}
	\item \evidence{initialisation} : on prouve $P(0)$.
	\item \evidence{hérédité}\index{heredite@hérédité} : qui commence par \og{}Je fixe $n\ge 0$ et je suppose que l'assertion $P(n)$ vraie. je vais montrer que 
l'assertion  $P(n+1)$ (au rang suivant) est vraie\ldots\fg{}
   \item \evidence{conclusion} : par le principe de récurrence $P(n)$ est vraie pour tout $n\in\Nn$.
\end{itemize}


\end{multicols}

\end{document}	


