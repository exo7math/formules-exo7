\documentclass[10pt,class=article,crop=false]{standalone}
\usepackage{../exo7formules}


\begin{document}

%%%%%%%%%%%%%%%%%%%%%%%%%%%%%%%%%%%%%%%%%%
\section{Polynômes}

\begin{multicols}{2}
	

%-----------------------------------------
\subsection{Définitions}

Un \defi{polynôme}\index{polynome@polynôme} à coefficients dans $\Kk$ ($\Qq$, $\Rr$ ou $\Cc$)
est une expression :
$$P(X) = a_n X^n + a_{n-1} X^{n-1} + \cdots + a_2 X^2 + a_1 X + a_0$$

\begin{itemize}
	\item Les $a_i \in \Kk$ sont appelés les \defi{coefficients} du polynôme.
	
	\item Si $a_n \neq 0$, $n \in \Nn$ est le \defi{degré}\index{degre@degré} de $P$, noté 
	$\deg P$. (Convention : le degré du polynôme nul est $-\infty$.)

	\item $\Kk[X]$ désigne l'ensemble des polynômes.
	
    \item $\Kk_n[X]$ est l'ensemble des polynômes de degré $\le n$.
    
    \item Deux polynômes sont \defi{égaux} si et seulement si ils ont les mêmes coefficients.
    
\end{itemize}


\textbf{Multiplication.}
Soient  $P=a_nX^n+a_{n-1}X^{n-1}+\cdots + a_1X+a_0$ et
$Q=b_mX^m+b_{m-1}X^{m-1}+\cdots+b_1X+b_0$.
$P \times Q$ est un polynôme de degré $n+m$ avec :
$$c_k=\sum_{i+j=k}a_ib_j  \text{ pour }k\in \{0,\ldots, r\}.$$


\mybox{$\deg(P\times Q)=\deg P + \deg Q
	\qquad
\deg(P+Q) \le \max(\deg P, \deg Q)$}

\begin{itemize}
	\item Les polynômes comportant un seul terme non nul (du type $a_kX^k$) sont
	appelés \defi{monômes}\index{monome@monôme}.
	
	\item Soit $P=a_nX^n+a_{n-1}X^{n-1}+\cdots + a_1X+a_0,$ un polynôme avec
	$a_n\neq0$. On appelle \defi{terme dominant} le monôme $a_nX^n$. Le coefficient
	$a_n$ est appelé le \defi{coefficient dominant} de $P$.
	
	\item Si le coefficient dominant est $1$, on dit que $P$ est un \defi{polynôme
		unitaire}.
\end{itemize}


%-----------------------------------------
\subsection{Arithmétique des polynômes}

% \textbf{Division euclidienne}


Soient $A,B \in\Kk[X]$, on dit que $B$ \defi{divise}\index{divisibilite@divisibilité} $A$ s'il existe  $Q\in\Kk[X]$ tel que $A=BQ$.
On note alors $B|A$.
On dit aussi que $A$ est multiple de $B$ ou que $A$ est divisible par $B$.

\begin{theoreme}[Division euclidienne des polynômes]
	Soient $A,B \in\Kk[X]$, avec $B \neq 0$, alors il existe un
	unique polynôme $Q$ et il existe un unique polynôme $R$ tels que :
	\mybox{$A=BQ+R \quad \text{ et } \quad \deg R < \deg B$.}
\end{theoreme}

$Q$ est appelé le \defi{quotient}\index{quotient} et $R$ le \defi{reste}\index{reste} et cette écriture
est la \defi{division euclidienne}\index{division euclidienne} de $A$ par $B$.

Notez que la condition $\deg R < \deg B$ signifie $R=0$
ou bien $0 \le \deg R < \deg B$.

Enfin $R=0$ si et seulement si $B|A$.


\begin{exemple}
	On pose une division de polynômes comme une division euclidienne de deux entiers.
	Par exemple si $A=2X^4-X^3-2X^2+3X-1$ et $B=X^2-X+1$.
	Alors on trouve $Q= 2X^2+X-3$ et $R=-X+2$.
	
	\myfigure{0.8}{
		\tikzinput{fig_polynome01}
	}
	
\end{exemple}

Le \defi{pgcd}\index{pgcd} (plus grand commun diviseur) de $A$ et $B$
est l'unique polynôme unitaire de plus grand degré qui
divise à la fois $A$ et $B$.





\textbf{Algorithme d'Euclide.}
\index{algorithme d'Euclide}
Soient $A$ et $B$ des polynômes, $B \neq 0$.
Si $A=BQ+R$ alors $\pgcd(A,B) = \pgcd(B,R)$.
On calcule des divisions euclidiennes successives,

$$\begin{array}{ll}
	A = B Q_1+R_1 \quad \quad & \deg R_1 < \deg B \\
	B = R_1 Q_2 + R_2  & \deg R_2 < \deg R_1 \\
	R_1=R_2Q_3+R_3 & \deg R_3 < \deg R_2 \\
	\cdots & \\
\end{array}$$

Le degré du reste diminue à chaque division.
Le pgcd est le dernier reste non nul $R_k$ (rendu unitaire).


$A$ et $ B$ sont \defi{premiers entre eux}\index{polynome@polynôme!premiers entre eux} si $\pgcd(A,B)=1$.
Pour $A,B$ quelconques on peut se ramener à des polynômes premiers entre eux :
si  $\pgcd(A,B)=D$ alors $A$ et $B$ s'écrivent : $A=DA'$, $B=DB'$ avec $\pgcd(A',B')=1$.

\begin{theoreme}[de Bézout]
	\index{theoreme@théorème!de Bézout}
	\label{thm_Bezout}
	Soient $A, B\in \Kk[X]$ des polynômes avec $A\neq 0$ ou $B\neq 0$.
	On note $D=\pgcd(A,B)$.
	Il existe deux polynômes $U, V\in \Kk[X]$ tels que $AU+BV=D$.
\end{theoreme}

\begin{corollaire}
	Soient $A$ et $B$ deux polynômes. $A$ et $B$ sont premiers entre eux si
	et seulement s'il existe deux polynômes $U$ et $V$ tels que $AU+BV=1$.
\end{corollaire}

\begin{corollaire}
	Soient $A, B, C\in \Kk[X]$ avec $A\neq 0$ ou $B\neq 0$.
	Si $C|A$ et $C|B$ alors $C|\pgcd(A,B)$.
\end{corollaire}

\begin{corollaire}[Lemme de Gauss]
	\index{lemme!de Gauss}
	Soient $A, B, C\in \Kk[X]$.
	Si $A|BC$ et $\pgcd(A,B)=1$ alors $A|C$.
\end{corollaire}


%\textbf{ppcm}

%\begin{proposition}
%	Soient $A, B\in \Kk[X]$ des polynômes non nuls, alors il existe un unique
%	polynôme unitaire $M$ de plus petit degré tel que $A|M$ et $B|M$.
%\end{proposition}

%Le \defi{ppcm}\index{ppcm} (plus petit commun multiple) de $A$ et $B$ est l'unique
%polynôme unitaire $M$ de plus petit degré tel que $A|M$ et $B|M$.

%\begin{proposition}
%	Soient $A, B\in\Kk[X]$ des polynômes non nuls et
%	$M=\ppcm(A,B)$. Si $C\in\Kk[X]$ est un polynôme tel que $A|C$ et $B|C$, alors $M|C$.
%\end{proposition}

%-----------------------------------------
\subsection{Racine d'un polynôme, factorisation}

$\alpha\in\Kk$ est une \defi{racine}\index{racine} (ou un \defi{zéro}) de $P\in\Kk[X]$ si $P(\alpha)=0$.

\mybox{$P(\alpha)=0 \quad \iff \quad X-\alpha \text{ divise } P$}

$\alpha$ est une \defi{racine de multiplicité $k$} de $P$ est équivalent à l'une des propriétés suivantes :
\begin{itemize}
	\item[(i)]  $P(\alpha)= P'(\alpha)=\cdots=P^{(k-1)}(\alpha)=0$ et $P^{(k)}(\alpha) \neq 0$.
	
	\item[(ii)] Il existe  $Q \in\Kk[X]$ tel que $P=(X-\alpha)^kQ,$ avec $Q(\alpha) \neq 0$.
	
	\item[(iii)]$(X-\alpha )^k$ divise $P$ alors que $(X- \alpha )^{k+1}$ ne divise pas $P$.
\end{itemize}


\begin{theoreme}[Théorème de d'Alembert-Gauss]
	Tout polynôme à coefficients complexes de degré $n \ge 1$
	a au moins une racine dans $\Cc$.
	Il admet exactement $n$ racines si on compte chaque racine
	avec multiplicité.
\end{theoreme}


\begin{exemple}
	Soit $P(X)=aX^2+bX+c$ de degré $2$ à coefficients $a,b,c$ réels.
	\begin{itemize}
		\item Si $\Delta = b^2-4ac > 0$, $P$ a $2$ racines réelles distinctes $\frac{-b\pm\sqrt{\Delta}}{2a}$.
		\item Si $\Delta < 0$, $P$ a $2$ racines complexes conjuguées $\frac{-b \pm \ii\sqrt{|\Delta|}}{2a}$.
		\item Si $\Delta = 0$ $P$ a une racine réelle double $\frac{-b}{2a}$.
	\end{itemize}
\end{exemple}


\textbf{Polynômes irréductibles}
\begin{itemize}
	\item Un polynôme \defi{irréductible} $P$ est donc un polynôme non constant
	dont les seuls diviseurs de $P$ sont les constantes ou $P$ lui-même
	(à une constante multiplicative près).
	Cela correspond à la notion de nombre premier pour l'arithmétique de $\Zz$.
	
	\item Dans le cas contraire, $P$ est \defi{réductible}\index{reductibilite@réductibilité} :
	il existe  $A, B \in \Kk[X]$ tels que $P=AB$, avec
	$\deg A \ge 1$ et  $\deg B \ge 1$.
\end{itemize}



%\begin{proposition}[Lemme d'Euclide]
%	\index{lemme!d'Euclide}
%	Soit $P\in\Kk[X]$ un polynôme irréductible et soient $A, B\in\Kk[X]$.
%	Si $P|AB$ alors $P|A$ ou $P|B$.
%\end{proposition}


\begin{theoreme}[Factorisation sur $\Cc$ et $\Rr$]
\sauteligne	
\begin{enumerate}
	\item Tout polynôme unitaire s'écrit de manière unique comme un produit de polynômes irréductibles unitaires.
	
    \item Les polynômes irréductibles de $\Cc[X]$ sont les polynômes de degré $1$.
	
	Tout $P\in\Cc[X]$ se factorise en produit de polynômes de degré $1$ dans $\Cc[X]$.

	\item Les polynômes irréductibles de $\Rr[X]$
	sont les polynômes de degré $1$ et ceux de degré $2$ ayant
	un discriminant $\Delta<0$.
	
	Tout $P\in\Cc[X]$ se factorise en produit de polynômes irréductibles de degré $1$ ou $2$ dans $\Cc[X]$.	
\end{enumerate}
\end{theoreme}



\begin{exemple}
Soit $P(X)=X^4+1$.
\begin{itemize}
	\item Sur $\Cc$. D'abord $P(X)=(X^2+\ii)(X^2-\ii)$.
	Les racines de $P$ sont les racines carrées complexes de $\ii$ et $-\ii$.
	Ainsi $P$ se factorise dans $\Cc[X]$ :
	$$P(X)=\big(X-\tfrac{\sqrt2}{2}(1+\ii)\big)\big(X+\tfrac{\sqrt2}{2}(1+\ii)\big)\big(X-\tfrac{\sqrt2}{2}(1-\ii)\big)
	\big(X+\tfrac{\sqrt2}{2}(1-\ii)\big)$$
	
	\item Sur $\Rr$ on regroupe les facteurs ayant des racines conjuguées :
	$$P(X) = \big[X^2+\sqrt2X+1\big]\big[X^2-\sqrt2X+1\big]$$ 
\end{itemize}
\end{exemple}

%-----------------------------------------
\subsection{Fractions rationnelles}


Une \defi{fraction rationnelle}\index{fraction rationnelle} à coefficients dans $\Kk$ est une expression de la forme
$F=\frac{P}{Q}$ où $P,Q \in \Kk[X]$ sont deux polynômes et $Q \neq 0$.



%Toute fraction rationnelle est somme de fractions rationnelles
%élémentaires : les \og éléments simples \fg. Mais les éléments simples sont différents sur $\Cc$ ou sur $\Rr$.


\begin{theoreme}[Décomposition en éléments simples sur $\Cc$]
	Soit $P/Q$ une fraction rationnelle avec $P,Q \in \Cc[X]$, $\pgcd(P,Q)=1$ et
	$Q=(X-\alpha_1)^{k_1}\cdots(X-\alpha_r)^{k_r}$.
	Alors il existe une et une seule écriture :
	$$\begin{array}{rl}
		\displaystyle \frac{P}{Q} \  =  \ \  E(X)  & + \
		\displaystyle  \frac{a_{1,1}}{(X-\alpha_1)^{k_1}}+\frac{a_{1,2}}{(X-\alpha_1)^{k_1-1}}+\cdots
		+\ \frac{a_{1,k_1}}{(X-\alpha_1)} \\[4mm]
		& \displaystyle+ \frac{a_{2,1}}{(X-\alpha_2)^{k_2}}+\cdots
		+\ \frac{a_{2,k_2}}{(X-\alpha_2)}  + \ \cdots
	\end{array}$$
\end{theoreme}

Le polynôme $E$ s'appelle la \defi{partie polynomiale}\index{partie polynomiale} 
(ou \defi{partie entière}).
Les termes $\frac{a}{(X-\alpha)^i}$ sont les \defi{éléments simples}\index{element simple@élément simple} sur $\Cc$.



\begin{theoreme}[Décomposition en éléments simples sur $\Rr$]
	Soit $P/Q$ une fraction rationnelle avec $P,Q \in \Rr[X]$, $\pgcd(P,Q)=1$.
	Alors $P/Q$ s'écrit de manière unique comme somme :
	\begin{itemize}
		\item d'une partie polynomiale $E(X)$,
		\item d'éléments simples du type $\frac{a}{(X-\alpha)^i}$,
		\item d'éléments simples du type $\frac{aX+b}{(X^2+\alpha X + \beta)^i}$,
	\end{itemize}
	où les $X-\alpha$ et $X^2+\alpha X + \beta$ sont les facteurs irréductibles de $Q(X)$
	et les exposants $i$ sont inférieurs ou égaux à la puissance correspondante dans cette factorisation.
\end{theoreme}




\end{multicols}

\end{document}	