\documentclass[10pt,class=article,crop=false]{standalone}
\usepackage{../exo7formules}


\begin{document}

%%%%%%%%%%%%%%%%%%%%%%%%%%%%%%%%%%%%%%%%%%
\section{Systèmes linéaires}

\begin{multicols}{2}
	

%-----------------------------------------
\subsection{Introduction aux systèmes d'équations linéaires}


%---------------------------------------------------------------
\textbf{Deux droites dans le plan}

Calculer l'intersection de deux droites $D_1$ et $D_2$ du plan équivaut à résoudre le système :
\begin{equation}
	\left\{\begin{array}{rcl}
		a x + b y & = & e\\
		c x + d y & = & f
	\end{array}\right.
	\tag{$S$}
	\label{eq:syslin1}
\end{equation}


Trois cas :
\begin{enumerate}
	\item Les droites $D_1$ et $D_2$ se coupent en un seul point :	
	le système (\ref{eq:syslin1}) a une seule solution.
	
	
	\item Les droites $D_1$ et $D_2$ sont parallèles : le système
	(\ref{eq:syslin1}) n'a pas de solution.
	
	
	
	\item Les droites $D_1$ et $D_2$ sont confondues : le système  (\ref{eq:syslin1}) a une infinité de solutions.
\end{enumerate}


\myfigure{0.4}{
	\tikzinput{fig_syslin01}
	\ \ 
	\tikzinput{fig_syslin02}
	\ \ 
	\tikzinput{fig_syslin03}
}




%---------------------------------------------------------------
\textbf{Résolution par la méthode de Cramer}

\index{regle@règle!de Cramer}

On note $\left| \begin{smallmatrix} a & b \\ c & d \end{smallmatrix}\right|=ad-bc$
le \defi{déterminant}\index{determinant@déterminant}. 
Si $ad-bc\neq 0$, on trouve une unique solution à (\ref{eq:syslin1}) dont les coordonnées $(x,y)$ sont :
$$x = \frac{\begin{vmatrix} e & b \\ f & d \end{vmatrix}}{\begin{vmatrix} a & b \\ c & d \end{vmatrix}} \qquad
y = \frac{\begin{vmatrix} a & e \\ c & f \end{vmatrix}}{\begin{vmatrix} a & b \\ c & d \end{vmatrix}}$$





%---------------------------------------------------------------
\textbf{Résolution par inversion de matrice}

%Pour ceux qui connaissent les matrices,
En termes matriciels, le système linéaire(\ref{eq:syslin1})
est équivalent à
$$AX = Y \quad \text{ où } \quad A = \begin{pmatrix} a & b \\ c & d \end{pmatrix},
\quad  X = \begin{pmatrix} x \\ y \end{pmatrix}, \quad Y = \begin{pmatrix} e \\ f \end{pmatrix}.$$

Si le déterminant de la matrice $A$ est non nul
alors la matrice $A$ est inversible et
$$A^{-1} = \frac{1}{ad-bc} \begin{pmatrix} d & -b \\ -c & a \end{pmatrix}$$
et l'unique solution $X=\left( \begin{smallmatrix} x \\ y \end{smallmatrix}\right)$
du système est donnée par
$$X = A^{-1} Y.$$


%-----------------------------------------
\subsection{Théorie des systèmes linéaires}



	On appelle \defi{équation linéaire}\index{equation lineaire@équation linéaire} dans les variables (ou \defi{inconnues})
	$x_1,\ldots,x_p$ toute relation de la forme
	\begin{equation}
		a_1 x_1 + \cdots + a_p x_p = b,
	\end{equation}
	où $a_1, \ldots, a_p $ et $b$ sont des nombres réels donnés.



Forme générale d'un \defi{système de $n$ équations linéaires à $p$ inconnues} :
$$\left\{
\begin{array}{cccccccc}
	a_{11}x_1 & +a_{12}x_2  &+a_{13}x_3&+&\cdots&+a_{1p}x_p&=&b_1   \\
	a_{21}x_1 & +a_{22}x_2  &+a_{23}x_3&+&\cdots&+a_{2p}x_p&=&b_2   \\
	\vdots &  \vdots  & \vdots&&& \vdots&=& \vdots   \\
	a_{i1}x_1 & +a_{i2}x_2  &+a_{i3}x_3&+&\cdots&+a_{ip}x_p&=&b_i  \\
	\vdots &  \vdots  & \vdots&&& \vdots&=& \vdots   \\
	a_{n1}x_1 & +a_{n2}x_2  &+a_{n3}x_3&+&\cdots&+a_{np}x_p&=&b_n   \\
\end{array}
\right.
$$

\begin{itemize}
	\item Les nombres $a_{ij}$, $i=1,\ldots, n$, $j=1,\ldots, p$, sont les \defi{coefficients} du système.
Ce sont des données. Les nombres $b_{i}$, $i=1,\ldots, n$, constituent le
\defi{second membre}\index{second membre} du système et sont également des données.


	\item 	Une \defi{solution} du système linéaire est une liste de
	$p$ nombres réels $(s_1,s_2,\ldots,s_p)$ (un $p$-uplet) tels que si
	l'on substitue $s_1$ pour $x_1$, $s_2$ pour $x_2$, etc., dans le système
	linéaire, on obtient une égalité. L'\,\defi{ensemble des solutions du système}
	est l'ensemble de tous ces $p$-uplets.


	\item 	\label{systemes equivalents}
	On dit que deux systèmes linéaires sont \defi{équivalents}\index{systeme@système!équivalent}
	s'ils ont le même ensemble de solutions.
\end{itemize}



\begin{theoreme}
	Un système d'équations linéaires n'a soit aucune solution,
	soit une seule solution, soit une infinité de solutions.
\end{theoreme}

\begin{itemize}
	\item Un système linéaire qui n'a aucune solution est dit \defi{incompatible}\index{systeme@système!incompatible}.


	\item Un \defi{système homogène}\index{systeme@système!homogène}
 c'est lorsque le second membre est nul : $b_1=b_2=\cdots=b_n=0$.
De tels systèmes sont toujours compatibles car ils admettent toujours
la \emph{solution triviale} $s_1=s_2=\cdots=s_p=0$. 
\end{itemize}

%-----------------------------------------
\subsection{Résolution par la méthode du pivot de Gauss}


	Un système est \defi{échelonné}\index{systeme@système!échelonné} si :
	\begin{itemize}
		\item le nombre de coefficients nuls commençant une ligne croît strictement ligne après ligne.
	\end{itemize}
	
	Il est \defi{échelonné réduit}\index{systeme@système!réduit} si en plus :
	\begin{itemize}
		\setcounter{enumi}{1}
		\item le premier coefficient non nul d'une ligne vaut $1$ ;
		
		\item et c'est le seul élément non nul de sa colonne.
	\end{itemize}




%---------------------------------------------------------------
\textbf{Opérations sur les équations d'un système}

Les trois opérations élémentaires sur les équations sont :
\begin{enumerate}
	\item $L_i \leftarrow \lambda L_i$ avec $\lambda \neq 0$ :
	on peut multiplier une équation par un réel non nul.
	
	\item $L_i \leftarrow L_i+\lambda L_j$ avec $\lambda \in \Rr$ (et $j\neq i$) :
	on peut ajouter à l'équation $L_i$ un multiple d'une autre équation $L_j$.
	
	\item $L_i \leftrightarrow L_j$ : on peut échanger deux équations.
\end{enumerate}

Ces trois opérations élémentaires ne changent pas les solutions d'un système linéaire ;
autrement dit ces opérations transforment un système linéaire en un système linéaire équivalent.


%---------------------------------------------------------------
\textbf{Méthode du pivot de Gauss}

La méthode du pivot de Gauss permet de trouver les solutions de n'importe
quel système linéaire par des opérations élémentaires sur les lignes. Les étapes sont :
\begin{enumerate}
	\item Passage à une forme échelonnée.
	\item Passage à une forme réduite.
    \item Solutions.
\end{enumerate}



\begin{theoreme}
	Tout système homogène d'équations linéaires dont le nombre d'inconnues est
	strictement plus grand que le nombre d'équations a une infinité de solutions.
\end{theoreme}



\end{multicols}

\end{document}	

