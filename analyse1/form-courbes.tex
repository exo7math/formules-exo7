\documentclass[10pt,class=article,crop=false]{standalone}
\usepackage{../exo7formules}


\begin{document}
	
%%%%%%%%%%%%%%%%%%%%%%%%%%%%%%%%%%%%%%%%%%
\section{Courbes paramétrées}

\begin{multicols}{2}


%-----------------------------------------
\subsection{Notions de base}

Une \defi{courbe paramétrée plane}\index{courbe!parametree@paramétrée} est une application
$$\begin{array}[t]{cccc}
	f :&D\subset\Rr&\rightarrow&\Rr^2\\
	&t&\mapsto&f(t)
\end{array}$$ d'un sous-ensemble $D$ de $\Rr$ dans $\Rr^2$.

On note aussi $f(t) = M(t) = \left(
\begin{smallmatrix}
	x(t)\\
	y(t)
\end{smallmatrix}
\right)$.


\myfigure{0.5}{
	\tikzinput{fig_courbes_part1_00}
}


%---------------------------------------------------------------
\textbf{Réduction du domaine d'étude}

On utilise des transformations pour réduire le domaine
d'étude d'une courbe paramétrée.
Voici l'effet des transformations usuelles sur le point $M(x,y)$ (dans un repère orthonormé $(O,\vec{i},\vec{j})$).

\begin{itemize}
	\item Translation de vecteur $\vec{u}(a,b)$ : $t_{\vec{u}}(M)=(x+a,y+b)$.
	\item Réflexion d'axe $(Ox)$ : $s_{(Ox)}(M)=(x,-y)$.
	\item Réflexion d'axe $(Oy)$ : $s_{(Oy)}(M)=(-x,y)$.
	\item Symétrie centrale de centre $O$ : $s_O(M)=(-x,-y)$.
	\item Symétrie centrale de centre $I(a,b)$ : $s_I(M)=(2a-x,2b-y)$.
	\item Réflexion d'axe la droite $(D)$ d'équation $y=x$ : $s_D(M)=(y,x)$.
	\item Réflexion d'axe la droite $(D')$ d'équation $y=-x$ : $s_{D'}(M)=(-y,-x)$.
	\item Rotation d'angle $\frac{\pi}{2}$ autour de $O$ : $\text{rot}_{O,\pi/2}(M)=(-y,x)$.
	\item Rotation d'angle $-\frac{\pi}{2}$ autour de $O$ : $\text{rot}_{O,-\pi/2}(M)=(y,-x)$.
\end{itemize}


%---------------------------------------------------------------
\textbf{Points simples, points multiples}


La \defi{multiplicité}\index{courbe!multiplicte@multiplicité} du point $A$ par rapport à la courbe $f$ est le nombre de réels $t$ pour lesquels $M(t)=A$.


\myfigure{0.5}{
	\tikzinput{fig_courbes_part1_09b}
}

On dit aussi \defi{point simple} (multiplicité $1$), \defi{point double} (multiplicité $2$)\ldots{}  
Une \defi{courbe paramétrée simple} est une courbe dont tous les points sont de multiplicité $1$, c'est-à-dire $t\mapsto M(t)$ est injective.

Pour trouver les points multiples d'une courbe, 
on cherche les couples $(t,u)\in D^2$ tels que $t>u$ et 
$M(t)=M(u)$.


%-----------------------------------------
\subsection{Tangente à une courbe paramétrée}


\begin{itemize}
	\item Une courbe admet une \defi{tangente} en $M(t_0)$ si la droite $(M(t_0)M(t))$
	admet une position limite quand $t$ tend vers $t_0$.
	
	
%	\myfigure{0.6}{
%		\tikzinput{fig_courbes_part2_03}
%	}
	
	
	\item Une courbe paramétrée $t\mapsto M(t)=(x(t),y(t))$ 
	est \defi{dérivable} si les fonctions $x$ et $y$ le sont. 
	Le \defi{vecteur dérivé}\index{vecteur derive@vecteur dérivé} de la courbe en $t_0$ est
	$\overrightarrow{\dfrac{\dd M}{\dd t}}(t_0) = 
	\left(
	\begin{matrix}
		x'(t_0)\\
		y'(t_0)
	\end{matrix}
	\right)$.
	
	\item Si $\overrightarrow{\frac{\dd M}{\dd t}}(t_0)\neq\vec{0}$,
	le point $M(t_0)$ est dit \defi{régulier}.
	
	\item Si $\overrightarrow{\frac{\dd M}{\dd t}}(t_0)=\vec{0}$,
	le point $M(t_0)$ est dit \defi{singulier}.
\end{itemize}

\begin{theoreme}
	En tout point régulier d'une courbe dérivable, cette courbe admet une tangente.
	La tangente en un point régulier est dirigée par le vecteur dérivé en ce point.
\end{theoreme}

\myfigure{0.6}{
	\tikzinput{fig_courbes_part2_05}
}





%-----------------------------------------
\subsection{Points singuliers -- Branches infinies}



%---------------------------------------------------------------
\textbf{Tangente en un point singulier}



En un point singulier le vecteur dérivé est nul, il n'est d'aucune utilité pour
la recherche d'une tangente.

En un point $M(t_0)$ singulier, on étudie
$\displaystyle \lim_{t\rightarrow t_0}\frac{y(t)-y(t_0)}{x(t)-x(t_0)}$.
Si cette limite est un réel $\ell$, la tangente en $M(t_0)$
existe et a pour coefficient directeur $\ell$.
Si cette limite existe mais est infinie, la tangente en $M(t_0)$
existe et est verticale.





%---------------------------------------------------------------
\textbf{Position d'une courbe par rapport à sa tangente}


Quand la courbe arrive en $M(t_0)$, le long de sa tangente, on a
plusieurs possibilités :
\begin{itemize}
	\item la courbe continue dans le même sens, sans traverser la tangente :
	c'est un \defi{point d'allure ordinaire}\index{point!d'allure ordinaire},
	
	\item la courbe continue dans le même sens, en traversant la tangente :
	c'est un \defi{point d'inflexion}\index{point!d'inflexion},
	
	\item la courbe rebrousse chemin le long de
	cette tangente en la traversant, c'est un
	\defi{point de rebroussement de première espèce}\index{point!de rebroussement},
	
	\item la courbe rebrousse chemin le long de
	cette tangente sans la traverser, c'est un
	\defi{point de rebroussement de seconde espèce}.
	
\end{itemize}

%\myfigure{0.6}{
%	\begin{tabular}{cc}
%		\tikzinput{fig_courbes_part3_02}&
%		\tikzinput{fig_courbes_part3_03}\\[3mm]
%		\tikzinput{fig_courbes_part3_04}&
%		\tikzinput{fig_courbes_part3_05}\\
%	\end{tabular}
%}


Pour déterminer la position de la courbe par rapport à sa tangente
en un point singulier $M(t_0)$,
on effectue un DL des coordonnées de $M(t) = \big(x(t),y(t)\big)$
au voisinage de $t=t_0$.
Supposons $t_0=0$ et 
$$M(t) = M(0) + t^p \vec{v} + t^q \vec{w} +t^q \vec{\epsilon}(t)$$
où :
\begin{itemize}
	\item $p<q$ sont des entiers,
	\item $\vec{v}$ et $\vec{w}$ sont des vecteurs non colinéaires,
	\item $\vec{\epsilon}(t)$ est un vecteur, tel que $\|\vec{\epsilon}(t)\| \to 0$
	lorsque $t\to t_0$.
\end{itemize}

En un tel point $M(0)$, la courbe $\mathcal{C}$ admet une tangente, dont un vecteur directeur est
$\vec{v}$. La position de la courbe $\mathcal{C}$ par rapport à cette tangente
est donnée par la parité de $p$ et $q$ :

\myfigure{0.7}{
	\begin{tabular}{cc}
		\tikzinput{fig_courbes_part3_06}&
		\tikzinput{fig_courbes_part3_07}\\[3mm]
		\tikzinput{fig_courbes_part3_08}&
		\tikzinput{fig_courbes_part3_09}\\
	\end{tabular}
}



%---------------------------------------------------------------
\textbf{Branches infinies}

Dans ce paragraphe, la courbe $f : t\mapsto M(t)$ est définie sur un intervalle
$I$ de $\Rr$ et $t_0$ désigne l'une des bornes de $I$ et n'est pas dans $I$
($t_0$ est soit un réel, soit $-\infty$, soit $+\infty$).


Il y a \defi{branche infinie}\index{branche infinie} en $t_0$ dès que l'une au moins des deux fonctions $|x|$ ou $|y|$ tend vers l'infini quand $t$ tend vers $t_0$.
Il revient au même de dire que $\lim_{t \to t_0}\|f(t)\|=+\infty$.

La droite d'équation $y = ax+b$ est \defi{asymptote}\index{asymptote} à $\mathcal{C}$
si $y(t) - \big( a x(t) + b \big) \to 0$, lorsque $t \to t_0$.

\begin{enumerate}
	\item Si, quand $t$ tend vers $t_0$, $x(t)$ tend vers $+\infty$ (ou
	$-\infty$) et $y(t)$ tend vers un réel $\ell$, la droite d'équation
	$y=\ell$ est \defi{asymptote horizontale} à $\mathcal{C}$.
	
	\item Si, quand $t$ tend vers $t_0$, $y(t)$ tend vers $+\infty$ (ou
	$-\infty$) et $x(t)$ tend vers un réel $\ell$, la droite d'équation
	$x=\ell$ est \defi{asymptote verticale} à $\mathcal{C}$.
 
   \item La droite d'équation
	$y=ax+b$ est \defi{asymptote oblique} à la courbe $\big(x(t),y(t) \big)$ si :
	\begin{enumerate}
		\item $\frac{y(t)}{x(t)}$ tend vers un réel non nul $a$,
		\item $y(t)-ax(t)$ tend vers un réel $b$ (nul ou pas).
	\end{enumerate}
\end{enumerate}


\myfigure{0.27}{
	\tikzinput{fig_courbes_part3_14}
	\tikzinput{fig_courbes_part3_15}
	\tikzinput{fig_courbes_part3_16}
}

Attention ! Une branche infinie peut ne pas admettre de droite asymptote, comme dans le cas d'une parabole.


%-----------------------------------------
\subsection{Plan d'étude d'une courbe paramétrée}

\begin{enumerate}
	\item Domaine de définition de la courbe.
	
	\item Vecteur dérivé.
		
	\item Tableau de variations conjointes.
		
	\item Étude des points singuliers.
	
	\item Étude des branches infinies.
	
	\item Construction méticuleuse de la courbe.
	
	\item Points multiples.
	
\end{enumerate}


%-----------------------------------------
\subsection{Courbes en polaires : théorie}


%---------------------------------------------------------------
\textbf{Coordonnées polaires}


Le plan est rapporté à un repère orthonormé
$(O,\overrightarrow{i},\overrightarrow{j})$. Pour $\theta$ réel,
on pose
$$\overrightarrow{u_\theta}=\cos\theta\overrightarrow{i}+\sin\theta\overrightarrow{j}
\quad \text{ et  } \quad
\overrightarrow{v_\theta}=-\sin\theta\overrightarrow{i}+\cos\theta\overrightarrow{j}
=\overrightarrow{u_{\theta+\pi/2}}.$$

$$M = [r:\theta]
\iff \overrightarrow{OM} =r\overrightarrow{u_\theta}
\iff M=O+r\overrightarrow{u_\theta}.$$

\myfigure{0.5}{
	\tikzinput{fig_courbes_part5_01} \ 
	\tikzinput{fig_courbes_part5_02}
}


%---------------------------------------------------------------
% \subsection{Courbes d'équations polaires}

La courbe d'\defi{équation polaire} $r=f(\theta)$
est l'application suivante, où les coordonnées des
points sont données en coordonnées polaires :
$$\begin{array}{cccc}
	F :&D&\rightarrow&\Rr^2\\
	&\theta&\mapsto&M(\theta) = \big[r(\theta):\theta\big] = O+r(\theta)\vec{u}_\theta
\end{array}
$$



Exemple. Spirale d'équation polaire $r = \sqrt{\theta}$, pour $\theta \in [0,+\infty[$.

\myfigure{0.5}{
	\tikzinput{fig_courbes_part5_03}
}




%---------------------------------------------------------------
\textbf{Calcul de la vitesse en polaires}

$$\frac{\dd\overrightarrow{u_\theta}}{\dd\theta}
=\overrightarrow{v_\theta} \qquad\frac{\dd\overrightarrow{v_\theta}}{\dd\theta}
=-\overrightarrow{u_\theta}$$


%---------------------------------------------------------------
% \textbf{Tangente en un point distinct de l'origine}


\begin{theoreme}[Tangente en un point distinct de l'origine]
	\index{courbe!tangente}
	\sauteligne
	\begin{enumerate}
		\item Tout point de $\mathcal{C}$ distinct de l'origine $O$ est un point régulier.
		
		\item Si $M(\theta)\neq O$, la tangente en $M(\theta)$ est dirigée
		par le vecteur
		\myboxinline{$\overrightarrow{\dfrac{\dd M}{\dd \theta}}(\theta)
			=r'(\theta)\overrightarrow{u_\theta}+r(\theta)\overrightarrow{v_\theta}$}
		
		\item L'angle $\beta$ entre le vecteur $\overrightarrow{u_\theta}$ et la tangente
		en $M(\theta)$ vérifie \myboxinline{$\tan(\beta)=\frac{r}{r'}$} si $r'\neq 0$,
		et $\beta=\frac\pi2 \pmod \pi$ sinon.
		
	\end{enumerate}
\end{theoreme}


\myfigure{0.8}{
	\tikzinput{fig_courbes_part5_08}
}



%---------------------------------------------------------------
% \textbf{}


\begin{theoreme}[Tangente à l'origine]
	\index{courbe!tangente}
	Si $M(\theta_0)=O$, la tangente en $M(\theta_0)$ est la droite
	d'angle polaire $\theta_0$.
\end{theoreme}

\myfigure{0.6}{
	\tikzinput{fig_courbes_part5_05}
}





%-----------------------------------------
\subsection{Courbes en polaires : exemples}

%---------------------------------------------------------------
\textbf{Réduction du domaine d'étude}

Le plan est rapporté à un repère orthonormé direct,
$M$ étant le point de coordonnées polaires $[r:\theta]$.

\begin{itemize}
	\item Réflexion d'axe $(Ox)$. $s_{(Ox)} : [r:\theta] \mapsto [r:-\theta]$.
	
	\item Réflexion d'axe $(Oy)$. $s_{(Oy)} : [r:\theta] \mapsto [r:\pi-\theta]$.
	
	\item Symétrie centrale de centre $O$. $s_O : [r:\theta] \mapsto [r:\theta+\pi]=[-r:\theta]$.
	
	\item Réflexion d'axe la droite $D$ d'équation $(y=x)$.
	$s_D(M) : [r:\theta] \mapsto [r:\frac{\pi}{2}-\theta]$.
	
	\item Réflexion d'axe la droite $D'$ d'équation $(y=-x)$.
	$s_{D'}(M) : [r:\theta] \mapsto [-r:\frac{\pi}{2}-\theta]=[r:-\frac{\pi}{2}-\theta]$.
	
	\item Rotation d'angle $\frac{\pi}{2}$ autour de $O$.
	$r_{O,\pi/2} : [r:\theta] \mapsto [r:\theta+\frac{\pi}{2}]$.
	
	\item Rotation d'angle $\varphi$ autour de $O$.
	$r_{O,\varphi} : [r:\theta] \mapsto [r:\theta+\varphi]$.
\end{itemize}

\medskip 

%---------------------------------------------------------------
\textbf{Plan d'étude}
\begin{enumerate}
	\item Domaine de définition.
	
	\item Passages par l'origine. 
	
	\item Variations de $r$. 
	
	\item Tangentes parallèles aux axes.
	
	\item Étude des branches infinies.
	
	\item Construction de la courbe.
	\begin{itemize}
		\item Si $r$ est positif et croît, on tourne dans le sens direct en s'écartant de l'origine.
		\item Si $r$ est négatif et décroît, on tourne dans le sens direct en s'écartant de l'origine.
		\item Si $r$ est positif et décroît, on tourne dans le sens direct en se rapprochant de l'origine.
		\item Si $r$ est négatif et croît, on tourne dans le sens direct en se rapprochant de l'origine.
	\end{itemize}
	\item Points multiples.
	
\end{enumerate}







\end{multicols}

\end{document}




