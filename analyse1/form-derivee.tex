\documentclass[10pt,class=article,crop=false]{standalone}
\usepackage{../exo7formules}


\begin{document}
	
%%%%%%%%%%%%%%%%%%%%%%%%%%%%%%%%%%%%%%%%%%
\section{Dérivée d'une fonction}

\begin{multicols}{2}
	

%-----------------------------------------
\subsection{Définition}


%---------------------------------------------------------------
\textbf{Dérivée en un point}

Soit $I$ un intervalle ouvert de $\Rr$ et $f : I \to \Rr$ une fonction. Soit $x_0 \in I$.
$f$ est \defi{dérivable en $x_0$} si la limite suivante existe :
\mybox{$\displaystyle f'(x_0)= \lim_{x\to x_0} \frac{f(x)-f(x_0)}{x-x_0}$}

\begin{proposition}
Si $f$ est dérivable en $x_0$ alors $f$ est continue en $x_0$.
\end{proposition}

La réciproque est \textbf{fausse} : par exemple, la fonction valeur absolue
est continue en $0$ mais n'est pas dérivable en $0$.

%\myfigure{1}{
%	\tikzinput{fig_derive03}
%}

%---------------------------------------------------------------
\textbf{Tangente}
Une équation de la \defi{tangente}\index{tangente} au point $(x_0,f(x_0))$ est :
\mybox{$y = (x-x_0) f'(x_0) + f(x_0)$}

%\myfigure{2}{
%	\tikzinput{fig_derive02}
%}



%-----------------------------------------
\subsection{Calculs des dérivées}

%---------------------------------------------------------------
%\textbf{Somme, produit,...}


	\mybox{$(u+v)'=u'+v' \qquad  (\lambda u)' = \lambda u' \qquad (u \times v)' = u'v+uv'$}
	\mybox{$\displaystyle\left(\frac{1}{u}\right)'=-\frac{u'}{u^2} \qquad
		\left(\frac{u}{v}\right)'=\frac{u'v-uv'}{v^2}$}



%---------------------------------------------------------------
\textbf{Dérivée de fonctions usuelles}

\index{fonction!derivee@dérivée}
\begin{center}
	%\noindent
	\setlength{\arrayrulewidth}{0.05mm}
	%\begin{tabular}{|l|l|l|} \hline
	\begin{tabular}[t]{|c|c@{\vrule depth 1.2ex height 3ex width 0mm \ }|}
		\hline
		\textbf{Fonction}         & \textbf{Dérivée} \\ \hline
		$x^n$         & $nx^{n-1}$  \quad ($n \in \Zz$)   \\ \hline
		$\frac 1x$    & $-\frac{1}{x^2}$              \\ \hline
		$\sqrt{x}$    & $\frac12 \frac1{\sqrt{x}}$   \\ \hline
		$x^\alpha$   & $\alpha x^{\alpha-1}$  \quad ($\alpha\in\Rr$)  \\ \hline
		$e^x$         & $e^x$                        \\ \hline
		$\ln x$       & $\frac 1x$                   \\ \hline
		$\cos x$      & $-\sin x$                    \\ \hline
		$\sin x$      & $\cos x$                     \\ \hline
		$\tan x$      & $1+\tan^2 x = \frac{1}{\cos^2 x}$        \\ \hline
	\end{tabular}
	\hspace*{0.2cm}
	\begin{tabular}[t]{|c|c@{\vrule depth 1.2ex height 3ex width 0mm \ }|}
		\hline
		\textbf{Fonction}         & \textbf{Dérivée} \\ \hline
		$u^n$         & $nu'u^{n-1}$  \quad  ($n \in \Zz$)   \\ \hline
		$\frac 1u$    & $-\frac{u'}{u^2}$              \\ \hline
		$\sqrt{u}$    & $\frac12 \frac{u'}{\sqrt{u}}$   \\ \hline
		$u^\alpha$   & $\alpha u' u^{\alpha-1}$ \quad ($\alpha\in\Rr$)  \\ \hline
		$e^u$         & $u'e^u$                        \\ \hline
		$\ln u$       & $\frac {u'}{u}$                   \\ \hline
		$\cos u$      & $-u'\sin u$                    \\ \hline
		$\sin u$      & $u'\cos u$                     \\ \hline
		$\tan u$      & $u'(1+\tan^2 u) = \frac{u'}{\cos^2 u}$        \\ \hline
	\end{tabular}
	\hfill
\end{center}


%---------------------------------------------------------------
\textbf{Composition}

	\mybox{$\big( g \circ f \big)'(x) = g'\big( f(x) \big) \cdot f'(x)$}



\begin{corollaire}
	Soit $I$ un intervalle ouvert. Soit $f : I \to J$ dérivable et bijective dont on note
	$f^{-1} : J \to I$ la bijection réciproque. Si $f'$ ne s'annule pas sur $I$ alors $f^{-1}$ est dérivable
	et on a pour tout $x \in J$ :
	\mybox{$\displaystyle \big(f^{-1}\big)'(x)= \frac{1}{f'\big( f^{-1}(x) \big)} $}
\end{corollaire}



Il peut être plus simple de retrouver la formule à chaque fois
en dérivant l'égalité $f\big( g(x) \big)  = x$
où $g=f^{-1}$ est la bijection réciproque de $f$.


%---------------------------------------------------------------
% \textbf{Dérivées successives}

\begin{theoreme}[Formule de Leibniz]
	\index{formule!de Leibniz}
	\sauteligne	
	$$\big( f \cdot g \big)^{(n)} = \sum_{k=0}^n \binom{n}{k} \ f^{(n-k)} \cdot g^{(k)}.$$
\end{theoreme}

Pour $n=1$ on retrouve $(f\cdot g)'= f' g + f g'$.
Pour $n=2$, on a $(f\cdot g)''= f''g + 2f' g' + fg''$.



%-----------------------------------------
\subsection{Extremum local, théorème de Rolle}


Soit $f : I \to \Rr$ une fonction définie sur un intervalle $I$.
	\begin{itemize}
		\item On dit que $x_0$ est un \defi{point critique}\index{point critique} de $f$ si $f'(x_0)=0$.
		
		\item On dit que $f$ admet un \defi{maximum local en $x_0$}\index{maximum} (resp. un \defi{minimum local
			en $x_0$}\index{minimum}) s'il existe un intervalle ouvert $J$ contenant $x_0$  tel que
		$$\text{pour tout } x\in I \cap J \quad f(x) \le f(x_0)$$
		(resp. $f(x) \ge f(x_0)$).
		\item On dit que $f$ admet un \defi{extremum local en $x_0$}\index{extremum} si $f$ admet un maximum
		local ou un minimum local en ce point.
	\end{itemize}


%\myfigure{0.8}{
%	\tikzinput{fig_derive04}
%}


\begin{theoreme}
	\label{th:extremum}
	Soit $I$ un intervalle ouvert et $f : I \to \Rr$ une fonction
	dérivable. Si $f$ admet un maximum local (ou un minimum local)
	en $x_0$ alors $f'(x_0)=0$.
\end{theoreme}

En d'autres termes, un maximum local (ou un minimum local) $x_0$ est toujours un point critique.
Géométriquement, au point $(x_0,f(x_0))$ la tangente au graphe est horizontale.

%\myfigure{1}{
%	\tikzinput{fig_derive05}
%}

La réciproque du théorème est fausse.
Par exemple la fonction $f : \Rr \to \Rr$, définie par $f(x)= x^3$
vérifie $f'(0)=0$ mais $x_0=0$ n'est ni maximum local ni un minimum local.



%---------------------------------------------------------------
% \subsection{Théorème de Rolle}

\begin{theoreme}[Théorème de Rolle]
	\index{theoreme@théorème!de Rolle}
	\label{th:rolle}
	Soit $f:[a,b] \to \Rr$ telle que
	\begin{itemize}
		\item $f$ est continue sur $[a,b]$,
		\item $f$ est dérivable sur $]a,b[$,
		\item $f(a)=f(b)$.
	\end{itemize}
	Alors il existe $c \in {}]a,b[$  tel que $f'(c)=0$.
\end{theoreme}
%
%\myfigure{1}{
%	\tikzinput{fig_derive09}
%}

Interprétation géométrique : il existe au moins un point du graphe de $f$ où la tangente est horizontale.


%-----------------------------------------
\subsection{Théorème des accroissements finis}


\begin{theoreme}[Théorème des accroissements finis]
	\index{theoreme@théorème!des accroissements finis}
	Soit $f:[a,b] \to \Rr$ une fonction continue sur $[a,b]$ et dérivable sur $]a,b[$.
	Il existe $c\in{}]a,b[$ tel que
	\mybox{$f(b)-f(a)= f'(c) \; (b-a)$}
\end{theoreme}

\myfigure{1}{
	\tikzinput{fig_derive10}
}

Interprétation géométrique : il existe au moins un point du graphe de $f$ où la tangente est
parallèle à la droite $(AB)$ où $A=(a,f(a))$ et $B=(b,f(b))$.



%---------------------------------------------------------------
%\textbf{Fonction croissante et dérivée}

\begin{corollaire}
	Soit $f:[a,b] \to \Rr$ une fonction continue sur $[a,b]$ et dérivable sur $]a,b[$.
	\begin{enumerate}
		\item \label{it:crois}$\forall x \in ]a,b[ \quad f'(x) \ge 0 \quad \iff \quad$ $f$ est croissante ;
		\item $\forall x \in ]a,b[ \quad f'(x) \le 0 \quad \iff \quad$ $f$ est décroissante ;
		\item  $\forall x \in ]a,b[ \quad f'(x) = 0 \quad \iff \quad$ $f$ est constante ;
		\item \label{it:stcrois} $\forall x \in ]a,b[ \quad f'(x) > 0 \quad \implies \quad$ $f$ est strictement croissante ;
		\item \label{it:stdec} $\forall x \in ]a,b[ \quad f'(x) < 0 \quad \implies \quad$ $f$ est strictement décroissante.
	\end{enumerate}
\end{corollaire}

La réciproque au point (\ref{it:stcrois}) (et aussi au  (\ref{it:stdec})) est fausse.
Par exemple la fonction $x \mapsto x^3$ est strictement croissante et pourtant sa dérivée s'annule en $0$.



%---------------------------------------------------------------
%\subsection{Inégalité des accroissements finis}


\begin{corollaire}[Inégalité des accroissements finis]
	\index{inegalite@inégalité!des accroissements finis}
	Soit $f : I \to \Rr$ une fonction dérivable sur un intervalle $I$ ouvert.
	S'il existe une constante $M$ telle que pour tout $x \in I$, $\big|f'(x)\big| \le M$ alors
	\mybox{$\forall x,y \in I \qquad \big| f(x)-f(y) \big| \le M |x-y|$}
\end{corollaire}


Exemple : \myboxinline{$|\sin x| \le |x|$} pour tout $x\in\Rr$.
Preuve : Soit $f(x)=\sin(x)$. Comme $f'(x)=\cos x$ alors $|f'(x)| \le 1$ pour tout $x\in \Rr$.
L'inégalité des accroissements finis s'écrit alors $|\sin x - \sin y | \le |x-y|$.
On conclut en fixant $y=0$.


%---------------------------------------------------------------
%\subsection{Règle de l'Hospital}


\begin{corollaire}[Règle de l'Hospital]
	\index{regle@règle!de l'Hospital}
	Soient $f,g : I \to \Rr$ deux fonctions dérivables et soit $x_0\in I$.
	On suppose que
	\begin{itemize}
		\item $f(x_0)=g(x_0)=0$,
		\item $\forall x \in I\setminus\{x_0\} \quad g'(x)\neq0$.
	\end{itemize}
	\mybox{Si \quad $\displaystyle \lim_{x\to x_0} \frac{f'(x)}{g'(x)} = \ell \quad (\in \Rr)$
		\quad alors \quad $\displaystyle \lim_{x\to x_0} \frac{f(x)}{g(x)} = \ell.$}
\end{corollaire}


\end{multicols}

\end{document}



