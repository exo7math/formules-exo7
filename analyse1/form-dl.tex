\documentclass[10pt,class=article,crop=false]{standalone}
\usepackage{../exo7formules}


\begin{document}
	
%%%%%%%%%%%%%%%%%%%%%%%%%%%%%%%%%%%%%%%%%%
\section{Développements limités}

\begin{multicols}{2}

%-----------------------------------------
\subsection{Formules de Taylor}

Les trois formules de Taylor s'écrivent sous la forme
$f(x) = T_n(x) + R_n(x)$
où $T_n(x)$ est toujours le même polynôme de Taylor :
$$T_n(x) =f(a)+f'(a)(x-a)+\frac{f''(a)}{2!}(x-a)^2+\cdots
+\frac{f^{(n)}(a)}{n!}(x-a)^n.$$

C'est l'expression du reste $R_n(x)$ qui change :

\textbf{Taylor-Young} (la plus utile), $f$ de classe $\mathcal{C}^{n}$ :
$$R_n(x) = (x-a)^n\epsilon(x) \quad \text{ avec } \epsilon(x) \xrightarrow[x\to a]{} 0$$

\textbf{Taylor avec reste intégral}, $f$ de classe $\mathcal{C}^{n+1}$ :
$$R_n(x) = \int_a^x \frac{f^{(n+1)}(t)}{n!}(x-t)^ndt$$

\textbf{Taylor avec reste $f^{(n+1)}(c)$}, $f$ est $n+1$ fois dérivable :
$$R_n(x) = \frac{f^{(n+1)}(c)}{(n+1)!}(x-a)^{n+1}$$




\textbf{Notations.}

\begin{itemize}
	\item  $f : I \to \Rr$ est une fonction de \defi{classe $\mathcal{C}^n$} si $f$ est $n$ fois dérivable sur $I$ et $f^{(n)}$ est continue.
	\item \og \defi{petit o} \fg{}. $o((x-a)^n)$ est une fonction telle que $\lim_{x\to a}\frac{o((x-a)^n)}{(x-a)^n}=0$.
	Donc le reste $(x-a)^n\epsilon(x)$ où $\epsilon(x) \xrightarrow[x\to 0]{} 0$
	est noté $o((x-a)^n)$.
\end{itemize}	


\textbf{Formule de Taylor-Young au voisinage de $0$.}

$$f(x)= f(0)+f'(0)x+f''(0)\frac{x^2}{2!}+\cdots
+f^{(n)}(0)\frac{x^n}{n!} + o(x^n)$$

\textbf{Approximations.}

Les restes sont de plus en plus petits lorsque $n$ croît.
Les graphes des polynômes $T_1, T_2, T_3,\ldots$ s'approchent de plus en plus du graphe de $f$. Ceci n'est vrai qu'autour de la valeur $a$ (Ci-dessous $f(x) = \ln(1+x)$ en $a=0$).

\myfigure{0.6}{
	\tikzinput{fig_dl02}
}


Approximation numérique de $\sin(0,01)$.
La formule de Taylor pour $f(x)=\sin x$ en $a=0$ à l'ordre $3$  :
$f(x)=0+1\cdot x +0\cdot \frac{x^2}{2!}-1\frac{x^3}{3!} + o(x^3)$,
Pour $x=0,01$ :
$\sin(0,01) \simeq 0,01 - \frac{(0,01)^3}{6}=0,00999983333\ldots$
	





%-----------------------------------------
\subsection{Développements limités au voisinage d'un point}

Soit $f : I \to \Rr$ une fonction sur $I$ un intervalle ouvert.
Pour $a\in I$ et $n\in \Nn$, on dit que $f$ admet un
\defi{développement limité}\index{developpement limite@développement limité} (\defi{DL}) au point $a$ et à l'ordre $n$, s'il existe
des réels $c_0, c_1,\ldots,c_n$ tels que pour tout $x\in I$ :
$$f(x)=c_0+c_1 (x-a)+\cdots+c_n(x-a)^n+o((x-a)^n).$$

La formule de Taylor-Young fournit des DL
en posant, pour $k=0,1,\ldots,n$ :
\mybox{$c_k = \frac{f^{(k)}(a)}{k!}$}


\begin{proposition}
Si $f$ admet un DL alors ce DL est unique.
\end{proposition}

Exemple : Si $f$ est paire (resp. impaire) alors la partie polynomiale de son DL en $0$
ne contient que des monômes de degrés pairs (resp. impairs).


%---------------------------------------------------------------
\textbf{DL des fonctions usuelles à l'origine}

\begin{center}
\mybox{$\exp x=1+\frac{x}{1!}+\frac{x^2}{2!}+\frac{x^3}{3!}+\cdots+\frac{x^n}{n!}
+x^n\epsilon(x)$}

\smallskip

$\ch x=1+\frac{x^2}{2!}+\frac{x^4}{4!}+\cdots+\frac{x^{2n}}{(2n)!}
+x^{2n+1}\epsilon(x)$

\smallskip

$\sh x=\frac{x}{1!}+\frac{x^3}{3!}+\frac{x^5}{5!}+\cdots
+\frac{x^{2n+1}}{(2n+1)!}
+x^{2n+2}\epsilon(x)$

\smallskip

$\cos x=1-\frac{x^2}{2!}+\frac{x^4}{4!}-\cdots+(-1)^n\frac{x^{2n}}{(2n)!}
+x^{2n+1}\epsilon(x)$

\smallskip

$\sin x=\frac{x}{1!}-\frac{x^3}{3!}+\frac{x^5}{5!}-\cdots
+(-1)^n\frac{x^{2n+1}}{(2n+1)!}
+x^{2n+2}\epsilon(x)$

\smallskip

\mybox{$\ln(1+x)=x-\frac{x^2}{2}+\frac{x^3}{3}-\cdots
+(-1)^{n-1}\frac{x^{n}}{n} +x^{n}\epsilon(x)$}

%\smallskip

\mybox{$(1+x)^{\alpha}=1+\alpha x+\frac{\alpha(\alpha-1)}{2!}x^2+\cdots
+\frac{\alpha(\alpha-1)...(\alpha-n+1)}{n!}x^n+x^n\epsilon(x)$}

% \smallskip

${\displaystyle \frac{1}{1+x}}=1-x+x^2-x^3+\cdots+(-1)^nx^n+x^n\epsilon(x)$

\smallskip

${\displaystyle \frac{1}{1-x}} = 1+x+x^2+\cdots+x^n+x^n\epsilon(x)$

\smallskip

$\sqrt{1+x}  =\ \ 1 + \frac{x}{2} - \frac{1}{8}x^2+ \cdots +
(-1)^{n-1} \frac{1\cdot1\cdot3\cdot5\cdots(2n-3)}{2^n n!}x^n\ \  + x^n\epsilon(x)$
\end{center}


%---------------------------------------------------------------
\textbf{DL des fonctions en un point quelconque}

On ramène le problème en $0$ avec le changement de variables $h=x-a$.

Exemple : DL de $f(x)=\exp x$ en $1$.
On pose $h=x-1$. Si $x$ est proche de $1$ alors $h$ est proche de $0$.
$\exp x 
= \exp( 1+ (x-1) ) 
= \exp(1) \exp (x-1) 
= e \exp h 
= e \left(1+h+ \frac{h^2}{2!} + \cdots  \right)
= e \left(1+(x-1)+\frac{(x-1)^2}{2!}+\cdots \right)$.


%-----------------------------------------
\subsection{Opérations sur les développements limités}

%---------------------------------------------------------------
\textbf{Somme, produit, composition.}
Soient $f$ et $g$ ayant des DL en $0$ à l'ordre $n$ :

$f(x)= C(x) + o(x^n) =c_0+c_1x + \cdots +c_nx^n + x^n\epsilon_1(x)$

$g(x)=D(c)+o(x^n)=d_0+d_1x + \cdots +d_nx^n + x^n\epsilon_2(x)$



\begin{proposition}
	\sauteligne
	\begin{itemize}
		\item $(f+g)(x)=C(x)+D(x) + o(x^n)$
		
		\item $(f \times g)(x) = C(x) \times D(x) + o(x^n)$ 
		
		\item Si $g(0)=0$ (c'est-à-dire $d_0=0$),
		$(f \circ g)(x) = C(D(x)) + o(x^n)$ 
	\end{itemize}
\end{proposition}

Pour le produit et la composition, il faut en plus 
\defi{tronquer} la partie polynomiale à l'ordre $n$, c-à-d conserver seulement
les monômes de degré $\le n$.


%---------------------------------------------------------------
\textbf{Division.}
Le DL d'un quotient se ramène au calcul de l'inverse $\frac{1}{1+u}$ :
$$\frac{1}{1+u} = 1-u+u^2-u^3+\cdots$$

%---------------------------------------------------------------
\textbf{Intégration.}
Soit $f$ une fonction de classe $\mathcal{C}^n$ dont le DL
en $0$ à l'ordre $n$ est $f(x)=c_0+c_1x+c_2x^2+\cdots+c_nx^n+o(x^n)$.
\begin{theoreme}
	Notons $F$ une primitive de $f$.
	Alors $F$ admet un DL en $0$ à l'ordre $n+1$ obtenu en intégrant terme à terme :
$$
		F(x)=F(0)+c_0 x + c_1\frac{x^2}{2}+ c_2\frac{x^3}{3}+\cdots 
		 +c_n\frac{x^{n+1}}{n+1}+o(x^{n+1})  
$$
\end{theoreme}

Cela signifie que l'on intègre la partie polynomiale terme à terme pour obtenir le DL de $F(x)$ à la constante $F(0)$ près.





%-----------------------------------------
\subsection{Applications des développements limités}


%---------------------------------------------------------------
\textbf{Calculs de limites.} 
Les DL sont très efficaces pour lever des formes indéterminées !

%---------------------------------------------------------------
\textbf{Position d'une courbe par rapport à sa tangente}

\begin{proposition}
	Soit $f : I \to \Rr$ une fonction admettant un DL en $a$ :
	$f(x)=c_0+c_1(x-a)+c_k(x-a)^k+(x-a)^k\epsilon(x)$,
	où $c_k \neq 0$.
	Alors l'équation de la tangente à la courbe de $f$ en $a$ est : $y=c_0+c_1(x-a)$ et
	la position de la courbe par rapport à la tangente pour $x$ proche de $a$ est
	donnée par le signe $f(x)-y$, c'est-à-dire le signe de $c_k(x-a)^k$.
\end{proposition}


%Il y a $3$ cas possibles.
\begin{itemize}
	\item Si ce signe est positif alors la courbe est au-dessus de la tangente.
	\item Si ce signe est négatif alors la courbe est en dessous de la tangente.
	\item Si ce signe change alors la courbe traverse
la tangente au point d'abscisse $a$. C'est un \defi{point d'inflexion}\index{point d'inflexion}.	
\end{itemize}	
	\myfigure{0.4}{
		\tikzinput{fig_dl03}
		\tikzinput{fig_dl04}
		\tikzinput{fig_dl05}				
	}


%---------------------------------------------------------------
\textbf{Développement limité en $+\infty$ (développement asymptotique)}

$f : {}]x_0,+\infty[ \to \Rr$ admet un \defi{DL en $+\infty$}\index{developpement limite@développement limité!en l'infini@en $+\infty$} à l'ordre $n$
s'il existe des réels $c_0,c_1,\ldots,c_n$ tels que
$f(x)=c_0+\frac{c_1}{x}+\cdots+\frac{c_n}{x^n}
+\frac{1}{x^n}\epsilon\big(\frac{1}{x}\big)$
où $\epsilon\big(\frac{1}{x}\big)$ tend vers $0$ quand $x\to+\infty$.

Cela équivaut à ce que $x \to f(\frac{1}{x})$ admet un DL en $0^+$ à l'ordre $n$.


\begin{proposition}
	Si 
	$\frac{f(x)}{x}= a_0 +\frac{a_1}{x}+\frac{a_k}{x^k}+\frac{1}{x^k}\epsilon(\frac{1}{x}),$
	où $a_k \neq 0$.
	Alors $\lim_{x\to+\infty} f(x)-(a_0x+a_1) =0$ (resp. $x\to -\infty$) : la droite $y= a_0x+a_1$ est une 
	\defi{asymptote}\index{asymptote}
	à la courbe de $f$ en $+\infty$ (ou $-\infty$) et la position de la courbe par rapport à
	l'asymptote est donnée par le signe de $f(x)-y$, c'est-à-dire le signe de $\frac{a_k}{x^{k-1}}$.
\end{proposition}

% \myfigure{0.6}{
% 	\tikzinput{fig_dl08}
% }



\end{multicols}

\end{document}


