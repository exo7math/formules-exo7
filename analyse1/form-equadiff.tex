\documentclass[10pt,class=article,crop=false]{standalone}
\usepackage{../exo7formules}


\begin{document}

%%%%%%%%%%%%%%%%%%%%%%%%%%%%%%%%%%%%%%%%%%
\section{Équations différentielles}

\begin{multicols}{2}

%-----------------------------------------
\subsection{Définitions}

\begin{itemize}
	\item Une \defi{équation différentielle}\index{equation differentielle@équation différentielle} d'ordre $n$ est une équation de la forme
	\begin{equation}
		F\left(x,y,y',\dots ,y^{(n)}\right)=0
		\label{eq:eqdiff}
		\tag{$E_{\text{diff}}$}
	\end{equation}
	où $F$ est une fonction de $(n+2)$ variables.
	
	\item Une \defi{solution} d'une telle équation sur un intervalle $I\subset \Rr$
	est une fonction $y :I \to \Rr$ qui est $n$ fois dérivable
	et qui vérifie l'équation (\ref{eq:eqdiff}). Si on change d'intervalle, on peut très bien obtenir d'autres solutions.


	\item  \emph{Exemple.} Une équation différentielle 
\defi{à variables séparées}\index{equation differentielle@équation différentielle!a variables separees@à variables séparées}
est une équation du type $y'=g(x)/f(y)$ ou $y'f(y)=g(x)$.
Une telle équation se résout par calcul de primitives de part et d'autre de l'égalité $y'f(y)=g(x)$.


	\item Une équation différentielle \defi{linéaire} est de la forme
	\begin{equation}
	a_0(x)y+a_1(x)y'+\dots +a_n(x)y^{(n)} = g(x)
	\label{eq:eqdifflinscnd}
	\tag{$E$}
	\end{equation}	
	où les $a_i$ et $g$ sont des fonctions continues sur un intervalle $I\subset \Rr$.
	
	\item Une équation différentielle linéaire est 
	\defi{homogène}\index{equation differentielle@équation différentielle!homogene@homogène}, ou 
	\defi{sans second membre}\index{equation differentielle@équation différentielle!sans second membre},
	si la fonction $g$ est nulle :
	\begin{equation}
	a_0(x)y+a_1(x)y'+\dots +a_n(x)y^{(n)} = 0
	\label{eq:eqdiffhomog}
	\tag{$E_h$}
    \end{equation}	

	
	\item Une équation différentielle linéaire est \defi{à coefficients constants} si
	les fonctions $a_i$ ci-dessus sont constantes :
	$$a_0y+a_1y'+\dots +a_ny^{(n)} = g(x)$$
	où les $a_i$ sont des constantes réelles et $g$ une fonction continue.
\end{itemize}

\begin{proposition}[Principe de linéarité]
	Si $y_1$ et $y_2$ sont solutions de l'équation différentielle linéaire homogène
	\eqref{eq:eqdiffhomog}
	alors, quels que soient $\lambda,\mu \in \Rr$, $\lambda y_1 + \mu y_2$ est aussi solution de cette équation.
\end{proposition}

\text{Méthode} pour résoudre une équation différentielle linéaire \eqref{eq:eqdifflinscnd} avec second membre ;
\begin{enumerate}
	\item Trouver une solution particulière $y_p$ de l'équation \eqref{eq:eqdifflinscnd}.
	\item Trouver l'ensemble $\mathcal{S}_h$ des solutions $y_h$ de l'équation homogène associée \eqref{eq:eqdiffhomog}.
    \item Conclure par le principe de linéarité : les solutions de \eqref{eq:eqdifflinscnd} sont les 
    $$y = y_p + y_h \quad \text{ avec } \quad  y_h \in \mathcal{S}_h.$$
\end{enumerate}



%-----------------------------------------
\subsection{Équation différentielle linéaire du premier ordre}

Une équation différentielle \defi{linéaire du premier ordre} est une équation du type $y'=a(x)y + b(x)$ où $a$ et $b$ sont des fonctions définies sur un intervalle ouvert $I$ de $\Rr$.

\begin{theoreme}[$y' = a y$]
	\label{th:eqdifflinordre1cst}
	Les solutions de $y' = a y$ où $a \in \Rr$ est une constante sont les fonctions $y$ définies sur $\Rr$ par :
	\mybox{$y(x) = k e^{ax}$}
	où $k\in \Rr$ est une constante quelconque.
\end{theoreme}

Preuve rapide : on intègre à gauche et à droite l'équation $\frac{y'}{y} =  a$
pour trouver : $\ln |y(x)| = ax+b$.
Donc $|y(x)| = e^{ax+b}$. Ainsi $y(x) = \pm e^b e^{ax}$.

Exemple : $3y' - 5y = 0$ a pour solution 
$y(x) = k e^{\frac53x}$, où $k \in \Rr$.


\begin{theoreme}[$y' = a(x) y$]
	\label{th:eqdifflinordre1}
	Soit $a : I \to \Rr$ une fonction continue. Soit $A : I \to \Rr$ une primitive de $a$. Les solutions de $y' = a(x) y$
	sont les fonctions $y$ définies par :
	\mybox{$y(x) = k e^{A(x)}$}
	où $k\in \Rr$ est une constante quelconque.
\end{theoreme}

Exemple : $x^2y'=y$ sur $I=\,]0,+\infty[$.
L'équation est $y'= \frac{1}{x^2}y$, donc $a(x)=\frac{1}{x^2}$, dont une primitive est $A(x)=-\frac1x$. Les solutions sont $y(x) = k e^{-\frac1x}$, où $k\in\Rr$.

\begin{theoreme}[$y' = a(x) y + b(x)$]
Soit l'équation $y' = a(x) y + b(x)$
où $a,b  : I \to \Rr$.
Soit $y_p$ une solution particulière et
$y_h(x)$ les solutions de l'équation homogène $y' = a(x) y$.
Les solutions sont les $y = y_p + y_h$.
\end{theoreme}




\textbf{Recherche d'une solution particulière : méthode de variation de la constante.}
\index{variation de la constante}
\begin{itemize}
	\item On trouve les solutions $y(x)=ke^{A(x)}$ de l'équation homogène $y' = a(x) y$ où $k$ est une constante.
	\item On cherche une solution particulière de $y' = a(x) y + b(x)$
	sous la forme $y_p(x)=k(x)e^{A(x)}$, où $k$ est maintenant une fonction.
	\item L'équation $y_p' = a(x) y_p + b(x)$ permet de déterminer $k'(x)$, puis $k(x)$.
\end{itemize}

	
\textbf{Recherche d'une solution particulière : cas des coefficients constants.}
$y' = ax + g(x)$, où $a \in \Rr^*$ est une constante.
Le principe est de chercher une solution particulière de la même forme que le second membre.
\begin{itemize}
	\item Si $g(x) = P(x)$ est un polynôme de degré $n$, on cherche une solution particulière sous la forme $y_p(x) = Q(x)$ où $Q$ est aussi un polynôme de degré $n$.
	\item Si $g(x) = ce^{\beta x}$, on cherche une solution particulière sous la forme $y_p(x) =de^{\beta x}$.
	
	\item Si $g(x) = c_1\cos(\beta x) + c_2 \sin(\beta x)$,  on cherche une solution particulière sous la forme $y_p(x) = d_1\cos(\beta x) + d_2 \sin(\beta x)$,	
\end{itemize}	


\begin{theoreme}[de Cauchy-Lipschitz]
	\index{theoreme@théorème!de Cauchy-Lipschitz}
	\label{th:cauchylipschitzord1}	
	Soit $y'=a(x)y + b(x)$ une équation différentielle linéaire du premier ordre,
	où $a,b : I \to \Rr$ sont des fonctions continues sur un intervalle ouvert $I$.
	Pour tout $x_0 \in I$ et pour tout $y_0 \in \Rr$, il existe une et une
	seule solution $y$ telle que $y'=a(x)y + b(x)$ et $y(x_0)=y_0$.
\end{theoreme}


%-----------------------------------------
\subsection{Équation différentielle linéaire du second ordre à coefficients constants}

\begin{equation}
	ay''+by'+cy=g(x)
	\label{eq:linscd}
	\tag{$E$}
\end{equation}
où $a,b,c \in \Rr$, $a \neq 0$ et $g$ continue sur $I$
 intervalle ouvert.

L'équation homogène associée
\begin{equation}
	ay''+by'+cy=0
	\label{eq:linscdhom}
	\tag{$E_h$}
\end{equation}

L'\defi{équation caractéristique} est $ar^2+br+c=0$, de discriminant $\Delta= b^2-4ac$.

\begin{theoreme}
	\sauteligne
	\begin{enumerate}
		
		\item Si $\Delta >0$, l'équation caractéristique a deux racines réelles distinctes
		$r_1\neq r_2$ et les solutions de (\ref{eq:linscdhom}) sont les
		\mybox{$y(x) = \lambda e^{r_1x}+ \mu e^{r_2x} \quad \text{ où } \lambda, \mu \in \Rr.$}
		
		\item Si $\Delta=0$, l'équation caractéristique a une racine double $r_0$
		et les solutions de (\ref{eq:linscdhom}) sont les
		\mybox{$y(x) = (\lambda+\mu x)e^{r_0 x} \quad \text{ où } \lambda, \mu \in \Rr.$}
		
		\item Si $\Delta<0$, l'équation caractéristique a deux racines complexes
		conjuguées $r_1=\alpha+\ii \beta$, $r_2=\alpha-\ii \beta$ et les solutions de (\ref{eq:linscdhom}) sont les
		\mybox{$y(x) = e^{\alpha x}\big(\lambda\cos (\beta x)+\mu\sin (\beta x)\big) \quad \text{ où }
			\lambda, \mu \in \Rr.$}
		
	\end{enumerate}
\end{theoreme}

(Attention ! $y''+y=0$ a pour équation caractéristique $r^2+1=0$.)

%---------------------------------------------------------------
\textbf{Équation avec second membre}


\begin{equation}
	ay''+by'+cy=g(x)
	%\label{eq:linscd}
	\tag{$E$}
\end{equation}

\begin{theoreme}[Théorème de Cauchy-Lipschitz]
	\index{theoreme@théorème!de Cauchy-Lipschitz}
	Pour chaque $x_0\in I$ et chaque couple $(y_0,y_1) \in \Rr^2$,
	l'équation (\ref{eq:linscd}) admet une \evidence{unique}
	solution $y$ sur $I$ satisfaisant aux conditions initiales :
	\mybox{$y(x_0) = y_0$ \quad et \quad $y'(x_0) = y_1$}
\end{theoreme}



\textbf{Second membre $g(x)=e^{\alpha x}P(x)$.} $\alpha \in \Rr$ et $P\in \Rr[X]$

Cela comprend le cas $g(x) = e^{\alpha x}$ (donc $P(x)=1$ et alors $Q(x)$ est une constante ci-dessous) et le cas $g(x) = P(x)$ (donc $\alpha=0$).

On cherche une solution particulière sous la forme
$y_p(x)=e^{\alpha x}x^{m}Q(x)$, où $Q$ est un polynôme de
même degré que $P$ avec :
\begin{itemize}
	\item $y_p(x)=e^{\alpha x}Q(x)$ ($m=0$), si $\alpha$ n'est pas une racine de l'équation caractéristique,
	\item $y_p(x)=xe^{\alpha x}Q(x)$ ($m=1$), si $\alpha$ est une racine simple de l'équation caractéristique,
	\item $y_p(x)=x^2e^{\alpha x}Q(x)$ ($m=2$), si $\alpha$ est une racine double de l'équation caractéristique.
\end{itemize}

\textbf{Second membre du type $e^{\alpha x}\big(P_1(x)\cos (\beta x)+P_2(x)\sin (\beta x)\big)$.}

Cela comprend le cas $g(x) = c_1\cos(\beta x) + c_2 \sin(\beta x)$ (donc $\alpha=0$ et $P_1$ et $P_2$ polynômes constants).

Si $g(x)=e^{\alpha x} \big(P_1(x)\cos (\beta x)+P_2(x)\sin (\beta x)\big)$,
où $\alpha ,\beta \in \Rr$ et $P_1,P_2\in\Rr[X]$, on cherche une solution particulière sous la forme :
\begin{itemize}
	\item $y_p(x)=e^{\alpha x} \big( Q_1(x)\cos (\beta x)+Q_2(x)\sin (\beta x) \big)$,
	si $\alpha +\ii \beta$ n'est pas une racine de l'équation caractéristique,
	
	\item $y_p(x)=xe^{\alpha x}  \big( Q_1(x)\cos (\beta x)+Q_2(x)\sin (\beta x) \big)$,
	si $\alpha +\ii \beta$ est une racine de l'équation caractéristique.
\end{itemize}
Dans les deux cas, $Q_1$ et $Q_2$ sont deux polynômes
de degré $n=\max\{\deg P_1,\deg P_2\}$.


\textbf{Méthode de variation des constantes.}
\index{variation de la constante}

Si $\{y_1,y_2\}$ est une base de solutions de l'équation homogène (\ref{eq:linscdhom}),
on cherche une solution particulière sous la forme
$y_p = \lambda y_1 + \mu y_2$, mais cette fois $\lambda$ et $\mu$ sont deux fonctions.



\end{multicols}

\end{document}	