\documentclass[10pt,class=article,crop=false]{standalone}
\usepackage{../exo7formules}


\begin{document}
	
%%%%%%%%%%%%%%%%%%%%%%%%%%%%%%%%%%%%%%%%%%
\section{Intégrales}

\begin{multicols}{2}
	


%-----------------------------------------
\subsection{L'intégrale de Riemann}

\textbf{Fonction en escalier}



Une fonction $f : [a,b] \to \Rr$ est une \defi{fonction en escalier}\index{fonction!en escalier}
s'il existe  une subdivision $(x_0,x_1,\ldots,x_n)$ et des nombres réels
$c_1,\ldots,c_n$ tels que pour tout $i\in \{1,\ldots,n\}$ on ait
$\forall x \in {}]x_{i-1},x_i[ \quad f(x)=c_i$.

Autrement dit $f$ est une fonction constante sur chacun des sous-intervalles de la subdivision.

L'intégrale d'une fonction en escalier est :
$$\displaystyle \int_a^b f(x) \; dx = \sum_{i=1}^n c_i(x_i-x_{i-1})$$

\myfigure{0.6}{
	\tikzinput{fig_int04}
}

Chaque terme $c_i(x_i-x_{i-1})$ est l'aire algébrique du rectangle compris entre les abscisses
$x_{i-1}$ et $x_i$ et de hauteur $c_i$.


\textbf{Fonction intégrable}

Notation : $f \le g$ signifie  $f(x) \le g(x)$ pour tout $x \in [a,b]$.

\bigskip

Soit $f : [a,b] \to \Rr$ bornée. On définit :
$$I^-(f) = \sup \left\{ \int_a^b \phi(x) \; dx \mid \phi \text{ en escalier et } \phi \le f \right\}$$
$$I^+(f) = \inf \left\{ \int_a^b \phi(x) \; dx \mid \phi \text{ en escalier et } \phi \ge f \right\}$$


\myfigure{0.6}{
	\tikzinput{fig_int05}
}

On a $I^-(f) \le I^+(f)$.

Une fonction bornée $f :[a,b] \to \Rr$ est dite \defi{intégrable}\index{fonction!integrable@intégrable} (\defi{au sens de Riemann})
si $I^-(f) = I^+(f)$. On note alors ce nombre $\int_a^b f(x)\; dx$.


% \textbf{Les fonctions continues sont intégrables}

\begin{theoreme}
	Si $f : [a,b] \to \Rr$ est continue alors $f$ est intégrable.
\end{theoreme}
Et aussi :
\begin{itemize}
  \item Si $f : [a,b] \to \Rr$ est continue par morceaux alors $f$ est intégrable.
  \item Si $f : [a,b] \to \Rr$ est monotone alors $f$ est intégrable.
\end{itemize}

%-----------------------------------------
\subsection{Propriétés}

Les fonctions sont supposées intégrables.

$$\int_a^a f(x) \;dx=0 \qquad \text{ et pour } a<b  \quad \int_b^a f(x) \;dx= -\int_a^b f(x) \; dx.$$

%---------------------------------------------------------------
\textbf{Relation de Chasles}

Pour $a,b,c$ quelconques :
\mybox{$\displaystyle\int_a^b f(x)\;dx = \int_a^c f(x)\;dx + \int_c^b f(x)\;dx$}


\textbf{Positivité de l'intégrale}

\begin{proposition}
	\index{integrale@intégrale!positivite@positivité}

\end{proposition}

L'intégrale d'une fonction positive est positive :
\mybox{Si \quad $f\ge 0$ \quad alors \quad $\displaystyle \int_a^bf(x)\;dx \ge 0$}

Variante :
	\mycenterline{Si $f\le g$ \quad alors \quad  $\displaystyle \int_a^b f(x)\;dx \le\int_a^b g(x)\;dx$}
	
	
%---------------------------------------------------------------
\textbf{Linéarité de l'intégrale}
\label{ssec:int23}

Pour tous réels $\lambda,\mu$
$$\int_a^b \big(\lambda f(x)+\mu g(x)\big)\;dx= \lambda\int_a^b f(x)\;dx+\mu\int_a^b g(x)\;dx$$
		

\mybox{$\displaystyle\left\vert\int_a^b f(x) \;dx\right\vert\le\int_a^b\big\vert f(x)\big\vert \;dx$}



%-----------------------------------------
\subsection{Primitive}

\textbf{Définition.}
Soit $f:I \to \Rr$ une fonction définie sur un intervalle.
$F : I \to \Rr$ est une \defi{primitive}\index{primitive} de $f$ si
$F$ est une fonction dérivable sur $I$ vérifiant $F'(x)=f(x)$ pour tout $x \in I$.

Trouver une primitive est donc l'opération inverse de calculer la fonction dérivée.
Trouver une primitive permet de les trouver toutes.

\begin{proposition}
\label{prop:primitunic}
Si $F : I \to \Rr$ est une primitive de $f$, alors toute primitive de $f$ s'écrit $G=F+c$ où $c\in \Rr$.
\end{proposition}

% Linéarité :
% $\int\big(\lambda f+ \mu g\big) \; dt=\lambda \int f+\mu \int g$


%---------------------------------------------------------------
\textbf{Primitives des fonctions usuelles}


\index{primitive}
\begin{center}
%\noindent
\setlength{\arrayrulewidth}{0.05mm}
%\begin{tabular}{|l|l|l|} \hline
%\begin{tabular}[t]{|c@{\vrule depth 1.2ex height 3ex width 0mm \ }|}
%\begin{tabular}{c@{\vrule depth 3ex height 3ex width 0mm \ }}
\begin{tabular}{c@{\vrule depth 3ex height 1ex width 0mm \ }}	
	$\int e^x \; dx  = e^x + c$  \quad sur $\Rr$ \\ 
	$\int \cos x \; dx  = \sin x  + c$  \quad sur $\Rr$ \\ 
	$\int \sin x \; dx  = -\cos x  + c$  \quad sur $\Rr$ \\ 
	$\int x^n \; dx = \frac{x^{n+1}}{n+1} + c$  \quad ($n \in \Nn$)  \quad sur $\Rr$ \\ 
	$\int x^\alpha \; dx = \frac{x^{\alpha+1}}{\alpha+1} + c$  \quad ($\alpha \in \Rr\setminus\{-1\}$)  sur $]0,+\infty[$\\ 
	$\int \frac 1x \; dx  = \ln |x|  + c$  \quad sur $]0,+\infty[$ ou $]-\infty,0[$ \\ 
	$\int\sh x \; dx=\ch x+c$, $\int \ch x \; dx=\sh x+c$ \quad sur $\Rr$ \\ 
	$\int \frac{dx}{1+x^2}= \arctan x+c$ \quad sur $\Rr$ \\ 
	$\int\frac{dx}{\sqrt{1-x^2}} = \left\{ \begin{array}{l}
		\arcsin x + c \\ \frac\pi2-\arccos x +c \end{array} \right.$ \quad  sur $]-1,1[$ \\ 
	$\int \frac{dx}{\sqrt {x^2+1}}=  \left\{ \begin{array}{l} \mbox{Argsh} x+c \\
		\ln\big(x+\sqrt{x^2+1}\big)+c  \end{array} \right.$ \quad sur $\Rr$ \\
	$\int \frac{dx}{\sqrt {x^2-1}} = \left\{ \begin{array}{l} \mbox{Argch} x+c \\
		\ln\big(x+\sqrt{x^2-1}\big)+c \end{array} \right.$ \quad sur $]1,+ \infty[$\\ 
\end{tabular}
\end{center}



%---------------------------------------------------------------
\textbf{Relation primitive-intégrale}

\begin{theoreme}
Soit $f : [a,b] \to \Rr$ une fonction continue.
La fonction $F:I \to \Rr$ définie par
\mybox{$\displaystyle F(x)=\int_a^x f(t) \; dt$}
est une primitive de $f$, c'est-à-dire $F$ est dérivable et $F'(x)=f(x)$.

Par conséquent pour une primitive $F$ quelconque de $f$:
\mybox{$\displaystyle \int_a^b f(t) \; dt = F(b)-F(a)$}
\end{theoreme}

\textbf{Notation.} $\big[F(x)\big]_a^b=F(b)-F(a)$.
Si l'on omet les bornes alors $\big[F\big]$ désigne la fonction $F+c$ où $c$ est une constante quelconque.

Remarques :
\begin{enumerate}
	
	\item $F(x)=\int_a^x f(t) \; dt$ est même \evidence{l'unique primitive de $f$ qui s'annule en $a$}.
	
	\item En particulier si $F$ est une fonction de classe $\mathcal{C}^1$ alors
	\myboxinline{$\int_a^b F'(t) \; dt = F(b)-F(a)$}.
	
\end{enumerate}



%---------------------------------------------------------------
\textbf{Sommes de Riemann}


\begin{theoreme}
Soit $f : [a,b] \to \Rr$ une fonction intégrable, alors 
\mybox{$\displaystyle S_n = \frac{b-a}{n} \sum_{k=1}^{n} f\big(a+k\frac{b-a}{n} \big)
	\qquad \xrightarrow[n\to+\infty]{} \qquad \int_a^b f(x) \; dx$}
\end{theoreme}


Cas particulier $a=0$, $b=1$, $\frac{b-a}{n}=\frac1n$ et
$f\big(a+k\frac{b-a}{n}\big) = f\big(\frac kn\big)$ : 
$$S_n = \frac{1}{n} \sum_{k=1}^{n} f\big(\frac kn \big)
\qquad \xrightarrow[n\to+\infty]{} \qquad \int_0^1 f(x) \; dx$$

\myfigure{0.6}{
\tikzinput{fig_int10}
}




%-----------------------------------------
\subsection{Intégration par parties - Changement de variable}

%---------------------------------------------------------------
\textbf{Intégration par parties}

\begin{theoreme}
	Soient $u$ et $v$ deux fonctions de classe $\mathcal{C}^1$ sur un intervalle $[a,b]$.
	\mybox{$\displaystyle\int_a^b u(x) \, v'(x)\;dx= \big[uv\big]_a^b - \int_a^b u'(x) \, v(x)\;dx$}
\end{theoreme}

Pour les primitives :
$$\int u(x)v'(x)\;dx= \big[uv\big] - \int u'(x)v(x)\;dx$$

%---------------------------------------------------------------
\textbf{Changement de variable}
\begin{theoreme}
	Soit $f$ une fonction définie sur un intervalle $I$ et $\varphi : J \to I$ une bijection de classe $\mathcal{C}^1$.
	Pour tout $a,b\in J$
	\mybox{$\displaystyle\int_{\varphi(a)}^{\varphi(b)} f(x) \; dx = \int_a^b f\big(\varphi(t)\big)\cdot\varphi'(t) \; dt$}
	
	Si $F$ est une primitive de $f$ alors $F\circ \varphi$ est une primitive de
	$\big(f \circ \varphi\big)\cdot\varphi'$.
	%Autrement dit: \mybox{$\displaystyle \left( \int f(x) \; dx \right) \circ \varphi
		%= \int f\big(\varphi(t)\big)\varphi'(t) \; dt$}
\end{theoreme}

\textbf{Méthodologie.}
\begin{itemize}
	\item Trouver le bon changement de variable $u =\varphi(x)$.
	\item Effectuer le changement de l'élément différentiel $du = \varphi'(x) \; dx$.
	\item Effectuer le changement de bornes : comme $x$ varie de \ldots{} à \ldots{} alors $u$ varie de \ldots{} à \ldots{}
	\item Appliquer la formule de changement de variable.
\end{itemize}
	
\begin{exemple}
	Calcul de $\int_0^{1/2}\frac{x}{(1-x^2)^{3/2}} \;dx$.
	
	Soit le changement de variable $u=\varphi(x) = 1-x^2$. Alors $du = \varphi'(x) \; dx = -2x \; dx$.
	Pour $x=0$ on a $u=\varphi(0)=1$ et pour $x=\frac12$ on a $u=\varphi(\frac{1}{2})=\frac34$.
	Comme $\varphi'(x)=-2x$, $\varphi$ est une bijection de $[0,\frac{1}{2}]$ sur $[1,\frac{3}{4}]$. Alors
	\begin{eqnarray*}
		\int_0^{1/2}\frac{x \; dx}{(1-x^2)^{3/2}} 
		&=& \int_1^{3/4} \frac{\frac{-du}{2}}{u^{3/2}}
		= -\frac12\int_1^{3/4} u^{-3/2}\;du \\
		&=& -\frac12\big[-2u^{-1/2}\big]_1^{3/4}
		%=\big[\frac1{\sqrt{u}}\big]_1^{3/4} 
		= \frac1{\sqrt{\frac34}}-1= \frac{2}{\sqrt3}-1.
	\end{eqnarray*}
\end{exemple}


%-----------------------------------------
\subsection{Intégration des fractions rationnelles}



%---------------------------------------------------------------
\textbf{Trois situations de base}

On souhaite intégrer $f(x)=\frac{\alpha x + \beta}{a x^2+b x+c}$.

\textbf{Premier cas.} Le dénominateur $a x^2+b x+c$ possède deux racines réelles distinctes $x_1,x_ 2\in \Rr$.


Écrire 
$f(x)
= \frac{\alpha x + \beta}{a(x - x_1)(x - x_2)} 
= \frac{A}{x - x_1}+\frac{B}{x -x_2}$ avec $A,B \in \Rr$ à déterminer. 
Alors 
$$\int f(x)\;dx = A \ln|x - x_1|+B\ln|x -x_2|+c$$
sur chacun des intervalles
$]-\infty,x_1[$, $]x_1,x_2[$, $]x_2,+\infty[$ (si $x_1<x_2$).


\textbf{Deuxième cas.} Le dénominateur $a x^2+b x+c$ possède une racine double $x_0 \in \Rr$.


Alors 
$f(x)
= \frac{\alpha x + \beta}{a(x -x_0)^2}
= \frac{A}{(x - x_0)^2}+\frac{B}{x - x_0}$ avec $A,B \in \Rr$ à déterminer. 

Alors
$$\int f(x)\;dx = -\frac{A}{x - x_0} + B\ln|x - x_0|+c$$
sur chacun des intervalles
$]-\infty,x_0[$, $]x_0,+\infty[$.


\textbf{Troisième cas.}  Le dénominateur $u = a x^2+b x+c$ ne possède pas de racine réelle.

\begin{enumerate}
	\item Faire apparaître une fraction du type $\frac{u'}{u}$
	(que l'on sait intégrer en $\ln|u|$).
	
    \item Il reste une partie du type $\frac{1}{u}$ qui par un changement de variable se ramène à une expression $\frac{1}{v^2+1}$ dont une primitive est $\arctan v$.
\end{enumerate}


%---------------------------------------------------------------
\textbf{Intégration des éléments simples}

Une fraction rationnelle $\frac{P(x)}{Q(x)}$ s'écrit comme somme d'un polynôme $E(x) \in \Rr[x]$ (la partie entière)
et d'éléments simples d'une des formes suivantes :
$$\frac{\gamma}{(x - x_0)^k} \quad \text{ ou } \quad \frac{\alpha x+\beta}{(a x^2+b x+c)^k} \text{ avec } b^2-4ac < 0$$
% où $\alpha,\beta,\gamma,a,b,c \in \Rr$ et $k \in \Nn\setminus\{0\}$.

\begin{enumerate}
	\item On sait intégrer le polynôme $E(x)$.
	
	\item Intégration de l'élément simple $\frac{\gamma}{(x - x_0)^k}$.
	\begin{enumerate}
		\item Si $k=1$ alors $\int \frac{\gamma \; dx}{x - x_0} = \gamma \ln|x - x_0|+c_0$.
		\item Si $k\ge 2$ alors  $\int \frac{\gamma \; dx}{(x - x_0)^k} = \gamma \int (x - x_0)^{-k} \; dx
		= \frac{\gamma}{-k+1}(x - x_0)^{-k+1}+c_0$.
	\end{enumerate}
	
	\item Intégration de l'élément simple $\frac{\alpha x+\beta}{(a x^2+b x+c)^k}$.
	On écrit cette fraction sous la forme
	$$\frac{\alpha x+\beta}{(a x^2+b x+c)^k} = \gamma \frac{2a x+b}{(a x^2+b x+c)^k} + \delta \frac{1}{(a x^2+b x+c)^k}$$
	\begin{enumerate}
		\item Si $k=1$, calcul de $\int \frac{2a x+b}{a x^2+b x+c} \; dx = \int \frac{u'(x)}{u(x)} \; dx = \ln |u(x)| + c_0
		= \ln |a x^2+b x+c|+c_0$.
		
		\item Si $k\ge 2$, calcul de $\int \frac{2a x+b}{(a x^2+b x+c)^k} \; dx = \int \frac{u'(x)}{u(x)^k} \; dx = \frac{1}{-k+1}u(x)^{-k+1}+c_0
		= \frac{1}{-k+1}(a x^2+b x+c)^{-k+1}+c_0$.
		
		\item Si $k=1$, calcul de $\int \frac{1}{a x^2+b x+c}\; dx$. Par un changement de variable $u= px+q$ on se ramène à
		calculer une primitive du type
		$\int \frac{du}{u^2+1}=\arctan u + c_0$.
		
		\item Si $k\ge 2$, calcul de $\int \frac{1}{(a x^2+b x+c)^k} \; dx$. On effectue le changement de variable $u=px+q$
		pour se ramener au calcul de $I_k =\int \frac{du}{(u^2+1)^k}$.
		Une intégration par parties permet de passer de $I_k$ à $I_{k-1}$.
	
	\end{enumerate}
\end{enumerate}


%---------------------------------------------------------------
\textbf{Intégration des fonctions trigonométriques}


\textbf{Les règles de Bioche.}\index{regle@règle!de Bioche}
On note $\omega(x) = f(x)\;dx$.
On a alors $\omega(-x)= f(-x)\;d(-x)=-f(-x)\;dx$ et
$\omega (\pi-x)= f(\pi-x)\;d(\pi-x)=-f(\pi-x)\;dx$.



\begin{itemize}
	\item Si $\omega(-x)=\omega(x)$ alors on effectue le changement de variable $u=\cos x$.
	
	\item Si $\omega(\pi-x)=\omega(x)$ alors on effectue le changement de variable $u=\sin x$.
	
	\item Si $\omega(\pi + x)=\omega(x)$ alors on effectue le changement de variable $u=\tan x$.
\end{itemize}



\textbf{Le changement de variable $t=\tan \frac x2$.}

Les formules de la \og tangente de l'arc moitié \fg{} permettent d'exprimer sinus, cosinus et tangente
en fonction de $\tan \frac x2$.
\mybox{
	\begin{tabular}{c}
		Avec \quad  $\displaystyle t=\tan \frac{x}{2}$ \quad on a \\[2ex]
		$\displaystyle\cos x = \frac {1-t^2}{1+t^2} 
		\qquad  \sin x = \frac{2t}{1+t^2} 
		\qquad \tan x = \frac{2t}{1-t^2}$ \\[2ex]
		et \qquad  $\displaystyle dx=\dfrac{2\;dt}{1+t^2}.$
	\end{tabular}
}



\end{multicols}
\end{document}



\end{document}


