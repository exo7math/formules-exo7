\documentclass[10pt,class=article,crop=false]{standalone}
\usepackage{../exo7formules}


\begin{document}

%%%%%%%%%%%%%%%%%%%%%%%%%%%%%%%%%%%%%%%%%%
\section{Les nombres réels}

\begin{multicols}{2}
	

%-----------------------------------------
\subsection{L'ensemble des nombres rationnels $\Qq$}

L'ensemble des \defi{nombres rationnels}\index{nombre!rationnel} est
\(
\Qq = \left\{ \frac{p}{q}  \mid p\in \Zz, q\in \Nn^*\right\}.
\)

%Un \defi{nombre décimal}\index{nombre!décimal} est un nombre de la forme $\frac{a}{10^n}$,
%avec $a\in \Zz$ et $n\in \Nn$ ; ce sont des nombres rationnels.

\begin{proposition}
Un nombre est rationnel si et seulement s'il admet une écriture décimale périodique ou finie.
\end{proposition}


%---------------------------------------------------------------
\textbf{}

\begin{proposition}
	$\sqrt2$ n'est pas un nombre rationnel :
	\myboxinline{$\sqrt{2}\notin \Qq$}
\end{proposition}

La démonstration classique par l'absurde est à connaître !

On représente les nombres réels sur la \og droite numérique\fg{} :
\myfigure{0.7}{
	\tikzinput{fig_reels03}
}


$\sqrt{2}\simeq 1,4142\ldots$ \quad
$\pi\simeq 3,14159265\ldots$ \quad $e\simeq 2,718\ldots$


La droite numérique \og{}achevée\fg{} est : 
$\overline{\Rr}=\Rr\cup\{-\infty,\infty\}$




%-----------------------------------------
\subsection{Propriétés de $\Rr$}

\begin{proposition}[addition et multiplication]
	$(\Rr,+,\times)$ est un \evidence{corps commutatif}.
C'est-à-dire, pour $a,b,c \in \Rr$ on a :
\begin{center}
	\begin{tabular}{ll}
		$a+b=b+a$ & $a\times b=b\times a$ \\
		$0+a=a$ & $1\times a =a\text{ si }a\neq 0$ \\
		$a+b=0 \iff a=-b$ & $ab=1 \iff a=\frac{1}{b}$\\
		$(a+b)+c=a+(b+c)$ & $(a\times b)\times c=a\times (b\times c)$\\
		&\\
		$a\times(b+c)=a\times b+a\times c$ & \\
		$a\times b=0 \iff (a=0 \text{ ou } b=0)$ &
	\end{tabular}
\end{center}	
\end{proposition}

\begin{proposition}[Ordre]
La relation $\leq$ sur $\Rr$ est une relation d'ordre, et de plus, elle est totale.
Nous avons donc :
\begin{itemize}
	\item pour tout $x \in \Rr$, $x \le x$ (\defi{réflexive}),
	\item pour tout $x,y \in \Rr$, si $x \le y$ et $y\le x$ alors $x=y$ (\defi{antisymétrique}),
	\item pour tout $x,y,z \in \Rr$ si $x\le y$ et $y\le z$ alors $x\le z$ (\defi{transitive}),
	\item pour tout $x,y \in \Rr$, on a $x \le y $ ou $y \le x$ (\defi{totale}).
\end{itemize}
\end{proposition}


\begin{proposition}[Propriété d'Archimède]
\label{propr:archi}
   $\Rr$ est \defi{archimédien}\index{archimedien@archimédien}, c'est-à-dire :
   Pour tout réel $x$, il existe un entier naturel $n$ strictement plus grand que $x$.
\end{proposition}


\begin{proposition}
	\label{prop:part_ent}
	Soit $x\in \Rr$, il \evidence{existe} un \evidence{unique} entier relatif, la 
	\defi{partie entière}\index{partie entiere@partie entière} notée $E(x)$, tel que :
	\mybox{$E(x)\leq x <E(x)+1$}
\end{proposition}



		
		\myfigure{0.4}{
			\tikzinput{fig_reels04}
		}


%---------------------------------------------------------------
% \subsection{Valeur absolue}

Pour un nombre réel $x$, on définit la \defi{valeur absolue}\index{valeur absolue} de $x$ par :
\mybox{$\displaystyle |x|=
	\begin{cases}
		x & \text{ si } x\geq 0 \\
		-x & \text{ si } x<0
	\end{cases}
	$}


\myfigure{0.8}{
	\tikzinput{fig_reels05}
}

\begin{proposition}
	\sauteligne
	\begin{enumerate}
		\item $|x|\geq 0$ \quad ; \quad  $|-x|=|x|$ \quad  ;  \quad $|x|>0 \iff x\neq 0$
		\item $\sqrt{x^2}=|x|$
		\item $|xy|=|x||y|$
		\item \defi{Inégalité triangulaire}\index{inegalite@inégalité!triangulaire} \myboxinline{$|x+y|\leq |x|+|y|$}
		\item \defi{Seconde inégalité triangulaire} $\big||x|-|y|\big|\leq |x-y|$
	\end{enumerate}
\end{proposition}

Sur la droite numérique, $|x-y|$ représente la distance entre les
réels $x$ et $y$ ; en particulier $|x|$ représente la distance entre les réels $x$ et $0$.

\myfigure{0.7}{
	\tikzinput{fig_reels06}
}

\begin{center}
	\begin{tikzpicture}[>=latex]
		
	\end{tikzpicture}
\end{center}

De plus  $|x-a|<r \iff x \in {}]a-r,a+r[$.



%-----------------------------------------
\subsection{Densité de $\Qq$ dans $\Rr$}

\begin{definition}
	Soit $a$ un réel, $V\subset \Rr$ un sous-ensemble.
	On dit que $V$ est un \defi{voisinage}\index{voisinage} de $a$ s'il existe un intervalle
	ouvert $I$ tel que $a\in I$ et $I\subset V$.
\end{definition}

\begin{theoreme}
	\sauteligne
	\begin{enumerate}
		\item $\Qq$ est \defi{dense}\index{densite@densité} dans $\Rr$ :
		tout intervalle ouvert (non vide) de $\Rr$ contient une infinité de rationnels.
		\item $\Rr\backslash\Qq$ est dense dans $\Rr$ :
		tout intervalle ouvert (non vide) de $\Rr$ contient une infinité d'irrationnels.
	\end{enumerate}
\end{theoreme}


%-----------------------------------------
\subsection{Borne supérieure}


\textbf{Maximum, minimum}


\begin{definition}
	\label{def:max}
	Soit $A$ une partie non vide de $\Rr$. Un réel $\alpha$ est un
	\defi{plus grand élément}\index{plus grand element@plus grand élément} (ou \defi{maximum}) de $A$ si :
	\mycenterline{$\alpha \in A$ \qquad et  \qquad  $\forall x \in A \;\; x\leq \alpha$.}
	S'il existe, le plus grand élément est unique, on le note alors $\max A$.
	
	Le \defi{plus petit élément}\index{plus petit element@plus petit élément} de $A$, (ou \defi{minimum}) noté $\min A$, 
	s'il existe est le réel
	$\alpha$ tel que $\alpha \in A$ et $\forall x \in A \;\; x \ge \alpha$.
\end{definition}

Le plus grand élément ou le plus petit élément n'existent pas toujours.


\begin{definition}
	\label{def:majorant-minorant}
	Soit $A$ une partie non vide de $\Rr$. Un réel $M$ est un
	\defi{majorant}\index{majorant} de $A$ si $\forall x \in A \;\; x\leq M$.
	
	Un réel $m$ est un \defi{minorant}\index{minorant} de $A$ si $\forall x \in A \;\; x\geq m$.
\end{definition}



Si un majorant (resp. un minorant) de $A$ existe on dit que $A$ est \defi{majorée} (resp. \defi{minorée}).



\begin{definition}
	\label{def:sup-inf}
	Soit $A$ une partie non vide de $\Rr$ et $\alpha$ un réel.
	\begin{itemize}
		\item $\alpha$ est la \defi{borne supérieure}\index{borne superieure@borne supérieure} de $A$ si $\alpha$ est un majorant de $A$ et
		si c'est le plus petit des majorants. S'il existe on le note $\sup A$.
		\item $\alpha$ est la \defi{borne inférieure}\index{borne inferieure@borne inférieure} de $A$ si $\alpha$ est un minorant de
		$A$ et si c'est le plus grand des minorants. S'il existe on le note $\inf A$.
	\end{itemize}
\end{definition}




\begin{theoreme}[$\Rr4$]
	Toute partie de $\Rr$ non vide et majorée admet une borne supérieure.
\end{theoreme}

De la même façon : Toute partie de $\Rr$ non vide et minorée admet une borne inférieure.

\begin{proposition}[Caractérisation de la borne supérieure]
	Soit $A$ une partie non vide et majorée de $\Rr$. La borne supérieure de $A$ est l'unique réel $\sup A$ tel que
	\begin{enumerate}
		\item[(i)] si $x\in A$, alors $x\leq \sup A$,
		\item[(ii)] pour tout $y<\sup A$, il existe $x\in A$ tel que $y<x$.
	\end{enumerate}
\end{proposition}





Caractérisation très utile de la borne supérieure.
\begin{proposition}
	Soit $A$ une partie non vide et majorée de $\Rr$. La borne supérieure de $A$ est l'unique réel $\sup A$ tel que
	\begin{enumerate}
		\item[(i)] $\sup A$ est un majorant de $A$,
		\item[(ii)] il existe une suite $(x_n)_{n\in\Nn}$ d'éléments de $A$ qui converge vers $\sup A$.
	\end{enumerate}
\end{proposition}


\end{multicols}

\end{document}	