\documentclass[10pt,class=article,crop=false]{standalone}
\usepackage{../exo7formules}


\begin{document}

%%%%%%%%%%%%%%%%%%%%%%%%%%%%%%%%%%%%%%%%%%
\section{Quelques mathématicien·ne·s}


\begin{multicols}{2}

%-----------------------------
\subsection{Antiquité}

\begin{biographie}{Thalès}{v. 625 av. J.-C. -- v. 545 av. J.-C}
Savant grec, né à Milet (aujourd'hui en Turquie).  Il cherche à expliquer le monde en se basant sur les sciences et non la mythologie. 
Les connaissances grecques se basent probablement sur la science égyptienne et les mathématiques babyloniennes (Mésopotamie).
Géomètre, il travaille à la fois sur les figures et en particulier les triangles vus comme des objets théoriques, mais aussi sur les applications pratiques de la géométrie. On lui attribue la mesure de la hauteur des pyramides d'Égypte grâce à leurs ombres. Son nom reste associé au fameux théorème de Thalès.
\end{biographie}


\begin{biographie}{Pythagore}{v. 580 av. J.-C. -- v. 495 av. J.-C}
Savant et philosophe grec. Son enseignement est reconnu avec de nombreux disciples et influencera Platon et Aristote. Pour l'école pythagoricienne les mathématiques sont la base pour modéliser le monde. Pythagore et son école s'intéressent à la fois à la géométrie et aux nombres, c'est par exemple le cas du célèbre théorème qui porte le nom de Pythagore et relie une figure géométrique à des nombres. Il aurait dit \og{}Tout l'univers repose sur l'ensemble des entiers naturels\fg{} car on pensait à l'époque que toutes les longueurs pouvait être calculées par des nombres rationnels.
\end{biographie}


\begin{biographie}{Euclide}{vers 300 av. J.-C.}
Mathématicien grec qui dans les 13 livres des \emph{Éléments} donne la première présentation systématique de la géométrie du plan et de l'espace, avec une approche déductive et rigoureuse des mathématiques : à partir d'axiomes et de définitions, on énonce des théorèmes qui font l'objet de démonstrations. Dans le livre X on trouve la démonstration de l'irrationalité de $\sqrt2$.
Son nom reste associé à la division euclidienne, l'algorithme d'Euclide, la géométrie euclidienne (droite, plan, conique,...).
\end{biographie}


\begin{biographie}{Archimède}{v. 287 av. J.-C. -- 212 av. J.-C.}
Savant grec de Syracuse (Sicile), il est fameux pour son \emph{eurêka} associé à la découverte de la poussée d'Archimède. Il est à l'origine de nombreuses inventions et d'améliorations : engrenages, vis sans fin, leviers, catapultes et autres machines de guerre.
Ses travaux mathématiques portent sur la géométrie, il obtient l'encadrement $\frac{223}{71} < \pi < \frac{22}{7}$ à l'aide de polygones encadrant un cercle.
Concentré à dessiner des figures sur le sol, il meurt lors d'une attaque romaine de Syracuse qu'il n'a pas vu venir, ayant lancé à un soldat romain \og{}Ne dérange pas mes cercles !\fg{}
\end{biographie}


\begin{biographie}{Hypatie d'Alexandrie}{v. 355 à 370 -- 415}
Astronome, mathématicienne et philosophe grecque d'Alexandrie (l'Égypte étant sous contrôle de l'empire romain). Comme les autres savants de l'époque, son travail principal consiste à commenter les travaux de ses prédécesseurs. Elle annote l'\emph{Almageste} de \textsc{Ptolémée} (v. 100 -- v. 160) qui est un livre d'astronomie où la Terre est au centre de l'univers. Elle y améliore l'algorithme de division des entiers. Elle commente aussi le traité \emph{Arithmétique} de \textsc{Diophante} et construit des outils astronomiques comme des astrolabes permettant aux marins de s'orienter grâce à la position des étoiles.
\end{biographie}


%-----------------------------
\subsection{Moyen-âge}

\begin{biographie}{Muhammad ibn Al-Khwarizmi}{v. 780 -- v. 850}
Savant perse né dans l'actuel Ouzbékistan et mort à Bagdad (actuel Irak).	Ses travaux portent sur les mathématiques, l'astronomie et la géographie.
Son livre \emph{Abrégé de calculs} est à l'origine du mot \emph{algèbre}, il y présente des méthodes  pour résoudre les équations de degré $1$ et $2$. Il utilise le système de numération décimale emprunté aux indiens. Il s'attache à présenter des méthodes systématiques pour résoudre les problèmes d'algèbre. 
Ses présentations se font sur des exemples numériques car on ne dispose pas encore des notation avec des lettres. Son nom a donné le mot \emph{algorithme}.
\end{biographie}


\begin{biographie}{Léonard de Pise - Fibonacci}{v. 1170 -- v. 1250}
Mathématicien italien, considéré comme l'un des plus grands mathématiciens du moyen-âge.
Dans son livre \emph{Liber abaci} (le livre des calculs), il introduit en Europe le système de numération décimale indo-arabe afin de remplacer l'usage des chiffres romains. Les deux innovations sont l'introduction d'une notation pour les dix chiffres et en particulier du zéro, et le fait que la valeur des nombres dépend de la position de ces chiffres (les unités à droite, puis les dizaines...).
Il modélise aussi l’évolution théorique d’une population de lapins dans 
la \emph{suite de Fibonacci} qui est définie par $F_0=0$, $F_1=1$ puis la formule de récurrence 
$F_{n+2} = F_{n+1} + F_n$ lorsque $n \ge 0$. Les premières valeurs sont $0,1,1,2,3,5,8,13,21,\ldots$
\end{biographie}


%-----------------------------
\subsection{Renaissance}

\begin{biographie}{John Napier}{1550--1617}
Mathématicien, astronome et théologien écossais (nom francisé en \textsc{Jean Neper}).
Il cherche des moyens de rendre plus efficace les calculs courants, en particulier pour l'astronomie et la navigation où il y a beaucoup de calculs avec de grands nombres.
Il calcule une table de logarithmes longue de 90 pages qui associe un produit à une somme, permettant ainsi d'obtenir le résultat d'une multiplication par une simple addition et la lecture de la table. 
Le logarithme permet à \textsc{Kepler} de découvrir expérimentalement sa troisième loi. Ces tables seront améliorées pour atteindre une précision de $14$ décimales et resteront en usage jusqu'à l'avènement des calculatrices. En langage moderne, la table correspond à l'identité $\ln(ab) = \ln(a) + \ln(b)$. L'identité $\ln(\sqrt{a}) = \frac12\ln(a)$ permet de calculer facilement des racines carrées. 
 Son nom reste attaché au \emph{logarithme népérien}, les logarithmes dans d'autres bases s'obtenant par $\log_b(x) = \frac{\ln(x)}{\ln(b)}$.
\end{biographie}


\begin{biographie}{René Descartes}{1596--1650}
Philosophe et mathématicien français.
Il introduit la notation avec des lettres : $a$, $b$, $c$\ldots{} pour les paramètres, $x$, $y$, $z$ pour les inconnues et aussi la notation d'exposant $x2$ (pour $x^2$), ce qui rend plus lisibles les textes et facilite les calculs. 
Il se fixe comme objectif de ramener les problèmes de géométrie à des équations algébriques grâce aux coordonnées $(x,y)$ dans ce qu'on appelle aujourd'hui un repère \emph{cartésien}.
La \emph{méthode de Descartes} consiste en quatre règles pour résoudre des problèmes de façon scientifique :
(i) S'assurer de bien comprendre la question en se l'appropriant.
(ii) Décomposer le problème en questions plus simples.
(iii) Ordonner ces questions logiquement et par ordre de difficulté puis les résoudre une à une.
(iv) Vérifier soigneusement l'ensemble du raisonnement.
\end{biographie}


\begin{biographie}{Pierre de Fermat}{v. 1605--1665}
Magistrat de la région toulousaine et esprit universel, il s’intéresse à plusieurs domaines scientifiques. En optique, il découvre le principe
qui affirme que la lumière se propage toujours suivant le trajet de durée minimale.  Sa correspondance avec \textsc{Pascal} jette les bases du calcul des probabilités. 
Son nom reste attaché à des résultats d'arithmétique : \og{}si $p$ est premier $a^p \equiv a \pmod{p}$\fg{} (le petit théorème de Fermat). Il énonce le \og{}grand théorème de Fermat\fg{} : \og{}Si $n>2$, l'équation $x^n + y^n = z^n$ n'a pas de solution avec $x$, $y$, $z$ entiers strictement positifs.\fg{} Dans une note, il affirme en avoir découvert une démonstration mais qu'il n'a pas assez de place pour l'écrire. Cet énoncé restera pendant plus de 300 ans une des plus grandes conjectures de l'histoire mathématique.
\end{biographie}


\begin{biographie}{Blaise Pascal}{1623--1662}
Savant français connu pour ses travaux en mathématiques, physique, théologie et philosophie.
Au niveau mathématique, il participe à la création de nouveaux domaines : les probabilités et la géométrie projective. Pour aider son père dans son travail de comptabilité, il invente la première machine à calculer, la \emph{Pascaline}, capable d'additionner et soustraire. Il est le premier à expliciter un raisonnement par récurrence.
Son nom reste attaché au triangle de Pascal (connu des mathématiciens perses et chinois plusieurs siècles auparavant). Par ses connaissances en combinatoire, il donne un sens mathématique au hasard : c'est l'invention des probabilités.
\end{biographie}


%-----------------------------
\subsection{Le calcul différentiel}

\begin{biographie}{Isaac Newton}{1643--1727}
Mathématicien, physicien, astronome et théologien.
Dans les \emph{Principia Mathematica} (1687), dont la traduction française est due à \textsc{Émilie du Châtelet}, il définit le concept de force (magnitude et direction). La somme de deux forces peut alors être représentée par la diagonale d’un parallélogramme. Reconnu comme le fondateur de la mécanique classique avec sa théorie de la gravitation universelle, en optique il développe une théorie de la couleur.
En mathématiques, il est considéré comme l’un des inventeurs du calcul infinitésimal. Il passe aussi une grande partie de son temps à étudier l'alchimie afin de transformer le plomb en or.
Son nom reste attaché à de nombreux théorèmes : la méthode de Newton (pour approcher le zéro d'une fonction), la formule du binôme de Newton $(a+b)^r$ (avec un exposant $r$ pouvant être un nombre rationnel).
\end{biographie}


\begin{biographie}{Gottfried Wilhelm Leibniz}{1646--1716}
Philosophe, savant aux multiples facettes et conseiller politique allemand. 
Durant sa vie il échange plus de $20\,000$ lettres avec des savants de toute l'Europe.
Avec \textsc{Newton}, il joue un rôle majeur dans l’invention du calcul infinitésimal (limite, dérivée,\ldots). Il introduit les notations $\frac{\dd f}{\dd x}$ pour la dérivée et $\int f$ pour l'intégrale. Il étudie aussi en détail l'écriture binaire des entiers.
Il s'intéresse aux fonctions classiques (logarithme, exponentielle,\ldots) et leur développement en série entière ; il est le premier à étudier la fonction $x \mapsto a^x$.
\end{biographie}


\begin{biographie}{Leonhard Euler}{1707--1783}
Savant suisse prolifique, il aborde l’ensemble de branches des mathématiques (analyse, nombres complexes, graphes, arithmétique\ldots). Par son travail, il contribue à la compréhension et à la diffusion du calcul infinitésimal au long du \textsc{xviii}\ieme{} siècle. Il introduit la notation $f(x)$ pour une fonction, il popularise l'usage de la notation $\pi$ (pour $3.14\ldots$),
le symbole $\Sigma$ pour les sommes et $\ii$ pour le nombre complexe avec $\ii^2=-1$. Devenu aveugle, il continue quand même de travailler.
La constante d'Euler $\gamma$ est la limite entre la série harmonique et le logarithme : $\gamma = \lim_{n\to+\infty} \sum_{k=1}^{n}{\frac1k} - \ln(n) = 0.577\ldots$
\end{biographie}


\begin{biographie}{Augustin-Louis Cauchy}{1789--1857}
\'Etudiant brillant et scientifique aux travaux abondants, il a vécu durant la période troublée d'après la Révolution française.
Dans le cadre de son \emph{Cours d’Analyse de l’École Polytechnique}, il s’attache à donner une définition rigoureuse d’une suite et de sa limite. Il est considéré comme ayant posé les bases de l’analyse moderne.
Son nom reste attaché à de nombreuses notions d'analyse : suite de Cauchy, règle de Cauchy pour les séries, théorème de Cauchy-Lipschitz pour les équations différentielles,  la formule intégrale de Cauchy en analyse complexe.
\end{biographie}


%-----------------------------
\subsection{La naissance de l'algèbre moderne}

\begin{biographie}{Évariste Galois}{1811--1832}
Mathématicien français mort très jeune dans un duel pour une obscure histoire romantico-politique. Une équation polynomiale de degré $2$ se résout \og{}par radicaux\fg{}, c'est-à-dire que les racines s'expriment en fonction des coefficients à l'aide d'expression faisant intervenir des racines carrés.  Il existe de telles formules pour les polynômes de degré 3  et degré 4. \textsc{Niels Abel} (1802--1829) prouve que de telles formules n'existent pas pour un polynôme quelconque de degré $5$. \textsc{Galois} traite le cas d'un degré quelconque. Il détermine exactement quels sont les polynômes résolubles par radicaux en associant à chaque polynôme un groupe de permutation de ses racines. Ses travaux ne sont ni compris, ni reconnus de son vivant. Pourtant cette nouvelle notion de groupe sera à la base des structures algébriques et des mathématiques modernes. 
\end{biographie}


\begin{biographie}{Sophie Germain}{1776--1831}
Née à Paris, elle se forme seule aux mathématiques à l'aide de livres.
Elle converse par lettres avec \textsc{Lagrange} et \textsc{Gauss} sous le pseudonyme de M.~Leblanc avant de révéler qu'elle est une femme.
Elle travaille sur les équations différentielles et l'arithmétique et en particulier sur la recherche d'une preuve du grand théorème de Fermat. 
Les nombres premiers de Germain sont des entiers $p$ tels que $p$ et $2p+1$ soient tous les deux des nombres premiers. La question de savoir s'il existe une infinité de tels nombres est toujours ouverte.
\end{biographie}


\begin{biographie}{Carl Friedrich Gauss}{1777--1855}
Scientifique allemand, surnommé le \og{}prince des mathématiques\fg{}.
Enfant précoce, on raconte qu'à 7 ans, répondant à un professeur, il avait été capable de calculer en quelques secondes la somme $1+2+3+\cdots+100$. 
Il s'intéresse à l'astronomie et au magnétisme, mais ses principaux travaux concernent l'algèbre, la géométrie, l'arithmétique et les probabilités (méthode des moindres carrés). La \emph{distribution de Gauss} correspond à une répartition sous la forme d'une \og{}courbe en cloche\fg{} donnée par l'équation $x \mapsto \exp(-x^2)$ (l'aire sous la courbe valant $\sqrt\pi$).
Son nom est associé au théorème de D'Alembert-Gauss, dit théorème fondamental de l'algèbre : \og{}Tout polynôme $P$ à coefficients complexes de degré $n>0$, admet exactement $n$ racines (comptées avec multiplicités).\fg{}
\end{biographie}	


\begin{biographie}{Ada Lovelace}{1815--1852}
Fille du poète anglais Lord Byron. En 1837, \textsc{Charles Babbage} élabore les plans d'une machine mécanique qui serait capable de faire des calculs assez généraux dont les instructions sont données à l'aide de cartes perforées (comme cela existait pour les métiers à tisser et les orgues de Barbarie). Lovelace se fascine pour cette machine et est la première personne à rendre public un \emph{algorithme}, c'est-à-dire une suite d'instructions permettant de résoudre un problème, ici le calcul des nombres de Bernoulli avec une boucle \emph{tant que}. De plus elle est la première à avoir compris qu'une telle machine à calculer, ancêtre d'un ordinateur, pouvait avoir des applications très générales au-delà du calcul numérique. 
\end{biographie}


%-----------------------------
\subsection{Les temps modernes}


\begin{biographie}{Henri Poincaré}{1854--1912}
Né à Nancy, il aborde dans ses travaux toutes les branches des mathématiques : géométrie, équations différentielles, mécanique céleste (problème des trois corps), il étudie aussi la théorie de la relativité.
C'est le fondateur de la \emph{topologie algébrique}, il s'agit d'étudier des ensembles (des \emph{variétés}) à bijection continue près (\emph{homéomorphisme}). Pour de tels ensembles, il introduit un groupe, \emph{le groupe fondamental} $\pi_1$. En 1904 il formule la conjecture suivante : \og{}Une variété compacte de dimension 3 dont le groupe fondamental est trivial est homéomorphe à la sphère de dimension 3.\fg{} Il faudra attendre \textsc{Pereleman} en 2003 pour une démonstration de la conjecture de Poincaré.
\end{biographie}


\begin{biographie}{David Hilbert}{1862--1943}
Grand mathématicien allemand qui a marqué l'entrée des mathématiques dans le \textsc{xx}\ieme{} siècle.  Il contribue à tous les domaines : analyse, théorie des nombres, logique, et aussi à la physique.
Il propose dans \emph{Les fondements de la géométrie},  une nouvelle axiomatisation qui place au premier plan les relations entre les objets géométriques (appartenance, ordre, parallélisme, etc.) et non leur nature.
En 1900 il énonce 23 problèmes fondamentaux à résoudre pour le siècle qui s'annonce. Par exemple, le 7\ieme{} problème, qui sera démontré en 1934, implique que $\sqrt{2}^{\sqrt{2}}$ est un nombre \emph{transcendant} (comme aussi $\pi$ et $e$), c'est-à-dire qu'il n'est la racine d'aucun polynôme à coefficients entiers, en particulier $\sqrt{2}^{\sqrt{2}}$ est irrationnel.
\end{biographie}


\begin{biographie}{Emmy Noether}{1882--1935}
Mathématicienne allemande, elle est l'une des plus grandes algébristes de son époque à travers ses contributions à la théorie des anneaux et des corps.
En tant que femme elle a du mal à obtenir un poste de professeur, il lui faudra l'aide de \textsc{David Hilbert} et \textsc{Félix Klein} pour y parvenir. En 1933, comme d'autres savants juifs, elle est expulsée de son université et s'exile aux États-Unis. Les \emph{anneaux noethériens} sont nommés en son honneur.
\end{biographie}	


\begin{biographie}{Srinivasa Ramanujan}{1887--1920}
Né en Inde, d'une santé fragile, passionné dès l'enfance par les mathématiques, il abandonne très vite ses études. À l'âge de 16 ans, il découvre un livre qui compile des milliers de formules sans preuve. N'ayant pas de contact avec d'autres mathématiciens, il suit ce modèle et va remplir des carnets contenant près de 4000 formules sans aucunes explications. Après avoir pris contact avec des mathématiciens britanniques en 1913, son travail est enfin reconnu. Ses formules concernent l'arithmétique, certaines étaient fausses, certaines déjà connues, mais la plupart étaient inédites. Par exemple que $e^{\pi\sqrt{163}}$ est très proche d'un entier (à $10^{-12}$ près), 
ou bien $\pi \simeq \frac{9801\sqrt{2}}{4412}$ ($6$ décimales exactes de $\pi$) ce qui correspond au premier terme de la formule exacte :
$\frac{1}{\pi}  = \frac{2\sqrt{2}}{9801} 
\sum_{n=0}^{+\infty} \frac{(4n)! (1103 + 26390n)}{(n!)^4 396^{4n}}$.
\end{biographie}


\begin{biographie}{Katherine Johnson}{1918--2020}
Mathématicienne américaine, elle travaille à la NASA dans une équipe de \og{}calculatrices humaines\fg{} qui analyse et calcule les vols d'avions et les trajectoires des missions spatiales. En effet, avant l'avènement des calculatrices et des ordinateurs, des  équipes, souvent composées de femmes, étaient chargées de faire les calculs à la main. 
Pour le premier vol en orbite autour de la Terre en 1962, les trajectoires sont calculées pour la première fois à l'aide d'un ordinateur, mais l'astronaute John Glenn demande que les calculs soient vérifiés avant le vol personnellement par Johnson.
Les difficultés liées à son statut de femme noire dans une Amérique ségrégationniste sont racontées dans le film \emph{Les figures de l'ombre}. L'unité \emph{kilogirl}, inventée à l'époque, signifiait l'équivalent de 1000 heures de calculs à la main.
\end{biographie}


%-----------------------------
\subsection{De nos jours}

\begin{biographie}{Andrew Wiles}{né en 1953}
Mathématicien britannique devenu célèbre par sa preuve du grand théorème de Fermat : \og{}L'équation $x^n+y^n = z^n$ n'admet pas de solutions $x,y,z$ entières si $n\ge3$\fg{} (pour $n=2$ de telles solutions existent, par exemple $2^2+3^2=5^2$). Cet énoncé apparaissait dans les travaux de \textsc{Fermat} vers 1637 mais aucune démonstration n'était connue. 
Durant des années, des centaines de mathématiciens professionnels ou amateurs proposent des preuves, qui se révèlent toutes fausses !
Wiles travaille en secret plusieurs années avant de présenter sa preuve de plus de cent pages en 1994. Sa persévérance se retrouve dans une de ses citations :
\og{}Ce n'est pas parce que nous ne trouvons pas de solution qu'il n'y en a pas.\fg{}
\end{biographie}


\begin{biographie}{Grigori Pereleman}{né en 1966}
Brillant étudiant à Leningrad (aujourd'hui Saint Petersbourg, Russie), ses recherches portent sur la topologie et la géométrie.
En 2003 il prouve la conjecture de Poincaré que l'on peut formuler ainsi : \og{}Tout ensemble de dimension 3 qui ressemble à une sphère est en fait une sphère\fg{}, problème majeur resté insoluble depuis près de cent ans.
Considéré comme l'un des plus grands mathématiciens actuels, il se retire cependant de la vie mathématique : il n'a jamais publié officiellement ses travaux, il a refusé un million de dollars pour avoir résolu l'un des \emph{sept problèmes du millinéaire} et n'est pas venu chercher sa médaille Fields attribuée en 2006 : \og{}Tout le monde comprend que si la preuve est correcte, aucune autre reconnaissance n'est nécessaire.\fg{}
\end{biographie}


\begin{biographie}{Terence Tao}{né en 1975}
Né en Australie d'une famille originaire de Hong-Kong, il fait ses études aux États-Unis.
Génie très précoce, il discute dès l'âge de 10 avec des mathématiciens comme \textsc{Paul Erdős}.
Mathématicien prolixe, il travaille tantôt seul, tantôt avec de nombreux collaborateurs sur des problèmes ardus (comme les équations de Navier-Stokes en mécanique des fluides) mais aussi sur des questions compréhensibles par tous. Il reçoit la médaille Fields en 2006.
La conjecture de Goldbach affirme que \og{}Tout entier pair $>3$ est la somme de deux nombres premiers\fg{}
(toujours non résolue). Tao a  prouvé une forme plus faible \og{}Tout entier impair est la somme de cinq nombres premiers (ou moins)\fg{}.
Il a aussi démontré que la conjecture de Syracuse est \og{}presque vraie\fg{} (en un certain sens probabiliste) :
\og{}Partant d'un entier $n>0$, si $n$ est pair on le divise par $2$, sinon on le remplace par $3n+1$. 
Est-ce qu'en itérant ce processus on atteint toujours la valeur $1$, quel que soit l'entier initial $n$ ?\fg{}.
\end{biographie}


\begin{biographie}{Maryam Mirzakhani}{1977--2017}
Mathématicienne iranienne.
Ses contributions portent sur le comptage des géodésiques sur des surfaces.
Une \emph{géodésique} est une courbe la plus courte reliant deux points. Dans le plan ou l'espace, les géodésiques sont des morceaux de droites ; c'est différent pour d'autres surfaces, par exemple le chemin le plus court entre deux points d'une sphère est un arc d'un \og{}grand cercle\fg{}, les géodésiques sur un tore (une surface comme une chambre à air ou un \emph{doughnut}) sont plus difficiles à déterminer.
C'est la première femme à avoir reçu la médaille Fields (2014). 
% Elle meurt d'un cancer à l'âge de 40 ans.
Elle a dit : \og{}Il faut dépenser de l'énergie et faire des efforts pour voir la beauté des mathématiques.\fg{}
\end{biographie}


\end{multicols}

\end{document}	
