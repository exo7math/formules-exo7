
%%%%%%%%%%%%%%%%%%%%%%%%%%%%%%%%%%%%%%%%%%
\section*{Pour aller plus loin}

\addcontentsline{toc}{section}{Pour aller plus loin}

\begin{multicols}{2}

%----------------------------------
\subsection*{Exo7}

\begin{itemize}
	\item Site Exo7 : cours, exercices corrigés : 
	\href{http://exo7.emath.fr/}
    {exo7.emath.fr}
    
    \item Vidéos Exo7 :
    \href{https://www.youtube.com/Exo7Math}
    {www.youtube.com/Exo7Math}
    
    \item Livre \og{}Algèbre\fg{} :    
    \href{http://exo7.emath.fr/cours/livre-algebre-1.pdf}{fichier pdf},    
    \href{https://www.amazon.fr/dp/1517683637}{Amazon.fr} (livre papier noir et blanc).

    \item Livre \og{}Analyse\fg{} :
	\href{http://exo7.emath.fr/cours/livre-analyse-1.pdf}{fichier pdf},
	\href{https://www.amazon.fr/dp/1522821643}{Amazon.fr} (livre papier noir et blanc).
  
\end{itemize}


%----------------------------------
\subsection*{Autres ressources}


\begin{itemize}
		
	\item \og{}Bibm@th\fg{} : \href{https://www.bibmath.net/}{www.bibmath.net}
	
	Cours et exercices niveau L1/Math Sup, L2/Math Spe, Capes.
	
	\item \og{}Les-Mathématiques.net\fg{}
	\href{https://les-mathematiques.net/}{les-mathematiques.net}
	
	Cours et exercices niveau L1/Math Sup, L2/Math Spe, Capes.
	Forum de discussions.
	
	\item \og{}Maths-France\fg{} de J.-L. Rouget :
	\href{https://maths-france.fr/}{maths-france.fr}.
	
	Cours et exercices niveau Terminale, L1/Math Sup, L2/Math Spe, Capes.
	
	\item Page de Michel Quercia : 
	\href{http://michel.quercia.free.fr/}{michel.quercia.free.fr}
	
	Cours et exercices niveau L1/Math Sup, L2/Math Spe.
	
	\item Site de David Delaunay :
	\href{http://ddmaths.free.fr/index.html}{ddmaths.free.fr}
	
	Exercices corrigés.
	

\end{itemize}
	

%----------------------------------
\subsection*{Vulgarisation}

\begin{itemize}
	
	\item \og{}Images des mathématiques\fg{} :
	\href{https://images.math.cnrs.fr/}{images.math.cnrs.fr}
	
	Articles nombreux et variés sur les mathématiques d'aujourd'hui. 

    \item \href{https://www.youtube.com/channel/UC4PasDd25MXqlXBogBw9CAg}{Chaîne Youtube Micmaths} de Mickael Launay.
    
    Vidéos passionnantes de vulgarisation.
    
    \item \href{https://www.youtube.com/channel/UCoxcjq-8xIDTYp3uz647V5A}
    {Chaîne Youtube Numberphile}
    
    Vidéos de vulgarisation (en anglais).
\end{itemize}
    
    
%----------------------------------
\subsection*{Auteurs}

Cet ouvrage de formules a été réalisé par Arnaud Bodin.

Ces formules sont principalement tirées des livres \og{}Analyse\fg{} et \og{}Algèbre\fg{} d'Exo7 issus d'un large travail collectif.
Les contributeur des livres sont :
{
	\setlength{\columnseprule}{0pt}
	\center
\begin{multicols}{2}	
			Arnaud Bodin\\
			Eva Bayer-Fluckiger\\
			Niels Borne\\
			Marc Bourdon\\
			Philippe Chabloz\\
			Sophie Chemla\\			
			Guoting Chen

			
			Gilles Costantini\\
			Laura Desideri\\
			Abdellah Hanani\\
			Jean-Louis Rouget\\
			Pascal Romon\\
			Lara Thomas\\
\end{multicols}			
}


Relectures (passées ou présentes) : Stéphanie Bodin, Vianney Combet, Barnabé Croizat, Christine Sacré.
Merci à Guillaume Jouve, Kroum Tzanev pour leurs conseils.

\end{multicols}

\vspace*{\fill}

\bigskip 

\begin{center}
\LogoExoSept{3}
\end{center}



\begin{center}
Ce livre est diffusé sous la licence \emph{Creative Commons -- BY-NC-SA -- 4.0 FR}.
\end{center}


%\pagenumbering{gobble} % remove page numbering 
%\printindex
%\pagenumbering{arabic}

