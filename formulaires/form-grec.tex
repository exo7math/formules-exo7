\documentclass[10pt,class=article,crop=false]{standalone}
\usepackage{../exo7formules}


\begin{document}

%%%%%%%%%%%%%%%%%%%%%%%%%%%%%%%%%%%%%%%%%%
\section{Alphabet grec}


\begin{multicols}{2}

	\large
\begin{center}
	%\noindent
	\setlength{\arrayrulewidth}{0.05mm}
	%\begin{tabular}{|l|l|l|} \hline
	\begin{tabular}[t]{|ll|l@{\vrule depth 1.2ex height 3ex width 0mm \ }|}
		\hline
		$\alpha$      &               & alpha   \\ \hline
		$\beta$       &               & beta    \\ \hline
		$\gamma$      & $\Gamma$      & gamma   \\ \hline
		$\delta$      & $\Delta$      & delta   \\ \hline
		$\varepsilon$ &               & epsilon \\ \hline
		$\zeta$       &               & zeta    \\ \hline
		$\eta$        &               & eta     \\ \hline
		$\theta$      & $\Theta$      & theta   \\ \hline
		$\iota$       &               & iota    \\ \hline
		$\kappa$      &               & kappa   \\ \hline
		$\lambda$     & $\Lambda$     & lambda  \\ \hline
		$\mu$         &               & mu      \\ \hline
	\end{tabular}
\end{center}

\begin{center}	
	\begin{tabular}[t]{|ll|l@{\vrule depth 1.2ex height 3ex width 0mm \ }|}
		\hline
		$\nu$         &               & nu      \\ \hline
		$\xi$         &               & xi      \\ \hline
		$o$           &               & omicron \\ \hline
		$\pi$         & $\Pi$         & pi      \\ \hline
		$\rho$        &               & rho     \\ \hline
		$\sigma$      & $\Sigma$      & sigma   \\ \hline
		$\tau$        &               & tau     \\ \hline
		$\upsilon$    &               & upsilon \\ \hline
		$\phi,\varphi$& $\Phi$        & phi     \\ \hline
		$\chi$        &               & chi     \\ \hline
		$\psi$        & $\Psi$        & psi     \\ \hline
		$\omega$      & $\Omega$      & omega   \\ \hline
	\end{tabular}
\end{center}



\end{multicols}


On rencontre aussi ``nabla'' $\nabla$, l'opérateur de dérivée
partielle $\partial$ (dit ``d rond''), et aussi la première
lettre de l'alphabet hébreu ``aleph'' $\aleph$.

\end{document}	