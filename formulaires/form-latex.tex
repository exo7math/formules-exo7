\documentclass[10pt,class=article,crop=false]{standalone}
\usepackage{../exo7formules}


\begin{document}

%%%%%%%%%%%%%%%%%%%%%%%%%%%%%%%%%%%%%%%%%%
\section{Écrire des mathématiques: LaTeX}

\begin{multicols}{2}

%-----------------------------------------
%\subsection{??}

Pour écrire des mathématiques, il existe un langage pratique et universel,
le langage LaTeX (prononcé [latek]). Il est utile pour rédiger des textes contenant des formules,
mais aussi accepté sur certains blogs et vous permet d'écrire des maths
dans un courriel ou un texto.

Une formule s'écrit entre deux dollars \verb?$\pi^2$? qui donne $\pi^2$ ou entre double dollars
si l'on veut la centrer sur une nouvelle ligne ; \verb?$$\lim u_n = +\infty$$? affichera:
$$\lim u_n = + \infty$$

Dans la suite on omettra les balises dollars.

%-----------------------------------------
\subsection{Premières commandes}

Les exposants s'obtiennent avec la commande \verb?^? et les indices avec \verb?_?:
$a^2$ s'écrit \verb?a^2? ; $u_n$ s'écrit \verb?u_n? ; $\alpha_i^2$ s'écrit \verb?\alpha_i^2?.
Les accolades \verb?{ }? permettent de grouper du texte: \verb?2^{10}? pour $2^{10}$ ;
\verb?a_{i,j}? pour $a_{i,j}$.


Il y a ensuite toute une liste de commandes (qui commencent par \verb?\?) dont voici les plus utiles:


\begin{tabular}{ccc@{\vrule depth 1.2ex height 4ex width 0mm \ }}
	
	\verb?\sqrt?  &
	$\sqrt{a}$ & \verb?\sqrt{a}? \\
	& $\sqrt{1+\sqrt{2}}$ & \verb?\sqrt{1+\sqrt{2}}?  \\
	&  $\sqrt[3]{x}$ & \verb?\sqrt[3]{x}?  \\
	
	\verb?\frac?  &
	$\dfrac{a}{b}$ & \verb?\frac{a}{b}? \\
	& $\dfrac{\pi^3}{12}$ & \verb?\frac{\pi^3}{12}?   \\
	& $\dfrac{1}{2+ \frac{3}{4}}$ & \verb?\frac{1}{2 + \frac{3}{4}}?  \\
	& $\gamma^{\frac{1}{n}}$ & \verb?\gamma^{\frac{1}{n}}? \\
	
	
	\verb?\lim?  &
	$\lim_{n \to + \infty} u_n = 0$ & \verb?\lim_{n \to +\infty} u_n = 0?  \\
	& $\lim_{x \to 0} f(x) < \epsilon$ & \verb?\lim_{x \to 0} f(x) < \epsilon?  \\
	
	\verb?\sum?  &
	$\displaystyle \sum_{i=1}^n \frac{1}{i}$ & \verb?\sum_{i=1}^n \frac{1}{i}? \\
	& $\displaystyle \sum_{i \ge 0} a_i$ & \verb?\sum_{i \ge 0} a_i?  \\
	
	\verb?\int?  &
	$\displaystyle \int_a^b \phi(t) dt$ & \verb?\int_a^b \phi(t) dt?  \\
	
\end{tabular}

%-----------------------------------------
\subsection{D'autres commandes}

Voici d'autres commandes, assez naturelles pour les anglophones.

\begin{center}
	\begin{tabular}{cc}
		$f : E \to F$ & \verb?f : E \to F?  \\
		$+\infty$ & \verb?+\infty? \\
		$a \le 0$ & \verb?a \le 0? \\
		$a > 0$ & \verb?a > 0?  \\
		$a \ge 1$ & \verb?a \ge 1? \\
		$\delta$ & \verb?\delta? \\
		$\Delta$ & \verb?\Delta? \\
	\end{tabular}\hspace{2cm}
	\begin{tabular}{cc}
		$a \in E$ & \verb?a \in E? \\
		$A \subset E$ & \verb?A \subset E? \\
		$P \implies Q$ & \verb?P \implies Q? \\
		$P \iff Q$ & \verb?P \iff Q? \\
		$\forall$ & \verb?\forall? \\
		$\exists$ & \verb?\exists? \\
		$\cup$ & \verb?\cup? \\
		$\cap$ & \verb?\cap?  \\
	\end{tabular}
\end{center}

%-----------------------------------------
\subsection{Pour allez plus loin}

Il est possible de créer ses propres commandes avec \verb?\newcommand?.
Par exemple avec l'instruction

\hfil \verb?\newcommand{\Rr}{\mathbb{R}}?

vous définissez une nouvelle commande \verb?\Rr? qui exécutera l'instruction
\verb?\mathbb{R}? et affichera donc $\Rr$.

Autre exemple, après avoir défini

\hfil \verb?\newcommand{\monintegrale}{\int_0^1 f(t) dt}?

la commande \verb?\monintegrale? affichera $\int_0^1 f(t) dt$.

\end{multicols}

\end{document}	