\documentclass[10pt,class=article,crop=false]{standalone}
\usepackage{../exo7formules}


\begin{document}

%%%%%%%%%%%%%%%%%%%%%%%%%%%%%%%%%%%%%%%%%%
\section{Longueurs, aires, volumes}

\begin{multicols}{2}
	

	
\begin{center}
Périmètre (longueur) d'un cercle de rayon $r$

\begin{minipage}{0.2\textwidth}	
\mybox{$L = 2\pi r$}
\end{minipage}
\begin{minipage}{0.25\textwidth}
\myfigure{0.45}{
	\tikzinput{fig_aire_01}
}
\end{minipage}
\end{center}

\bigskip

\begin{center}
Aire d'un rectangle de longueur $L$ et largeur $\ell$
	
\begin{minipage}{0.2\textwidth}	
\mybox{$A = L \times \ell$}
\end{minipage}
\begin{minipage}{0.25\textwidth}
	\myfigure{0.6}{
		\tikzinput{fig_aire_02}
	}
\end{minipage}
\end{center}

\bigskip

\begin{center}
Aire d'un disque de rayon $r$

	\begin{minipage}{0.2\textwidth}	
	\mybox{$A = \pi r^2$}	
	\end{minipage}
	\begin{minipage}{0.25\textwidth}
		\myfigure{0.45}{
			\tikzinput{fig_aire_03}
		}
	\end{minipage}
\end{center}

\bigskip
	
\begin{center}
Aire d'un triangle de base $b$ et de hauteur $h$
	
	\begin{minipage}{0.2\textwidth}	
		\mybox{$A = \frac{b \times h}{2}$}
	\end{minipage}
	\begin{minipage}{0.25\textwidth}
		\myfigure{0.7}{
			\tikzinput{fig_aire_04}
		}
	\end{minipage}
\end{center}

\bigskip 
 
\begin{center}
Aire d'un parallélogramme de base $b$ et de hauteur $h$	
	
	\begin{minipage}{0.2\textwidth}	
	\mybox{$A = b \times h$}
	\end{minipage}
	\begin{minipage}{0.25\textwidth}
		\myfigure{0.6}{
			\tikzinput{fig_aire_05}
		}
	\end{minipage}
\end{center}

\bigskip	

\begin{center}
	Aire d'un trapèze de bases $B$ et $b$, et de hauteur $h$
	
	\begin{minipage}{0.2\textwidth}	
	\mybox{$A = \frac{(B+b) \times h}{2}$}	
	\end{minipage}
	\begin{minipage}{0.25\textwidth}
		\myfigure{0.6}{
			\tikzinput{fig_aire_06}
		}
	\end{minipage}
\end{center}

\bigskip


\begin{center}
Aire d'une sphère de rayon $r$

	\begin{minipage}{0.2\textwidth}	
	\mybox{$A = 4\pi r^2$}	
	\end{minipage}
	\begin{minipage}{0.25\textwidth}
		\myfigure{0.8}{
			\tikzinput{fig_aire_07}
		}
	\end{minipage}
\end{center}

\bigskip

\begin{center}
Volume d'un cylindre (ou d'un prisme) de base d'aire $A$ et de hauteur $h$	
	
	\begin{minipage}{0.2\textwidth}	
	\mybox{$V = A \times h$}	
	\end{minipage}
	\begin{minipage}{0.25\textwidth}
		\myfigure{0.4}{
			\tikzinput{fig_aire_08}
		}
	\end{minipage}
\end{center}

\bigskip

\begin{center}
Volume d'un cône (ou d'une pyramide) de base d'aire $A$ et de hauteur $h$	
	
	\begin{minipage}{0.2\textwidth}	
	\mybox{$V = \frac13 A \times h$}  	
	\end{minipage}
	\begin{minipage}{0.25\textwidth}
		\myfigure{0.35}{
			\tikzinput{fig_aire_09}
		}
	\end{minipage}
\end{center}

\bigskip

\begin{center}
Volume d'une boule de rayon $r$ 	
	
	\begin{minipage}{0.2\textwidth}	
	\mybox{$A = \frac{4}{3} \pi r^3$} 	
	\end{minipage}
	\begin{minipage}{0.25\textwidth}
		\myfigure{0.8}{
			\tikzinput{fig_aire_07}
		}
	\end{minipage}
\end{center}





\end{multicols}


\end{document}	