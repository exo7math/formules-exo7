\documentclass[10pt,class=article,crop=false]{standalone}
\usepackage{../exo7formules}


\begin{document}

%%%%%%%%%%%%%%%%%%%%%%%%%%%%%%%%%%%%%%%%%%
\section{Trigonométrie au collège}

\begin{multicols}{2}
	

$$
\sin \alpha = \frac{\text{longueur du côté opposé}}{\text{longueur de l'hypoténuse}}
$$

$$ 
\cos \alpha = \frac{\text{longueur du côté adjacent}}{\text{longueur de l'hypoténuse}}
$$

$$
\tan \alpha = \frac{\text{longueur du côté opposé}}{\text{longueur du côté adjacent}}
$$

\myfigure{1.2}{
	\tikzinput{fig_trigo_triangle}
}

Règle \og{}SOH CAH TOA\fg{} : 
$$
\sin = \frac{o}{h} \qquad 
\cos = \frac{a}{h} \qquad 
\tan = \frac{o}{a}
$$

\og{}SOH CAH TOA\fg{} est un moyen mnémotechnique de se souvenir des formules à l'aide des initiales
\og{}SOH\fg{} : le Sinus, c'est le côté Opposé sur l'Hypoténuse.
\og{}CAH\fg{} : le Cosinus, c'est le côté Adjacent sur l'Hypoténuse.
\og{}TOA\fg{} : la Tangente, c'est le côté Opposé sur le côté Adjacent.

$$\tan \alpha = \frac{\sin \alpha}{\cos \alpha}$$

$$\sin^2 \alpha + \cos^2 \alpha = 1$$

\end{multicols}
	

\end{document}	