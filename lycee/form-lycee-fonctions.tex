\documentclass[10pt,class=article,crop=false]{standalone}
\usepackage{../exo7formules}


\begin{document}

%%%%%%%%%%%%%%%%%%%%%%%%%%%%%%%%%%%%%%%%%%
\section{Fonctions}


\begin{multicols}{2}

%-----------------------------------------
\subsection{Fonction polynomiale}


Soit $P(x) = ax^2+bx+c$ avec $a,b,c \in \Rr$, $a\neq0$.
Soit \myboxinline{$\Delta = b^2-4ac$} le \defi{discriminant}.

\textbf{Racines}
\begin{itemize}
	\item Si $\Delta > 0$, l'équation $P(x)=0$ a deux solutions réelles :
	\mybox{$\displaystyle 
		x_1 = \frac{-b - \sqrt{\Delta}}{2a} 
		\qquad \text{ et } \qquad
	x_2 = \frac{-b + \sqrt{\Delta}}{2a} $}	
		
	\item Si $\Delta = 0$, l'équation $P(x)=0$ a une solution double :
$$x_0 = \frac{-b}{2a}$$	

	\item Si $\Delta < 0$, l'équation $P(x)=0$ n'a pas de solution réelle.
\end{itemize}

\textbf{Factorisation}
\begin{itemize}
	\item Si $\Delta > 0$, $P(x)= a(x-x_1)(x-x_2)$.
	\item Si $\Delta = 0$, $P(x) = a(x-x_0)^2$.
\end{itemize}

\textbf{Signe}
\begin{itemize}
	\item Si $\Delta > 0$, $P(x)$ est du signe de $a$ à l'extérieur des racines
	(c'est-à-dire sur $]-\infty,x_1] \cup [x_2,+\infty[$). 
	
	\item Si $\Delta = 0$ ou $\Delta < 0$, $P(x)$ est du signe de $a$ sur $\Rr$.
\end{itemize}


%-----------------------------------------
\subsection{Exponentielle}
% \textbf{Exponentielle}

La fonction \defi{exponentielle}\index{exponentielle} : $\exp : \Rr \to   ]0,+\infty[$.
Pour $x \in \Rr$ on note aussi $e^x$ pour $\exp x$.

\myfigure{0.7}{
	\tikzinput{fig_usuelles03}
}


La fonction exponentielle vérifie les propriétés suivantes :
\begin{itemize}
	% \item \myboxinline{$\exp(\ln x) = x$ pour tout $x >0$} et
	% \myboxinline{$\ln(\exp x) = x$ pour tout $x \in \Rr$}
	
	\item $\exp(a+b) = \exp(a) \times \exp(b)$
	
	\item $\exp(nx) = (\exp x)^n$
	
	\item $\exp : \Rr \to   ]0,+ \infty[$ est une fonction continue, strictement croissante,
	
	\item $\lim_{x\to-\infty} \exp x = 0$,
	
	\item $\lim_{x\to +\infty} \exp = +\infty$,
	
	\item $\exp(1) = e \simeq 2,718\ldots$
\end{itemize}


%-----------------------------------------
\subsection{Logarithme}

% \textbf{Logarithme}
\index{logarithme}

Le \defi{logarithme népérien} $\ln : ]0,+\infty[ \to \Rr$.

\myfigure{0.7}{
	\tikzinput{fig_usuelles02}
}


\begin{itemize}
	\item $\ln (a \times b) = \ln a + \ln b$ (pour tout $a,b >0$),
	\item $\ln(\frac 1 a) = - \ln a$,
	\item $\ln(a^n) = n \ln a$, (pour tout $n \in \Nn$),
	\item $\ln$ est une fonction continue, strictement croissante,
	\item $\ln'(x) = \frac 1x$ pour tout $x>0$,
	\item $\ln(1) = 0$, $\ln(e)=1$,
	\item $\lim_{x\to +\infty} \ln x = +\infty$,
	\item $\lim_{x\to 0^+} \ln x = -\infty$.
\end{itemize}

\medskip

Lien logarithme/exponentielle :

\mybox{$\exp(\ln x) = x$ pour tout $x >0$} 

\mybox{$\ln(\exp x) = x$ pour tout $x \in \Rr$}

Pour $x \in \Rr$ et $y>0$ :
\mybox{$y = \exp(x) \iff x = \ln(y)$}


\end{multicols}


\end{document}	