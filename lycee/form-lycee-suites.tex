\documentclass[10pt,class=article,crop=false]{standalone}
\usepackage{../exo7formules}


\begin{document}

%%%%%%%%%%%%%%%%%%%%%%%%%%%%%%%%%%%%%%%%%%
\section{Suites}

\begin{multicols}{2}
	
%-----------------------------------------
\subsection{Suites arithmétiques}

Soit $u_0$ un \defi{terme initial} et $a$ une \defi{raison}, la \defi{suite arithmétique} $(u_n)$ est définie par:
$$u_n = a n + u_0$$

La formule de récurrence est :
$$u_{n+1} = u_n + a$$

Ainsi la raison se calcule par $a = u_{n+1} - u_n$ pour n'importe quel $n \ge 0$. 
Par exemple $a = u_1 - u_0$.

Somme des $n$ premiers entiers :
$$1+2+3+\cdots+n = \frac{n(n+1)}{2}$$


%-----------------------------------------
\subsection{Suites géométriques}

Soit $u_0$ un \defi{terme initial} et $q$ une \defi{raison}, la \defi{suite géométrique} $(u_n)$ est définie par:
$$u_n = u_0 \cdot q^n$$

La formule de récurrence est :
$$u_{n+1} = q \cdot u_n$$

Ainsi la raison se calcule par $q = \frac{u_{n+1}}{u_n}$ pour n'importe quel $n \ge 0$ (avec $u_n \neq 0$).
Par exemple $q = \frac{u_1}{u_0}$.

Somme des $n$ premiers termes d'une suite géométrique  de raison $q$ :
$$1+q+q^2+\cdots+q^n = \frac{1-q^{n+1}}{1-q}$$


\end{multicols}

\end{document}	