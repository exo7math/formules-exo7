\documentclass[10pt,class=article,crop=false]{standalone}
\usepackage{../exo7formules}


\begin{document}

%%%%%%%%%%%%%%%%%%%%%%%%%%%%%%%%%%%%%%%%%%
\section{Droites et vecteurs}

\begin{multicols}{2}
	
%-----------------------------------------
\subsection{Géométrie}

La \defi{distance} entre $A(x_A,y_A)$ et $B(x_B,y_B)$ est $AB = \sqrt{(x_B-x_A)^2 + (y_B-y_A)^2}$.

Le \defi{milieu} $M$ de $[AB]$ a pour coordonnées $\left( \frac{x_A+y_A}{2}, \frac{x_B+y_B}{2} \right)$.


%-----------------------------------------
\subsection{Vecteurs}

Si $A(x_A,y_A)$ et $B(x_B,y_B)$ sont deux points du plan, le \defi{vecteur} de $A$ à $B$ est $\vec{AB} = (x_B-x_A,y_B-y_A)$.

La \defi{norme} d'un vecteur $\vec v = (x,y)$ est $\|\vec v \| = \sqrt{x^2+y^2}$. 



Soient $\vec u = (x,y)$ et $\vec v = (x',y')$ deux vecteurs et $k \in \Rr$.

Addition. $\vec u + \vec v = (x+x',y+y')$.

Multiplication par un scalaire. $k\vec u = (kx,ky)$.

Opposé. $-\vec u = (-x,-y)$.

Produit scalaire. $\vec u \cdot \vec v = x x' + y y'$.

$\vec u \cdot \vec v = \| \vec u \| \| \vec v \| \cos \theta$ où $\theta$ est l'angle entre $\vec u$ et $\vec v$.

$$\| \vec v \|^2 = \vec v \cdot \vec v$$

Vecteurs orthogonaux. Deux vecteurs $\vec u$ et $\vec v$ sont \defi{orthogonaux} si $\vec u \cdot \vec v = 0$.

Si $\vec u = (a,b)$ alors $\vec n = (-b,a)$ est un vecteur orthogonal à $\vec v$.


Vecteurs colinéaires. Deux vecteurs $\vec u$ et $\vec v$ sont \defi{colinéaires} (ou \defi{parallèles}) si 
il existe $k \in \Rr$ tel que $\vec u = k\vec v$ (ou $\vec v = k\vec u$).
$\vec u = (a,b)$ et $\vec v = (c,d)$ sont colinéaires
si et seulement si $ad - bc = 0$.

On note $\det (\vec u, \vec v) = \begin{vmatrix}
a & b \\
c & d
\end{vmatrix}
= ad - bc$ le \defi{déterminant}.


%-----------------------------------------
\subsection{Droites}

Équation cartésienne. 
$$ax+by+c=0$$
avec $a$, $b$ et $c$ des nombres réels (et $a$ et $b$ non tous les deux nuls).

Vecteur directeur. $\vec v = (-b,a)$ est un vecteur directeur de la droite d'équation $ax+by+c=0$

Vecteur normal. $\vec n = (a,b)$ est un vecteur normal à la droite d'équation $ax+by+c=0$.


Équation réduite.
$$y = ax+b$$
avec $a$ et $b$ des nombres réels (et $a$ non nul).
\begin{itemize}
    \item $a$ est la \defi{pente} ou \defi{coefficient directeur},
    \item $b$ est l'\defi{ordonnée à l'origine}.
\end{itemize}

Si $A(x_A,y_A)$ et $B(x_B,y_B)$ sont deux points de la droite, alors $a = \frac{y_B-y_A}{x_B-x_A}$.
On trouve ensuite l'ordonnée à l'origine à l'aide de la relation $y_A = a x_A+b$.

\textbf{Équation paramétrique.}
$$\begin{cases}
x = x_A + t \alpha \\
y = y_A + t \beta
\end{cases}$$
avec $A(x_A,y_A)$ un point de la droite et $\vec v = (\alpha,\beta)$ un vecteur directeur de la droite.


\textbf{Droites parallèles.}
Deux droites sont parallèles si et seulement si leurs vecteurs directeurs sont colinéaires.
Ainsi $ax+by+c=0$ et $ax'+by'+c'=0$ sont deux droites parallèles
si et seulement si 
$\det( \begin{pmatrix} -b \\ a\end{pmatrix},  \begin{pmatrix} -b' \\ a'\end{pmatrix}) = 0$ c'est-à-dire 
$$ab' - a'b=0$$

\textbf{Droites perpendiculaires.}
Deux droites sont perpendiculaires si et seulement si leurs vecteurs directeurs sont orthogonaux.
Ainsi $ax+by+c=0$ et $ax'+by'+c'=0$ sont deux droites perpendiculaires
si et seulement si 
$$aa' + bb'=0.$$ 

\end{multicols}

\end{document}	